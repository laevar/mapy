\documentclass[a4paper,10pt,DIV15]{scrartcl}
\usepackage[psamsfonts]{amssymb}
\usepackage{amsmath}
%\usepackage{latexsym}
\usepackage{theorem}


\usepackage{fontspec,xunicode,xltxtra}
\usepackage[ngerman]{babel}
\selectlanguage{ngerman}


\usepackage[svgnames,hyperref]{xcolor} %color definitions
%\usepackage{tikz}
%\usetikzlibrary{shadows}
%\usetikzlibrary{fit}
%\usetikzlibrary{shapes}
%\usetikzlibrary{backgrounds}


\usepackage{mcode}
%\usepackage{pstricks,pst-node,pst-text,pst-3d}
\parindent0cm % Abs�tze nicht einr�cken 

%---- neue Umgebung f�r Aufgaben
\theoremstyle{break}
\theoremheaderfont{\Large \bf}
\theorembodyfont{\normalfont}


% Definieren einer neuen Farbe
\definecolor{light-gray}{gray}{.9}

\newcounter{zaehler}     % neuen Z�hler einf�hren
\stepcounter{zaehler}    % Z�hler einen hochz�hlen

\newenvironment{aufg}%
%---- Header
{\begin{samepage}
\colorbox{light-gray}{                         % Box in gray
 \makebox[\textwidth]{                           % Box in linewidth
\textbf{Aufgabe} \arabic{zaehler} :}}\\[0.1cm]       % Header
%\begin{minipage}{0.5cm} \end{minipage}    % Insert 0.5cm
\begin{minipage}{\textwidth}}
%-----  foot
{\end{minipage} \nopagebreak %\begin{minipage}{1cm} \end{minipage}
\\[0.1cm] 
%\begin{minipage}{0.1cm} \end{minipage} 
%\hrulefill \begin{minipage}{1cm} \end{minipage}\\[1cm]  
\stepcounter{zaehler}                           % increase counter
 \end{samepage}%
}

%-------------------------------------------------------------------------------
\begin{document}
%-------------------------------------------------------------------------------

%--------------------------------------------------- Header
\begin{center}
\textbf{\LARGE Einf\"uhrung in MATLAB }\\
\end{center}
\begin{minipage}{6cm}
Dr. J. Schulz\\
\end{minipage}\hfill
\begin{minipage}{2cm}
\textbf{Einheit 1}\\
%30.07.2007
\end{minipage}\\[1cm]
%-----------------------------------------------------------------------------------
%-----------------------------------------------------------------------------------
\emph{Hinweis:} Lösen sie alle Aufgaben wenn möglich ohne Verwendung von
Schleifen!\\
 
\begin{aufg}[0]
Starten Sie das Programm
\mcode{plot_poly}. Der Graph welchen Polynoms wird dargestellt?
Erklären Sie das Programm \mcode{ausw_poly2}.
\end{aufg}

\begin{aufg}[0]
Stellen Sie das Polynom 
\[ p(x)= x^5-4x^4 -10x^3 +40x^2 +9x -36\]
 grafisch dar. Wo sind die Nullstellen?\\
\textit{Hinweis:} Die Nullstellen sind die Eigenwerte (\mcode{eig}) der Begleitmatrix
\[ \left( \begin{array} {ccccccc}
 0 & 0 & 0 & 0 & 36\\
1 & 0  & 0 & 0  & -9\\ 
0 & 1  & 0 & 0  & -40\\ 
 0 & 0& 1 & 0 &  10\\
 0 & 0 & 0 & 1  &  4 \\ 
\end{array} \right) \]
\end{aufg}

%--------------------------------------------------
\begin{aufg}[0]
Berechnen Sie $\sum_{j=2}^{1000} \frac{1}{log(j)j}$ und
  $\sum_{j=1}^{1000} \frac{1}{j}$.
\end{aufg}
%--------------------------------------------------
\begin{aufg}[0]
Welchen Grenzwert hat $\frac{1}{\pi^2} \sum_{j=1}^\infty
\frac{1}{j^2}$?
\end{aufg}
%--------------------------------------------------
\begin{aufg}[0]
Betrachten Sie die Mandelbrot-Menge in $[-1,-0.4]\times
[-0.6,0]$!
\end{aufg}
%--------------------------------------------------
\begin{aufg}[0]
 Interpretieren Sie das Ergebnis der Eingabe 
\begin{lstlisting}
a=100:2:200;
b=[1 4 10];
a(b)
\end{lstlisting}
\end{aufg}
%--------------------------------------------------
\newpage
\begin{aufg}[0]
\label{vander}
Erzeugen Sie die $(100 \times 100)$ - Matrix 
\[ \left( \begin{array} {ccccccc}
 2 & -1 &  & &   0 \\
-1 & 2  & -1 &    & \\ 
   & \ddots & \ddots & \ddots   &\\
   & &  -1 & 2  & -1  \\ 
0 &  &    & -1 & 2 \\
\end{array} \right) \]
und berechnen Sie ihre Determinante.
\end{aufg}

%--------------------------------------------------
\begin{aufg}[0]
Erzeugen Sie eine Hilbert Matrix der Größe $50$. (Befehl
  \lstinline!hilb!) Addieren
  Sie die Einträge der dritten Spalte! 
\end{aufg}

\begin{aufg}[0]
Schreiben Sie eine Funktion, die zu einem gegebenen Vektor
  dessen Durchschnitt berechnet und zurückgibt.
\end{aufg}

\begin{aufg}[0]
Schreiben Sie eine Funktion, die zu einem gegebenen Vektor
  $x=(x_1, \dots ,x_n)$ die Vandermonde-Matrix
\[ V:= \left(\begin{array}{ccccc} 
1 & x_1 & x_1^2 & \hdots & x_1^{n-1}\\
1 & x_2 & x_2^2 & \hdots & x_2^{n-1}\\
\vdots & \vdots & \vdots & \vdots & \vdots\\
1 & x_n & x_n^2 & \hdots & x_n^{n-1}\\
\end{array} \right)  \]
berechnet und zurückgibt. \\

{\it Hinweis:} \mcode{V=A.^B}. mit
\[
A:= \left(\begin{array}{ccccc} 
x_1 & x_1 & x_1 & \hdots & x_1\\
x_2 & x_2 & x_2 & \hdots & x_2\\
\vdots & \vdots & \vdots & \vdots & \vdots\\
x_n & x_n & x_n & \hdots & x_n\\ \end{array} 
\right) , \quad B:= 
 \left(\begin{array}{ccccc} 
0 & 1 & 2 & \hdots & n-1\\
0 & 1 & 2 & \hdots & n-1\\
\vdots & \vdots & \vdots & \vdots & \vdots\\
0 & 1 & 2 & \hdots & n-1\\
\end{array} 
\right)
\]
\end{aufg}

\end{document}
