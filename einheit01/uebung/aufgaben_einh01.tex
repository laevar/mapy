\documentclass[a4paper,10pt,DIV15]{scrartcl}
\usepackage[psamsfonts]{amssymb}
\usepackage{amsmath}
%\usepackage{latexsym}
\usepackage{theorem}


\usepackage{fontspec,xunicode,xltxtra}
\usepackage[ngerman]{babel}
\selectlanguage{ngerman}


\usepackage[svgnames,hyperref]{xcolor} %color definitions
%\usepackage{tikz}
%\usetikzlibrary{shadows}
%\usetikzlibrary{fit}
%\usetikzlibrary{shapes}
%\usetikzlibrary{backgrounds}


\usepackage{mcode}
%\usepackage{pstricks,pst-node,pst-text,pst-3d}
\parindent0cm % Abs�tze nicht einr�cken 

%---- neue Umgebung f�r Aufgaben
\theoremstyle{break}
\theoremheaderfont{\Large \bf}
\theorembodyfont{\normalfont}


% Definieren einer neuen Farbe
\definecolor{light-gray}{gray}{.9}

\newcounter{zaehler}     % neuen Z�hler einf�hren
\stepcounter{zaehler}    % Z�hler einen hochz�hlen

\newenvironment{aufg}%
%---- Header
{\begin{samepage}
\colorbox{light-gray}{                         % Box in gray
 \makebox[\textwidth]{                           % Box in linewidth
\textbf{Aufgabe} \arabic{zaehler} :}}\\[0.1cm]       % Header
%\begin{minipage}{0.5cm} \end{minipage}    % Insert 0.5cm
\begin{minipage}{\textwidth}}
%-----  foot
{\end{minipage} \nopagebreak %\begin{minipage}{1cm} \end{minipage}
\\[0.1cm] 
%\begin{minipage}{0.1cm} \end{minipage} 
%\hrulefill \begin{minipage}{1cm} \end{minipage}\\[1cm]  
\stepcounter{zaehler}                           % increase counter
 \end{samepage}%
}

%-------------------------------------------------------------------------------
\begin{document}
%-------------------------------------------------------------------------------

%--------------------------------------------------- Header
\begin{center}
\textbf{\LARGE Einf\"uhrung in MATLAB }\\
\end{center}
\begin{minipage}{6cm}
Dr. J. Schulz\\
\end{minipage}\hfill
\begin{minipage}{2cm}
\textbf{Einheit 1}\\
%30.07.2007
\end{minipage}\\[1cm]
%-----------------------------------------------------------------------------------
%-----------------------------------------------------------------------------------
\emph{Hinweis:} Alle Aufgaben (bis auf Aufgabe 7) sollen ohne Verwendung von
Schleifen gel\"ost werden.\\
 

\begin{aufg}
Finde die Lösung $x$ von $Ax=b$ mit 
\[ A:=\left( \begin{array}{cccc} 
2 & 3 & 4 & 5 \\ 
1 &    1 &    1 &    1\\
1 &    0 &    1 &    0 \\
9 &     3 &    2 &    1
\end{array} \right) , \qquad b:=\left( \begin{array}{c} 
14\\
     4\\
     2\\
    15
\end{array} \right) . \] 
\end{aufg}
%--------------------------------------------------
\begin{aufg}
Finde die Lösung $x$ von $Ax=b$ mit 
{ \[ A:=\left( \begin{array}{ccc} 
1 & 2 & 3 \\ 
4 &    5 &    6\\
7 &    8 &    9  
\end{array} \right) , \qquad b:=\left( \begin{array}{c} 
6\\
     15\\
     24
\end{array} \right) . \] }
\end{aufg}
%--------------------------------------------------
\begin{aufg}
Sind die folgenden Vektoren linear unabhängig?
{ \[
v_1=
\left( \begin{array}{c} 0 \\ 1 \\ 0 \\ 1 \end{array} \right), 
\ v_2=
\left( \begin{array}{c} 1 \\ 2 \\ 3 \\ 4 \end{array} \right),
\ v_3=
 \left( \begin{array}{c} 1 \\ 0 \\ 1 \\  0\end{array} \right),
\ v_4=
 \left( \begin{array}{c} 0 \\ 0 \\ 1 \\ 1 \end{array} \right)
. \] }
\end{aufg}
%--------------------------------------------------
\begin{aufg}
Geben Sie die folgende Zeile ein: 
\begin{lstlisting}
>>  x=1e-15; ((1+x)-1)/x
\end{lstlisting} 
Wie interpretieren Sie das Ergebnis? 
(Testen Sie auch \lstinline!x=1e-16!!)
\end{aufg}
%--------------------------------------------------
\begin{aufg}
Berechnen Sie $\sum_{j=2}^{1000} \frac{1}{log(j)j}$ und
  $\sum_{j=1}^{1000} \frac{1}{j}$.
\end{aufg}
%--------------------------------------------------
\begin{aufg}
Welchen Grenzwert hat $\frac{1}{\pi^2} \sum_{j=1}^\infty
\frac{1}{j^2}$?
\end{aufg}
%--------------------------------------------------
\begin{aufg}
Betrachten Sie die Mandelbrot-Menge in $[-1,-0.4]\times
[-0.6,0]$!
\end{aufg}
\newpage
%--------------------------------------------------
\begin{aufg}
 Interpretieren Sie das Ergebnis der Eingabe 
\begin{lstlisting}
>> a=100:2:200; b=[1 4 10]; a(b)
\end{lstlisting}
\end{aufg}
%--------------------------------------------------
\begin{aufg}
Erzeugen Sie die $(100 \times 100)$ - Matrix 
{ 
\[ \left( \begin{array} {ccccccc}
 2 & -1 &  & &   0 \\
-1 & 2  & -1 &    & \\ 
   & \ddots & \ddots & \ddots   &\\
   & &  -1 & 2  & -1  \\ 
0 &  &    & -1 & 2 \\
\end{array} \right) \]
}
und berechnen Sie ihre Determinante.
\end{aufg}
%--------------------------------------------------
\begin{aufg}
Zerlegen Sie das Intervall $[0,1]$ durch 
  \lstinline!0:(1/101):1!. Berechnen Sie mit Hilfe von 
Finiten Differenzen eine approx. Lösung von
{
\begin{eqnarray*}
-u''(x) & = & 1, \quad x \in (0,1)\\
u(0) & = & u(1) =0
\end{eqnarray*}}
\vspace*{-0.5cm}
\end{aufg}
%--------------------------------------------------
\begin{aufg}
Erzeugen Sie eine Hilbert Matrix der Größe $50$. (Befehl
  \lstinline!hilb!) Addieren
  Sie die Einträge der dritten Spalte! 
\end{aufg}
%--------------------------------------------------
\begin{aufg}
Berechnen Sie die Frobenius-Norm 
\[ \|A \|_F := \sqrt{ \sum_{i,j=1}^n a_{ij}^2 }, \quad A=(a_{ij}) \in
\mathbb{R}^{n \times n}  \]
der Vandermode Matrix 
\lstinline!vander(0:0.02:1)! 
\end{aufg}
%--------------------------------------------------
\begin{aufg}
Ändern Sie in Aufgabe 10 die rechte Seite $1$ in
  $\sin(4 \pi x)$ und berechnen Sie eine Näherungslösung. 
\end{aufg}
%--------------------------------------------------
\begin{aufg}
Berechnen Sie die Eigenwerte und Eigenvektoren der Matrix
\[ A = \left( \begin{array}{cccc}
30 & 1 & 2 & 3\\
4 & 15 & -4 & -2\\
-1 & 0 & 3 & 5\\
-3 & 5 & 0 & -1 \end{array}
\right) \]
Bestimmen Sie auch die $QR$-Zerlegung.
\end{aufg}

\begin{aufg}
Sei \lstinline!A=hilb(n)! und \lstinline!x=ones(n,1)!. Berechnen Sie
  für $n=5$ und $n=15$ den Vektor $b=A*x$, \lstinline!norm(x-A\b)! und die Kondition
  von $A$. Was stellen Sie fest? Erklären Sie das Ergebnis!
\end{aufg}
\end{document}
