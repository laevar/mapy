%----------------------------------------------------------------
%  Einf�hrung in MATLAB
% 
% Blockpraktikum Sommer 2007
% (2 Wochen)
%----------------------------------------------------------------

\documentclass[12pt]{article}
\usepackage{german, a4}
\usepackage{amsmath}
\usepackage{latexsym}
\usepackage{amssymb}
\usepackage[ansinew]{inputenc}
\usepackage{exscale}
\usepackage{theorem}
\usepackage[dvips]{graphics}
%\usepackage{pstricks,pst-node,pst-text,pst-3d}
\usepackage{color}
\usepackage{times}                  % Postscript font!!!
\parindent0cm % Abs�tze nicht einr�cken 

%-- Selbstdefinierte Befehle und Parametereinstellungen ------------------------
\setlength{\headheight}{1.1\baselineskip}
\setlength{\textheight}{22cm}       
\setlength{\textwidth}{15cm}
\setlength{\topmargin}{-2cm} %oberer Rand bis Kopfzeile angehoben
\setlength{\oddsidemargin}{0.5cm} %linker Rand f"ur ungerade Seiten


%---- neue Umgebung f�r Aufgaben
\theoremstyle{break}
\theoremheaderfont{\Large \bf}
\theorembodyfont{\normalfont}


% Definieren einer neuen Farbe
\definecolor{light-gray}{gray}{.9}

\newcounter{zaehler}     % neuen Z�hler einf�hren
\stepcounter{zaehler}    % Z�hler einen hochz�hlen




\newenvironment{aufg}%
%---- Header
{\begin{samepage}
\colorbox{light-gray}{                         % Box in gray
 \makebox[\textwidth]{                           % Box in linewidth
\bf \color{red} Aufgabe \arabic{zaehler} :}}\\[0.2cm]       % Header
\begin{minipage}{0.5cm} \end{minipage} \hfill   % Insert 0.5cm
\begin{minipage}{13.5cm}}                         
%-----  foot
{\end{minipage} \nopagebreak \begin{minipage}{1cm} \end{minipage}
\\[0.2cm] 
\begin{minipage}{0.1cm} \end{minipage} 
\hrulefill \begin{minipage}{1cm} \end{minipage}\\[1cm]  
\stepcounter{zaehler}                           % increase counter
 \end{samepage}%
}


%-------------------------------------------------------------------------------
\begin{document}
%-------------------------------------------------------------------------------

%--------------------------------------------------- Header
\begin{center}
{\LARGE \bf Einf\"uhrung in MATLAB }\\
%{\LARGE \bf (Blockveranstaltung)}\\
%SS 2007\\[0.8cm]
\end{center}
\begin{minipage}{6cm}
Dr. G. Rapin,\\
%Th. Wassong\\
\end{minipage} \hfill 
\begin{minipage}{2cm}
{\bf Einheit 2}\\
%31.07.2007
\end{minipage}\\[1cm]

%----------------------------
% Folie 
%----------------------------
\begin{aufg}
Starten Sie das Programm
\verb+plot_poly+. Der Graph welchen Polynoms wird dargestellt?
Erkl�ren Sie das Programm \verb+ausw_poly2+.
\end{aufg}

%----------------------------
% Folie 
%----------------------------
\begin{aufg}
Stellen Sie das Polynom 
\[ p(x)= x^5-4x^4 -10x^3 +40x^2 +9x -36\]
 grafisch dar. Wo sind die Nullstellen?
\end{aufg}

%----------------------------
% Folie 
%----------------------------
\begin{aufg}
Starten Sie das Programm \verb+randwertaufgabe+. Berechnen
  Sie die L�sung der Randwert-Aufgabe f�r die rechte Seite $f=16 \pi^2 \sin( 4 \pi
  x)$. Erraten Sie die L�sung!
\end{aufg}

%----------------------------
% Folie 
%----------------------------
\begin{aufg}
Schreiben Sie eine Funktion, die zu einem gegebenen Vektor
  deren Durchschnitt berechnet und zur�ckgibt.
\end{aufg}
\newpage
\begin{aufg}
Schreiben Sie eine Funktion, die zu einem gegebenen Vektor
  $x=(x_1, \dots ,x_n)$ die Vandermonde-Matrix
{\tiny \[ V:= \left(\begin{array}{ccccc} 
1 & x_1 & x_1^2 & \hdots & x_1^{n-1}\\
1 & x_2 & x_2^2 & \hdots & x_2^{n-1}\\
\vdots & \vdots & \vdots & \vdots & \vdots\\
1 & x_n & x_n^2 & \hdots & x_n^{n-1}\\
\end{array} \right)  \]}
berechnet und zur�ckgibt. \\

\underline{Hinweis:} \verb+V=A.^B+. mit
{\tiny \[
A:= \left(\begin{array}{ccccc} 
x_1 & x_1 & x_1 & \hdots & x_1\\
x_2 & x_2 & x_2 & \hdots & x_2\\
\vdots & \vdots & \vdots & \vdots & \vdots\\
x_n & x_n & x_n & \hdots & x_n\\ \end{array} 
\right) , \quad B:= 
 \left(\begin{array}{ccccc} 
0 & 1 & 2 & \hdots & n-1\\
0 & 1 & 2 & \hdots & n-1\\
\vdots & \vdots & \vdots & \vdots & \vdots\\
0 & 1 & 2 & \hdots & n-1\\
\end{array} 
\right)
\]}
\end{aufg}

\begin{aufg}
L�sen Sie n\"aherungsweise die Fixpunktgleichung
\[ x_f=exp(-x_f). \]
\end{aufg}

\begin{aufg}
Berechnen Sie eine Nullstelle von 
\[ f(x) = \cos^2(x) -x. \]
\end{aufg}
%
%-----------------------------------------------------------
%
\begin{aufg}
Schreiben Sie eine Funktion, die f�r $n \in \mathbb{N}$ die
  Hilbert-Matrix $H=(h_{ij})_{i,j=1}^n$ mit $h_{ij}=\frac{1}{i+j-1}$
  berechnet. Berechnen Sie $H^{-1}$ f�r $n=4$. 
\end{aufg}
%
%-----------------------------------------------------------
%
\newpage
\begin{aufg}
 Berechnen Sie die Nullstellen von 
\[ x^2-2, \quad x^2-2x+1, \quad x^2 -4x +10. \]
\end{aufg}
%
%-----------------------------------------------------------
%
\begin{aufg}
Die Fibonacci-Folge ist definiert durch
\[ f_1:=1, \quad f_2:=1, \quad f_{k+2}:=f_{k+1}+f_k, \ k \in \mathbb{N}.
\]
Schreiben Sie ein Programm, das 
{\tiny \[ g_{k}:= \frac{f_{k+1}}{f_k}, \quad k \in \mathbb{N} \]}
berechnet. Stoppen Sie, falls {\tiny $|g_{k}-g_{k+1}| \leq TOL$}. 
Geben Sie f�r $TOL=10^{-3}$ und  $TOL=10^{-4}$ das entsprechende $k$
und das entsprechende $g_k$ an.\\

{\it Hinweis}: Benutzen Sie eine \verb+while+-Schleife.
\end{aufg}
%
%-----------------------------------------------------------
\begin{aufg}
Seien $y_1,y_2$ zwei Punkte im $\mathbb{R}^2$. Wir betrachten die Strecke mit
Endpunkten $y_1$ und $y_2$. Wir ersetzen  diese Strecke durch 4 Strecken 
$\overline{y_1 z_1}$, $\overline{z_1 z_2}$, $\overline{z_2 z_3}$,
$\overline{z_3 y_2}$ mit Endpunkten $z_1=\frac23 y_1 + \frac13 y_2$,
$z_3=\frac13 y_1 + \frac23 y_2$ und 
\[ z_2 = \frac{\sqrt{3}}{6} \left( \begin{array}{cc}
0 & 1 \\ -1 & 0 \\
\end{array} \right)
(y_1 - y_2) + \frac12 (y_1 + y_2). \]
Analog zum Beispiel des Sierpinski-Dreiecks soll jede neue Teilstrecke
wiederum mittels der gleichen Prozedur durch 4 Strecken ersetzt werden. 
Schreiben Sie ein Programm, dass
diese Prozedur $k$-mal wiederholt und das Ergebnis plottet.
\end{aufg}

%-------------------------------------------------------------------------------
\end{document}
%-------------------------------------------------------------------------------
