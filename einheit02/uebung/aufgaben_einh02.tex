\documentclass[a4paper,10pt,DIV15]{scrartcl}
\usepackage[psamsfonts]{amssymb}
\usepackage{amsmath}
%\usepackage{latexsym}
\usepackage{theorem}


\usepackage{fontspec,xunicode,xltxtra}
\usepackage[ngerman]{babel}
\selectlanguage{ngerman}


\usepackage[svgnames,hyperref]{xcolor} %color definitions
%\usepackage{tikz}
%\usetikzlibrary{shadows}
%\usetikzlibrary{fit}
%\usetikzlibrary{shapes}
%\usetikzlibrary{backgrounds}


\usepackage{mcode}
%\usepackage{pstricks,pst-node,pst-text,pst-3d}
\parindent0cm % Abs�tze nicht einr�cken 

%---- neue Umgebung f�r Aufgaben
\theoremstyle{break}
\theoremheaderfont{\Large \bf}
\theorembodyfont{\normalfont}


% Definieren einer neuen Farbe
\definecolor{light-gray}{gray}{.9}

\newcounter{zaehler}     % neuen Z�hler einf�hren
\stepcounter{zaehler}    % Z�hler einen hochz�hlen

\newenvironment{aufg}%
%---- Header
{\begin{samepage}
\colorbox{light-gray}{                         % Box in gray
 \makebox[\textwidth]{                           % Box in linewidth
\textbf{Aufgabe} \arabic{zaehler} :}}\\[0.1cm]       % Header
%\begin{minipage}{0.5cm} \end{minipage}    % Insert 0.5cm
\begin{minipage}{\textwidth}}
%-----  foot
{\end{minipage} \nopagebreak %\begin{minipage}{1cm} \end{minipage}
\\[0.1cm] 
%\begin{minipage}{0.1cm} \end{minipage} 
%\hrulefill \begin{minipage}{1cm} \end{minipage}\\[1cm]  
\stepcounter{zaehler}                           % increase counter
 \end{samepage}%
}

%-------------------------------------------------------------------------------
\begin{document}
%-------------------------------------------------------------------------------

%--------------------------------------------------- Header
\begin{center}
\textbf{\LARGE Einf\"uhrung in MATLAB }\\
\end{center}
\begin{minipage}{6cm}
Dr. J. Schulz\\
\end{minipage}\hfill
\begin{minipage}{2cm}
\textbf{Einheit 2}\\
%30.07.2007
\end{minipage}\\[1cm]

Das Ergebnis der Rechnung ist Abbildung 1. Warum werden die Ergebnisse
f ̈r negative x groß? Warum sehen sie f ̈r positive x klein aus? Sind sie
u
u
wirklich klein? Oder kann man sie nur nicht sehen? Was passiert, wenn man
sich nur positive x ansieht? Man ver ̈ndere das Programm an den richtigen
a
Stellen, um sich die Lage genauer anzusehen.


\begin{aufg}
Finde die Lösung $x$ von $Ax=b$ mit 
\[ A:=\left( \begin{array}{cccc} 
2 & 3 & 4 & 5 \\ 
1 &    1 &    1 &    1\\
1 &    0 &    1 &    0 \\
9 &     3 &    2 &    1
\end{array} \right) , \qquad b:=\left( \begin{array}{c} 
14\\
     4\\
     2\\
    15
\end{array} \right) . \] 
\end{aufg}
%--------------------------------------------------
\begin{aufg}
Finde die Lösung $x$ von $Ax=b$ mit 
{ \[ A:=\left( \begin{array}{ccc} 
1 & 2 & 3 \\ 
4 &    5 &    6\\
7 &    8 &    9  
\end{array} \right) , \qquad b:=\left( \begin{array}{c} 
6\\
     15\\
     24
\end{array} \right) . \] }
\end{aufg}
%--------------------------------------------------
\begin{aufg}
Sind die folgenden Vektoren linear unabhängig?
{ \[
v_1=
\left( \begin{array}{c} 0 \\ 1 \\ 0 \\ 1 \end{array} \right), 
\ v_2=
\left( \begin{array}{c} 1 \\ 2 \\ 3 \\ 4 \end{array} \right),
\ v_3=
 \left( \begin{array}{c} 1 \\ 0 \\ 1 \\  0\end{array} \right),
\ v_4=
 \left( \begin{array}{c} 0 \\ 0 \\ 1 \\ 1 \end{array} \right)
. \] }
\end{aufg}


\begin{aufg}
Schreiben sie das Programm \mcode{randwertaufgabe} um in eine Funktion welche als Inputparameter 
den Parameter $n$ erhält. Die Funktion soll prüfen ob der Parameter n in dem Bereich 20-200 liegt und
falls nicht das programm abbrechen. Das Resultat der Berechnung soll als Vektor zurückgeben werden.

\emph{Hinweis:} das Abbrechen des Programms kann mit \mcode{return} erreicht werden.
\end{aufg}


\begin{aufg}
Lösen Sie n\"aherungsweise die Fixpunktgleichung
\[ x_f=e^{(-x_f)}. \]
\end{aufg}

\begin{aufg}
Berechnen Sie eine Nullstelle von 
\[ f(x) = \cos^2(x) -x. \]
\end{aufg}

\begin{aufg}
Schreiben Sie eine Funktion, die für $n \in \mathbb{N}$ die
  Hilbert-Matrix $H=(h_{ij})_{i,j=1}^n$ mit $h_{ij}=\frac{1}{i+j-1}$
  berechnet. Berechnen Sie $H^{-1}$ für $n=4$. 
\end{aufg}
%
%-----------------------------------------------------------
%
\newpage
\begin{aufg}
 Berechnen Sie die Nullstellen von 
\[ x^2-2, \quad x^2-2x+1, \quad x^2 -4x +10. \]
\end{aufg}
%
%-----------------------------------------------------------
%
\begin{aufg}
Die Fibonacci-Folge ist definiert durch
\[ f_1:=1, \quad f_2:=1, \quad f_{k+2}:=f_{k+1}+f_k, \ k \in \mathbb{N}.
\]
Schreiben Sie ein Programm, das 
\[ g_{k}:= \frac{f_{k+1}}{f_k}, \quad k \in \mathbb{N} \]
berechnet. Stoppen Sie, falls $|g_{k}-g_{k+1}| \leq TOL$.
Geben Sie für $TOL=10^{-3}$ und  $TOL=10^{-4}$ das entsprechende $k$
und das entsprechende $g_k$ an.\\

{\it Hinweis}: Benutzen Sie eine \mcode{while}-Schleife.
\end{aufg}
%
%-----------------------------------------------------------
\begin{aufg}
Seien $y_1,y_2$ zwei Punkte im $\mathbb{R}^2$. Wir betrachten die Strecke mit
Endpunkten $y_1$ und $y_2$. Wir ersetzen  diese Strecke durch 4 Strecken 
$\overline{y_1 z_1}$, $\overline{z_1 z_2}$, $\overline{z_2 z_3}$,
$\overline{z_3 y_2}$ mit Endpunkten $z_1=\frac23 y_1 + \frac13 y_2$,
$z_3=\frac13 y_1 + \frac23 y_2$ und 
\[ z_2 = \frac{\sqrt{3}}{6} \left( \begin{array}{cc}
0 & 1 \\ -1 & 0 \\
\end{array} \right)
(y_1 - y_2) + \frac12 (y_1 + y_2). \]
Analog zum Beispiel des Sierpinski-Dreiecks soll jede neue Teilstrecke
wiederum mittels der gleichen Prozedur durch 4 Strecken ersetzt werden. 
Schreiben Sie ein Programm, dass
diese Prozedur $k$-mal wiederholt und das Ergebnis plottet.
\end{aufg}

%--------------------------------------------------
\begin{aufg}
Zerlegen Sie das Intervall $[0,1]$ durch 
  \lstinline!0:(1/101):1!. Berechnen Sie mit Hilfe von 
Finiten Differenzen eine approx. Lösung von
{
\begin{eqnarray*}
-u''(x) & = & 1, \quad x \in (0,1)\\
u(0) & = & u(1) =0
\end{eqnarray*}}
\vspace*{-0.5cm}
\end{aufg}

%--------------------------------------------------
\begin{aufg}
Berechnen Sie die Frobenius-Norm 
\[ \|A \|_F := \sqrt{ \sum_{i,j=1}^n a_{ij}^2 }, \quad A=(a_{ij}) \in
\mathbb{R}^{n \times n}  \]
der Vandermode Matrix 
\lstinline!vander(0:0.02:1)! 
\end{aufg}
%--------------------------------------------------
\begin{aufg}
Ändern Sie in Aufgabe 11 die rechte Seite $1$ in
  $\sin(4 \pi x)$ und berechnen Sie eine Näherungslösung. 
\end{aufg}
%--------------------------------------------------
\newpage
\begin{aufg}
Berechnen Sie die Eigenwerte und Eigenvektoren der Matrix
\[ A = \left( \begin{array}{cccc}
30 & 1 & 2 & 3\\
4 & 15 & -4 & -2\\
-1 & 0 & 3 & 5\\
-3 & 5 & 0 & -1 \end{array}
\right) \]
Bestimmen Sie auch die $QR$-Zerlegung.
\end{aufg}

\begin{aufg}
Sei \lstinline!A=hilb(n)! und \lstinline!x=ones(n,1)!. Berechnen Sie
  für $n=5$ und $n=15$ den Vektor $b=A*x$, \lstinline!norm(x-A\b)! und die Kondition
  von $A$. Was stellen Sie fest? Erklären Sie das Ergebnis!
\end{aufg}

%-------------------------------------------------------------------------------
\end{document}
%-------------------------------------------------------------------------------
