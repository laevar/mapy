\documentclass[a4paper,10pt,DIV15]{scrartcl}
\usepackage[psamsfonts]{amssymb}
\usepackage{amsmath}
%\usepackage{latexsym}
\usepackage{theorem}


\usepackage{fontspec,xunicode,xltxtra}
\usepackage[ngerman]{babel}
\selectlanguage{ngerman}


\usepackage[svgnames,hyperref]{xcolor} %color definitions
%\usepackage{tikz}
%\usetikzlibrary{shadows}
%\usetikzlibrary{fit}
%\usetikzlibrary{shapes}
%\usetikzlibrary{backgrounds}


\usepackage{mcode}
%\usepackage{pstricks,pst-node,pst-text,pst-3d}
\parindent0cm % Abs�tze nicht einr�cken 

%---- neue Umgebung f�r Aufgaben
\theoremstyle{break}
\theoremheaderfont{\Large \bf}
\theorembodyfont{\normalfont}


% Definieren einer neuen Farbe
\definecolor{light-gray}{gray}{.9}

\newcounter{zaehler}     % neuen Z�hler einf�hren
\stepcounter{zaehler}    % Z�hler einen hochz�hlen

\newenvironment{aufg}%
%---- Header
{\begin{samepage}
\colorbox{light-gray}{                         % Box in gray
 \makebox[\textwidth]{                           % Box in linewidth
\textbf{Aufgabe} \arabic{zaehler} :}}\\[0.1cm]       % Header
%\begin{minipage}{0.5cm} \end{minipage}    % Insert 0.5cm
\begin{minipage}{\textwidth}}
%-----  foot
{\end{minipage} \nopagebreak %\begin{minipage}{1cm} \end{minipage}
\\[0.1cm] 
%\begin{minipage}{0.1cm} \end{minipage} 
%\hrulefill \begin{minipage}{1cm} \end{minipage}\\[1cm]  
\stepcounter{zaehler}                           % increase counter
 \end{samepage}%
}

%-------------------------------------------------------------------------------
\begin{document}
%-------------------------------------------------------------------------------

%--------------------------------------------------- Header
\begin{center}
\textbf{\LARGE Einf\"uhrung in MATLAB }\\
\end{center}
\begin{minipage}{6cm}
Dr. J. Schulz\\
\end{minipage}\hfill
\begin{minipage}{2cm}
\textbf{Einheit 2}\\
%30.07.2007
\end{minipage}\\[1cm]


%--------------------------------------------------
\begin{aufg}[0]
Geben Sie die folgende Zeile ein: 
\begin{lstlisting}
x=1e-15; ((1+x)-1)/x
\end{lstlisting} 
Wie interpretieren Sie das Ergebnis? 
(Testen Sie auch \lstinline!x=1e-16!!)
\end{aufg}

% \begin{aufg}[0]
% \label{exp}
% Worum gehts? loesung machen
% 
% Warum werden die Ergebnisse für negative $x$ groß? Warum sehen sie für positive x klein aus? Sind sie
% wirklich klein? Oder kann man sie nur nicht sehen? Was passiert, wenn man
% sich nur positive x ansieht? Man verändere das Programm an den richtigen
% Stellen, um sich die Lage genauer anzusehen.
% \end{aufg}


\begin{aufg}[0]
Differenzieren Sie $f(x)=exp(x)$ in $x=0$ durch den zentralen Differenzen-
quotienten. Plotten Sie den Approximationsfehler für die Approximation der
ersten Ableitung durch den zentralen Differenzenquotienten für die 
Exponentialfunktion an der Stelle $x=0$ mit doppelt logarithmischen Achsen und
interpetieren sie das Ergebnis.

Tips:
\begin{enumerate}
\item Bauen Sie sich einen Vektor, der eine passende Anzahl von positiven
h–Werten $h_1 \ldots h_n$ enthält.
\item Daraus bauen Sie sich Vektoren, die die Werte $exp(h_j)$ bzw. $exp(−h_j)$
enthalten, und dann
\item einen Vektor, der alle zentralen Differenzenquotienten enthält.
\item Berechnen Sie dann den Vektor, der die absoluten Fehler enthält,
\item und plotten Sie ihn gegen den Vektor der $h$-Werte.
\item Schauen Sie in der Doku nach, wie man einen doppelt logarithmischen
Plot macht.
\item Vermutlich werden Sie Gründe haben, Ihre Wahl der $h_j$ noch einmal
zu revidieren, um den Effekt klarer herauskommen zu lassen.
\end{enumerate}
\end{aufg}



\begin{aufg}[0]
Lösen Sie n\"aherungsweise die Fixpunktgleichung
\[ x_f=e^{(-x_f)}. \]
\end{aufg}

\begin{aufg}[0]
Berechnen Sie eine Nullstelle von 
\[ f(x) = \cos^2(x) -x. \]
\end{aufg}

\begin{aufg}[0]
Schreiben Sie eine Funktion, die für $n \in \mathbb{N}$ die
  Hilbert-Matrix $H=(h_{ij})_{i,j=1}^n$ mit $h_{ij}=\frac{1}{i+j-1}$
  berechnet. Berechnen Sie $H^{-1}$ für $n=4$. 
\end{aufg}
%
%-----------------------------------------------------------
%
\newpage

\begin{aufg}[0]
 Berechnen Sie die Nullstellen von 
\[ x^2-2, \quad x^2-2x+1, \quad x^2 -4x +10. \]
\end{aufg}
%
%-----------------------------------------------------------
%

\begin{aufg}[0]
Die Fibonacci-Folge ist definiert durch
\[ f_1:=1, \quad f_2:=1, \quad f_{k+2}:=f_{k+1}+f_k, \ k \in \mathbb{N}.
\]
Schreiben Sie ein Programm, das 
\[ g_{k}:= \frac{f_{k+1}}{f_k}, \quad k \in \mathbb{N} \]
berechnet. Stoppen Sie, falls $|g_{k}-g_{k+1}| \leq TOL$.
Geben Sie für $TOL=10^{-3}$ und  $TOL=10^{-4}$ das entsprechende $k$
und das entsprechende $g_k$ an.\\

{\it Hinweis}: Benutzen Sie eine \mcode{while}-Schleife.
\end{aufg}
%




\begin{aufg}[0]
Schreiben Sie eine Funktion, die einen String 'invertiert'.  
\end{aufg}

\begin{aufg}[0]
Schreiben Sie eine Funktion, die als Input-Parameter einen String erh\"alt und
die berechnet wie oft ein \mcode{char} in dem String auftritt.
\end{aufg}

%-------------------------------------------------------------------------------
\end{document}
%-------------------------------------------------------------------------------
