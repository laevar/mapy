\begin{aufg}[0]
Differenzieren Sie $f(x)=exp(x)$ in $x=0$ durch den zentralen Differenzen-
quotienten. Plotten Sie den Approximationsfehler für die Approximation der
ersten Ableitung durch den zentralen Differenzenquotienten für die 
Exponentialfunktion an der Stelle $x=0$ mit doppelt logarithmischen Achsen und
interpetieren sie das Ergebnis.

Tips:
\begin{enumerate}
\item Bauen Sie sich einen Vektor, der eine passende Anzahl von positiven
h–Werten $h_1 \ldots h_n$ enthält.
\item Daraus bauen Sie sich Vektoren, die die Werte $exp(h_j)$ bzw. $exp(−h_j)$
enthalten, und dann
\item einen Vektor, der alle zentralen Differenzenquotienten enthält.
\item Berechnen Sie dann den Vektor, der die absoluten Fehler enthält,
\item und plotten Sie ihn gegen den Vektor der $h$-Werte.
\item Schauen Sie in der Doku nach, wie man einen doppelt logarithmischen
Plot macht.
\item Vermutlich werden Sie Gründe haben, Ihre Wahl der $h_j$ noch einmal
zu revidieren, um den Effekt klarer herauskommen zu lassen.
\end{enumerate}
\end{aufg}