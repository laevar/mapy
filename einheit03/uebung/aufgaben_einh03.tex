\documentclass[a4paper,10pt,DIV15]{scrartcl}
\usepackage[psamsfonts]{amssymb}
\usepackage{amsmath}
%\usepackage{latexsym}
\usepackage{theorem}


\usepackage{fontspec,xunicode,xltxtra}
\usepackage[ngerman]{babel}
\selectlanguage{ngerman}


\usepackage[svgnames,hyperref]{xcolor} %color definitions
%\usepackage{tikz}
%\usetikzlibrary{shadows}
%\usetikzlibrary{fit}
%\usetikzlibrary{shapes}
%\usetikzlibrary{backgrounds}


\usepackage{mcode}
%\usepackage{pstricks,pst-node,pst-text,pst-3d}
\parindent0cm % Abs�tze nicht einr�cken 

%---- neue Umgebung f�r Aufgaben
\theoremstyle{break}
\theoremheaderfont{\Large \bf}
\theorembodyfont{\normalfont}


% Definieren einer neuen Farbe
\definecolor{light-gray}{gray}{.9}

\newcounter{zaehler}     % neuen Z�hler einf�hren
\stepcounter{zaehler}    % Z�hler einen hochz�hlen

\newenvironment{aufg}%
%---- Header
{\begin{samepage}
\colorbox{light-gray}{                         % Box in gray
 \makebox[\textwidth]{                           % Box in linewidth
\textbf{Aufgabe} \arabic{zaehler} :}}\\[0.1cm]       % Header
%\begin{minipage}{0.5cm} \end{minipage}    % Insert 0.5cm
\begin{minipage}{\textwidth}}
%-----  foot
{\end{minipage} \nopagebreak %\begin{minipage}{1cm} \end{minipage}
\\[0.1cm] 
%\begin{minipage}{0.1cm} \end{minipage} 
%\hrulefill \begin{minipage}{1cm} \end{minipage}\\[1cm]  
\stepcounter{zaehler}                           % increase counter
 \end{samepage}%
}

%-------------------------------------------------------------------------------
\begin{document}
%-------------------------------------------------------------------------------

%--------------------------------------------------- Header
\begin{center}
\textbf{\LARGE Einf\"uhrung in MATLAB }\\
\end{center}
\begin{minipage}{6cm}
Dr. J. Schulz\\
\end{minipage}\hfill
\begin{minipage}{2cm}
\textbf{Einheit 3}\\
%30.07.2007
\end{minipage}\\[1cm]

\begin{aufg}
\begin{itemize}
\item Interpolieren Sie an den durch \lstinline!x=linspace(-5,5,13)! gegebenen Stellen  die Funktion $f(x):=x^2\exp(-|x|)$.
\item Berechnen Sie approximativ den maximalen Fehler zwischen $f$ und
  ihrer Interpolierenden auf $[-5,5]$. 
(Hinweis: Befehl \lstinline!max!)
\item Ändern Sie den Vektor der Stützstellen
  \lstinline!x=linspace(-5,5,13)!, so dass 
\[ x_i = - 5 \cos(\pi (i-1)/12), \quad i=1, \dots , 13. \]
Berechnen Sie erneut den maximalen Fehler.
\item Betrachten Sie auch die Stützstellen
\[ x_i = - 5 \cos(\pi (i-1)/49), \quad i=1, \dots , 50. \] 
\end{itemize}
\end{aufg}

%-----------------------------------------------------------------------------------

\begin{aufg}
Schreiben Sie ein Programm, dass zu einem gegebenen $a>0$ die
  Funktion
\[ f(x):= 1/(x^2+a) \]
auf dem Intervall $[-3,3]$ plottet. 

%Erstellen Sie daraus eine Animation f\"ur $a \in [0,5]$.
\end{aufg}

%-----------------------------------------------------------------------------------
\begin{aufg}
Schreiben Sie eine Funktion, die einen String 'invertiert'.  
\end{aufg}

\newpage

%-----------------------------------------------------------------------------------
\begin{aufg}
\begin{minipage}{6cm}
Versuchen Sie die Grafik  selbst zu erstellen (inclusive
allen Beschriftungen) \\

{\it Hinweis:} $\pi$ wird durch {\lstinline!\pi!} dargestellt. 

\end{minipage} 
\hspace*{1cm}
\begin{minipage}{12.5cm}
\includegraphics[width=12cm, height=10cm]{aufgabe2}
\end{minipage}\\
\end{aufg}


%-----------------------------------------------------------------------------------
\begin{aufg}
Berechnen Sie $\int_0^1 xe^x dx$ exakt. Machen Sie die
  Probe, indem Sie das Programm \lstinline!integral.m! modifizieren. Wie
  groß muß $N$ mindestens gewählt werden, damit der absolute Fehler
  kleiner als $10^{-4}$ ist?
\end{aufg}
%-----------------------------------------------------------------------------------
\begin{aufg}
Stellen Sie die Funktion
\[ f(x,y,z)= \sin (4 \pi x) \sin( \pi y) y^2 (z^2-1), \quad (x,y,z) \in
[-1,1]^3 \]
grafisch dar.
\end{aufg}

\newpage
%-----------------------------------------------------------------------------------
\begin{aufg}
 Erstellen Sie eine Funktion, die zu einer gegebenen natürlichen
  Zahl $n$ ein regelmäßiges $n$-Eck zeichnet.\\

Wenden Sie auf die Kanten eines regelm\"a{\ss}igen Sechsecks, die rekursive
Funktion aus Blatt 2, Aufgabe 11 an.\\ 

{\it Hinweis:} Die Eckpunkte $(x_i,y_i)$ sind {
\[ x_i=\sin( 2 {\pi i}/{n} ), \quad  y_i=\cos( 2 {\pi i}/{n} ),
  \quad i=1, \dots ,n \]  }

\end{aufg}

%-----------------------------------------------------------------------------------
\begin{aufg}
Plotten Sie mit Hilfe von \lstinline!surf! die folgenden Funktionen auf
  $[-1,1]\times [-1,1]$
{
\[ \sin( \pi^2 xy), \ (x^2-1)(y^2-1), \ \sin(\pi x ^2), \ \sin(- \pi
  e^{-x^2-y^2}) \]}
in einem Grafikfenster. 
\end{aufg}

%-----------------------------------------------------------------------------------
\begin{aufg}
Plotten Sie die Funktion \[ f(x):= 1/(x^2+a) \]
auf dem Intervall $[-3,3]$ f\"ur $a=1:20$ und erstellen Sie daraus eine Animation! 
\end{aufg}


%-----------------------------------------------------------------------------------
\begin{aufg}
Schreiben Sie eine Funktion, die als Input-Parameter einen String erh\"alt und
die berechnet wie oft ein \lstinline!char! in dem String auftritt.
\end{aufg}


%-------------------------------------------------------------------------------
\end{document}
%-------------------------------------------------------------------------------
