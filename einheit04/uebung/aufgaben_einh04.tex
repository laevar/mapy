\documentclass[a4paper,10pt,DIV15]{scrartcl}
\usepackage[psamsfonts]{amssymb}
\usepackage{amsmath}
\usepackage[svgnames]{xcolor} %color definitions

\usepackage{fontspec,xunicode,xltxtra}
%\usepackage{fontspec,xunicode}
%\usepackage{polyglossia}
%\setdefaultlanguage[spelling=new, latesthyphen=true]{german}
%\setsansfont{DejaVu Sans}
%\setsansfont{Verdana}
%\setsansfont{Arial}
%\setromanfont[Mapping=tex-text]{Linux Libertine}
%\setsansfont[Mapping=tex-text]{Myriad Pro}
%\setmonofont[Mapping=tex-text]{Courier New}

%\setsansfont{Linux Biolinum}

\usepackage[ngerman]{babel}
\selectlanguage{ngerman}

%
% math/symbols
%
\usepackage{amssymb}
\usepackage{amsthm}
% \usepackage{latexsym}
\usepackage{amsmath}
%\usepackage{amsxtra} %Weitere Extrasymbole
%\usepackage{empheq} %Gleichungen hervorheben
%\usepackage{bm}
 %\bm{A} Boldface im Mathemodus

\usepackage{multimedia}
%\usepackage{tikz}

\usepackage{cellspace}
\setlength{\cellspacetoplimit}{2pt}
\setlength{\cellspacebottomlimit}{2pt}

%%%%%%%%%%%%%%%%%% Fuer Frames [fragile]-Option verwenden!
%Programm-Listing
%%%%%%%%%%%%%%%%%%
%Listingsumgebung fuer verbatim
%Grauhinterlegeter Text
%Automatischer Zeilenumbruch ist aktiviert
%\usepackage{listings}
\usepackage[framed]{mcode}
%\usepackage{mcode}
% This command allows you to typeset syntax highlighted Matlab
% code ``inline''.
% mcode fuer matlab

\definecolor{lgray}{gray}{0.80}
\definecolor{gray}{gray}{0.3}
\definecolor{darkgreen}{rgb}{0,0.4,0}
\definecolor{darkblue}{rgb}{0,0,0.8}
\definecolor{key}{rgb}{0,0.5,0} 
\newcommand{\imatlab}[1]{\lstset{basicstyle=\color[gray]{0.6}}\lstinline|#1|}
\newcommand{\isage}[1]{\lstset{basicstyle=\color[gray]{0.3}}\lstinline|#1|}
%\lstset{backgroundcolor=\color{lgray}, frame=single, basicstyle=\ttfamily, breaklines=true}
%\lstnewenvironment{sage}{\lstset{,language=python, keywordstyle=color{blue},    commentstyle=color{green}, emphstyle=\color{red}, %frame=single, stringstyle=\color{red}, basicstyle=\ttfamily, ,mathescape =true,escapechar=§}}{}

\lstnewenvironment{matlab}[1][]{\lstset{xleftmargin=0.2cm,frame=none,backgroundcolor=\color{white},basicstyle=\color{darkblue}\ttfamily\small,#1}}{} 
\lstnewenvironment{matlabin}[1][]{\lstset{
%language=python,
backgroundcolor=\color[gray]{0.9},
breaklines=true,
basicstyle=\ttfamily\small,
%otherkeywords={ =},
%keywordstyle=\color{blue},
%stringstyle=\color{darkgreen},
showstringspaces=false,
%emph={for, while, if, elif, else, not, and, or, printf, break, continue, return, end, function},
%emphstyle=\color{blue},
%emph={[2]True, False, None, self, NaN, NULL},
%emphstyle=[2]\color{key},
%emph={[3]from, import, as},
%emphstyle=[3]\color{blue},
%upquote=true,
%morecomment=[s]{"""}{"""},
%commentstyle=\color{gray}\slshape,
%framexleftmargin=1mm, framextopmargin=1mm, 
%title=\tiny matlab,
frame=single,
%mathescape =true,
%escapechar=§
,#1}
}{} 
\newcommand{\matinput}[1]{\lstset{
%language=python,
backgroundcolor=\color[gray]{0.9},
breaklines=true,
basicstyle=\ttfamily\small,
%otherkeywords={ =},
%keywordstyle=\color{blue},
%stringstyle=\color{darkgreen},
showstringspaces=false,
%emph={for, while, if, elif, else, not, and, or, printf, break, continue, return, end, function},
%emphstyle=\color{blue},
%emph={[2]True, False, None, self, NaN, NULL},
%emphstyle=[2]\color{key},
%emph={[3]from, import, as},
%emphstyle=[3]\color{blue},
%upquote=true,
%morecomment=[s]{"""}{"""},
%commentstyle=\color{gray}\slshape,
%framexleftmargin=1mm, framextopmargin=1mm, 
%title=\tiny matlab,
frame=single}\lstinputlisting{#1}}

\lstnewenvironment{pyout}[1][]{\lstset{language=python,xleftmargin=0.2cm,frame=none,backgroundcolor=\color{white},basicstyle=\color{darkblue}\ttfamily\small,#1}}{}
\lstnewenvironment{pyin}[1][]{\lstset{
language=python,
backgroundcolor=\color[gray]{0.7},
breaklines=true,
basicstyle=\ttfamily\small,
%otherkeywords={ =},
keywordstyle=\color{blue},
stringstyle=\color{darkgreen},
showstringspaces=false,
emph={class, pass, in, for, while, if, is, elif, else, not, and, or,
def, print, exec, break, continue, return},
emphstyle=\color{blue},
emph={[2]True, False, None, self},
emphstyle=[2]\color{key},
emph={[3]from, import, as},
emphstyle=[3]\color{blue},
upquote=true,
morecomment=[s]{"""}{"""},
%commentstyle=\color{gray}\slshape,
%framexleftmargin=1mm, framextopmargin=1mm, 
%title=\tiny python,
%caption=python,
frame=single,
%frameround=tttt,
mathescape =true,
escapechar=§,
#1
}
}{}

%\usepackage{caption}
%\DeclareCaptionFont{white}{ \color{white} }
%\DeclareCaptionFormat{listing}{
%  \colorbox[cmyk]{0.43, 0.35, 0.35,0.01 }{
%      \parbox{\textwidth}{\hspace{15pt}#1#2#3}
%        }
%        }
%        \captionsetup[lstlisting]{ format=listing, labelfont=white, textfont=white, singlelinecheck=false, margin=0pt, font={bf,footnotesize} }


\usepackage{mydef}
%\usepackage{cmap} % you can search in the pdf for umlauts and ligatures
\usepackage{colonequals} %corrects the definition-symbols \colonequals (besides others)

\usepackage{ifthen}

%%%%%%%%%%%%%%%%%%%
%Neue Definitionen
%%%%%%%%%%%%%%%%%%%

%Newcommands
\newcommand{\Fun}[1]{\mathcal{#1}}      %Mathcal fuer Funktoren
\newcommand{\field}[1]{\mathbb{#1}}     %Grundkoerper ?? in mathds

\newcommand{\A}{\field{A}}              %Affines A
\newcommand{\Fp}{\field{F}_{\!p}}       %Endlicher Koerper mit p Elementen
\newcommand{\Fq}{\field{F}_{\!q}}       %Endlicher Koerper mit q Elementen
\newcommand{\Ga}{\field{G}_{a}}         %Add Gruppenschema
\newcommand{\K}{\field{K}}              %Generischer Koerper 
\newcommand{\N}{\field{N}}              %Nat Zahlen
\newcommand{\Pj}{\field{P}}             %Projektives P
\newcommand{\R}{\field{R}} 		%Reelle Zahlen
\newcommand{\Q}{\field{Q}}              %Rationale Zahlen  
\newcommand{\Qt}{\field{H}}             %Quaternionen 
\newcommand{\V}{\field{V}}              %Vektorbuendel V
\newcommand{\Z}{\field{Z}}              %Ganze Zahlen
\DeclareMathOperator{\Real}{Re}

\newcommand{\fdg}{\;|\;}                 %fuer die gilt

%Operatoren
\DeclareMathOperator{\Abb}{Abb}
%\usepackage{sagetex}


%
% Aufgaben
%
\parindent0cm % Abs�tze nicht einr�cken 
% Definieren einer neuen Farbe
\definecolor{light-gray}{gray}{.9}

\newcounter{zaehler}     % neuen Z�hler einf�hren
\newenvironment{aufgn}[2][0]
%---- Header
{\begin{samepage}%
%\colorbox{light-gray}{%                         % Box in gray
% \makebox[\textwidth]{%                           % Box in linewidth
%\textbf{Aufgabe \arabic{zaehler} } }\hspace{-\textwidth}\makebox[\textwidth]{\hfill #1 Punkte} }\\[0.05cm]       % Header
\dotfill\\
{\large\textbf{Aufgabe \arabic{zaehler} \ifthenelse{ \equal{#2}{} }{}{: \emph{ #2 } }}\ifthenelse{-1=#1}{(testierbar)}{}\ifthenelse{0=#1 \or -1=#1}{}{\hfill #1 Punkte} }\\[0.4cm]
%{\large\textbf{Aufgabe \arabic{zaehler}  #2 }\ifthenelse{-1=#1}{(testierbar)}{}\ifthenelse{0=#1 \or -1=#1}{}{\hfill #1 Punkte} }\\[0.4cm]
\begin{minipage}{\textwidth}%
}%
%-----  foot
{\end{minipage}\nopagebreak%\begin{minipage}{1cm} \end{minipage}
%\\ 
%\begin{minipage}{0.1cm} \end{minipage} 
%\hrulefill \begin{minipage}{1cm} \end{minipage}\\[1cm]  
\stepcounter{zaehler}                           % increase counter
\end{samepage}%
\\%
\bigskip%
}


\newenvironment{aufg}[1][0]
%---- Header
{\begin{samepage}%
\refstepcounter{zaehler}% increase counter
%\colorbox{light-gray}{%                         % Box in gray
% \makebox[\textwidth]{%                           % Box in linewidth
%\textbf{Aufgabe \arabic{zaehler} } }\hspace{-\textwidth}\makebox[\textwidth]{\hfill #1 Punkte} }\\[0.05cm]       % Header
\dotfill\\
{\large\textbf{Aufgabe \arabic{zaehler} }\ifthenelse{-1=#1}{(testierbar)}{}\ifthenelse{0=#1 \or -1=#1}{}{\hfill #1 Punkte} }\\[0.4cm]
\begin{minipage}{\textwidth}%
}%
%-----  foot
{\end{minipage}\nopagebreak%\begin{minipage}{1cm} \end{minipage}
%\\ 
%\begin{minipage}{0.1cm} \end{minipage} 
%\hrulefill \begin{minipage}{1cm} \end{minipage}\\[1cm]  
\end{samepage}%
\\%
\bigskip%
}


%\usepackage{tikz}
%\usetikzlibrary{shadows}
%\usetikzlibrary{fit}
%\usetikzlibrary{shapes}
%\usetikzlibrary{backgrounds}


%-------------------------------------------------------------------------------
\begin{document}
%-------------------------------------------------------------------------------

%--------------------------------------------------- Header
\begin{center}
\textbf{\LARGE Wissenschaftliches Rechnen mit Matlab/Python}\\
\end{center}
\begin{minipage}{6cm}
Jochen Schulz
\end{minipage}\hfill

\begin{minipage}{2cm}
\textbf{Einheit 4}\\
%30.07.2007
\end{minipage}\\[1cm]

%-----------------------------------------------------------------------------------
\begin{aufg}
\begin{itemize}
\item Betrachten Sie die Datei '\mcode{daten.dat}' mittels des Befehls:
  \mcode{type daten.dat}.
\item Schreiben Sie ein Programm, dass die Daten importiert und die
  Funktion anhand der gegebenen Daten plottet.
\end{itemize}
\end{aufg}
%-----------------------------------------------------------------------------------

\begin{aufg}
\begin{itemize}
\item Erzeugen Sie drei Vektoren durch 
\begin{lstlisting}
x = rand(2000,1);
y = rand(2000,1);
z = sin(4*pi*x).*cos(2*pi*y);
\end{lstlisting}
\item Plotten Sie zuerst nur die Punkte.
\item Interpolieren sie die Punkte auf einem regelmässigen Gitter und erstellen Sie Grafiken mit \mcode{surf()}, \mcode{mesh()} und \mcode{contour()}.
\item Beschriften Sie die Konturlinien von \mcode{contour}.
\item Untersuchen Sie den Einflu{\ss} der verschiedenen
  Interpolationsmethoden. 
\end{itemize}
\end{aufg}
%-----------------------------------------------------------------------------------
\begin{aufg}
\begin{itemize}
\item [(a)] Laden Sie mittels \mcode{load census} die
  U.S. Population  von  1790 bis 1990 in ihren Speicher und stellen Sie die Zahlen grafisch
  dar.
\item[(b)] Interpolieren Sie mit Hilfe der Matlab-Oberfläche die Daten!
  Welche Methode funktioniert am besten?
\item [(c)] Schätzen Sie mit Hilfe des kubischen Splines die
  Bevölkerungszahl 2050.
\end{itemize}
\end{aufg}
%-----------------------------------------------------------------------------------%
\begin{aufg}
Modifizieren Sie das Programm \mcode{mandel.m} aus der ersten Vorlesung derart,
dass der analysierte Ausschnitt $[x_{min},x_{max}]\times [y_{min},y_{max}]$
interaktiv verändert wird. 
\begin{itemize}
\item Durch Drücken der
linken Maustaste auf einen bestimmten Punkt $(x,y)$ der Grafik soll  die Grafik neu
erstellt werden, wobei $(x,y)$ das Zentrum des neuen Ausschnitts mit
Gr\"o{\ss}e 
\[ \left (  2\frac{x_{max}-x_{min}}{3},  2\frac{y_{max}-y_{min}}{3} \right)\]
ist.
\item Durch Drücken der
rechten Maustaste auf einen bestimmten Punkt $(x,y)$ der Grafik soll  die Grafik neu
erstellt werden, wobei $(x,y)$ das Zentrum des neuen Ausschnitts mit
Gr\"o{\ss}e 
\[ \left (  3\frac{x_{max}-x_{min}}{2},  3\frac{y_{max}-y_{min}}{2} \right)\]
ist.
\item Drücken der mittleren Maustaste beendet das Programm. 
\end{itemize} 
\end{aufg}
\newpage
%-----------------------------------------------------------------------------------
\begin{aufg}
Schreiben Sie eine Funktion, die eine beliebige Textdatei einliest und
auswertet. Output-Argumente sollen die Anzahl der Zeilen und Zeichen der Datei
und das h\"aufigste vorkommene Zeichen sein. 
\end{aufg}
%-----------------------------------------------------------------------------------
\begin{aufg}
Schreiben Sie eine rekursive Funktion, die aus gegebenen $x$ und $n \in
\mathbb{N}$ die Potenz $x^n$ berechnet. \\
{\it Hinweis:} Benutzen Sie 
\[ x^n = \left \{ \begin{array}{ll} 
x^{n/2} x^{n/2}, & n \mbox{ gerade} \\
x x^{(n-1)/2} x^{(n-1)/2}, &  n \mbox{ ungerade}\\
\end{array} \right.  \]
und die MATLAB-Funktion \mcode{mod(n,2)}, die $0$ ergibt, falls $n$ gerade ist und
$1$ sonst.
\end{aufg}
%-----------------------------------------------------------------------------------
\begin{aufg}
Wir betrachten die Folge 
\[ x_{n+1} = \frac12 x_n + \frac{1}{x_n}, \qquad n =1,2,\dots \]
Schreiben Sie eine Funktion, die zu einem gegebenen Startwert $x_1$ die kleinste
Zahl $n$ berechnet mit 
\[ |x_{n+1} - x_n| \leq 10^{-4} .\]
\end{aufg}
%-----------------------------------------------------------------------------------
\begin{aufg}
Welche Werte besitzen $x1$, $x2$, $x3$  am Ende der jeweiligen Eingaben?
\begin{itemize}
\item (a) 
\begin{lstlisting}
x = [ 1 2 3 4];
y = [ 0 3 2 4];
x1 = (y-x > 0) & (x ~= 1)
\end{lstlisting}
\item (b)
\begin{lstlisting}
x = sin(1:100);
x2 = length(x)
\end{lstlisting}
\item (c)
\begin{lstlisting}
x = linspace(0,1,10);
y = 1:2:6;
x3 = x(y)
\end{lstlisting}
\end{itemize} 
\end{aufg}

\begin{aufg}
Die folgenden Befehlszeilen sind fehlerhaft. Erkl\"aren Sie jeweils den Fehler!
\begin{itemize}
\item (a)
\begin{lstlisting}
x = 100:200;
y = linspace(300,400,100);
z = x.*y.*2
\end{lstlisting}
\item (b)
\begin{lstlisting}
clear all
z(5) = 10;
a = (6:10).^z;
b = a.*ones(5,1)
\end{lstlisting}

\end{itemize} 

\end{aufg}

%-------------------------------------------------------------------------------
\end{document}
%-------------------------------------------------------------------------------
