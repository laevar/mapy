\documentclass[a4paper,10pt,DIV15]{scrartcl}
\usepackage[psamsfonts]{amssymb}
\usepackage{amsmath}
%\usepackage{latexsym}
\usepackage{theorem}


\usepackage{fontspec,xunicode,xltxtra}
\usepackage[ngerman]{babel}
\selectlanguage{ngerman}


\usepackage[svgnames,hyperref]{xcolor} %color definitions
%\usepackage{tikz}
%\usetikzlibrary{shadows}
%\usetikzlibrary{fit}
%\usetikzlibrary{shapes}
%\usetikzlibrary{backgrounds}


\usepackage{mcode}
%\usepackage{pstricks,pst-node,pst-text,pst-3d}
\parindent0cm % Abs�tze nicht einr�cken 

%---- neue Umgebung f�r Aufgaben
\theoremstyle{break}
\theoremheaderfont{\Large \bf}
\theorembodyfont{\normalfont}


% Definieren einer neuen Farbe
\definecolor{light-gray}{gray}{.9}

\newcounter{zaehler}     % neuen Z�hler einf�hren
\stepcounter{zaehler}    % Z�hler einen hochz�hlen

\newenvironment{aufg}%
%---- Header
{\begin{samepage}
\colorbox{light-gray}{                         % Box in gray
 \makebox[\textwidth]{                           % Box in linewidth
\textbf{Aufgabe} \arabic{zaehler} :}}\\[0.1cm]       % Header
%\begin{minipage}{0.5cm} \end{minipage}    % Insert 0.5cm
\begin{minipage}{\textwidth}}
%-----  foot
{\end{minipage} \nopagebreak %\begin{minipage}{1cm} \end{minipage}
\\[0.1cm] 
%\begin{minipage}{0.1cm} \end{minipage} 
%\hrulefill \begin{minipage}{1cm} \end{minipage}\\[1cm]  
\stepcounter{zaehler}                           % increase counter
 \end{samepage}%
}

%-------------------------------------------------------------------------------
\begin{document}
%-------------------------------------------------------------------------------

%--------------------------------------------------- Header
\begin{center}
\textbf{\LARGE Einf\"uhrung in MATLAB }\\
\end{center}
\begin{minipage}{6cm}
Dr. J. Schulz\\
\end{minipage}\hfill
\begin{minipage}{2cm}
\textbf{Einheit 4}\\
%30.07.2007
\end{minipage}\\[1cm]

%-----------------------------------------------------------------------------------
\begin{aufg}
\begin{itemize}
\item Betrachten Sie die Datei '\mcode{daten.dat}' mittels des Befehls:
  \mcode{type daten.dat}.
\item Schreiben Sie ein Programm, dass die Daten importiert und die
  Funktion anhand der gegebenen Daten plottet.
\end{itemize}
\end{aufg}
%-----------------------------------------------------------------------------------

\begin{aufg}
\begin{itemize}
\item Erzeugen Sie drei Vektoren durch 
\begin{lstlisting}
x = rand(2000,1);
y = rand(2000,1);
z = sin(4*pi*x).*cos(2*pi*y);
\end{lstlisting}
\item Plotten Sie zuerst nur die Punkte.
\item Interpolieren sie die Punkte auf einem regelmässigen Gitter und erstellen Sie Grafiken mit \mcode{surf()}, \mcode{mesh()} und \mcode{contour()}.
\item Beschriften Sie die Konturlinien von \mcode{contour}.
\item Untersuchen Sie den Einflu{\ss} der verschiedenen
  Interpolationsmethoden. 
\end{itemize}
\end{aufg}
%-----------------------------------------------------------------------------------
\begin{aufg}
\begin{itemize}
\item [(a)] Laden Sie mittels \mcode{load census} die
  U.S. Population  von  1790 bis 1990 in ihren Speicher und stellen Sie die Zahlen grafisch
  dar.
\item[(b)] Interpolieren Sie mit Hilfe der Matlab-Oberfläche die Daten!
  Welche Methode funktioniert am besten?
\item [(c)] Schätzen Sie mit Hilfe des kubischen Splines die
  Bevölkerungszahl 2050.
\end{itemize}
\end{aufg}
%-----------------------------------------------------------------------------------%
\begin{aufg}
Modifizieren Sie das Programm \mcode{mandel.m} aus der ersten Vorlesung derart,
dass der analysierte Ausschnitt $[x_{min},x_{max}]\times [y_{min},y_{max}]$
interaktiv verändert wird. 
\begin{itemize}
\item Durch Drücken der
linken Maustaste auf einen bestimmten Punkt $(x,y)$ der Grafik soll  die Grafik neu
erstellt werden, wobei $(x,y)$ das Zentrum des neuen Ausschnitts mit
Gr\"o{\ss}e 
\[ \left (  2\frac{x_{max}-x_{min}}{3},  2\frac{y_{max}-y_{min}}{3} \right)\]
ist.
\item Durch Drücken der
rechten Maustaste auf einen bestimmten Punkt $(x,y)$ der Grafik soll  die Grafik neu
erstellt werden, wobei $(x,y)$ das Zentrum des neuen Ausschnitts mit
Gr\"o{\ss}e 
\[ \left (  3\frac{x_{max}-x_{min}}{2},  3\frac{y_{max}-y_{min}}{2} \right)\]
ist.
\item Drücken der mittleren Maustaste beendet das Programm. 
\end{itemize} 
\end{aufg}
\newpage
%-----------------------------------------------------------------------------------
\begin{aufg}
Schreiben Sie eine Funktion, die eine beliebige Textdatei einliest und
auswertet. Output-Argumente sollen die Anzahl der Zeilen und Zeichen der Datei
und das h\"aufigste vorkommene Zeichen sein. 
\end{aufg}
%-----------------------------------------------------------------------------------
\begin{aufg}
Schreiben Sie eine rekursive Funktion, die aus gegebenen $x$ und $n \in
\mathbb{N}$ die Potenz $x^n$ berechnet. \\
{\it Hinweis:} Benutzen Sie 
\[ x^n = \left \{ \begin{array}{ll} 
x^{n/2} x^{n/2}, & n \mbox{ gerade} \\
x x^{(n-1)/2} x^{(n-1)/2}, &  n \mbox{ ungerade}\\
\end{array} \right.  \]
und die MATLAB-Funktion \mcode{mod(n,2)}, die $0$ ergibt, falls $n$ gerade ist und
$1$ sonst.
\end{aufg}
%-----------------------------------------------------------------------------------
\begin{aufg}
Wir betrachten die Folge 
\[ x_{n+1} = \frac12 x_n + \frac{1}{x_n}, \qquad n =1,2,\dots \]
Schreiben Sie eine Funktion, die zu einem gegebenen Startwert $x_1$ die kleinste
Zahl $n$ berechnet mit 
\[ |x_{n+1} - x_n| \leq 10^{-4} .\]
\end{aufg}
%-----------------------------------------------------------------------------------
\begin{aufg}
Welche Werte besitzen $x1$, $x2$, $x3$  am Ende der jeweiligen Eingaben?
\begin{itemize}
\item (a) 
\begin{lstlisting}
x = [ 1 2 3 4];
y = [ 0 3 2 4];
x1 = (y-x > 0) & (x ~= 1)
\end{lstlisting}
\item (b)
\begin{lstlisting}
x = sin(1:100);
x2 = length(x)
\end{lstlisting}
\item (c)
\begin{lstlisting}
x = linspace(0,1,10);
y = 1:2:6;
x3 = x(y)
\end{lstlisting}
\end{itemize} 
\end{aufg}

\begin{aufg}
Die folgenden Befehlszeilen sind fehlerhaft. Erkl\"aren Sie jeweils den Fehler!
\begin{itemize}
\item (a)
\begin{lstlisting}
x = 100:200;
y = linspace(300,400,100);
z = x.*y.*2
\end{lstlisting}
\item (b)
\begin{lstlisting}
clear all
z(5) = 10;
a = (6:10).^z;
b = a.*ones(5,1)
\end{lstlisting}

\end{itemize} 

\end{aufg}

%-------------------------------------------------------------------------------
\end{document}
%-------------------------------------------------------------------------------
