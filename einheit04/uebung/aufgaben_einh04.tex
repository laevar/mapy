 %----------------------------------------------------------------
%  Einführung in MuPAD
% 
% Blockpraktikum August 2007
% (2 Wochen)
%----------------------------------------------------------------

\documentclass[12pt]{article}
\usepackage{german, a4}
\usepackage{amsmath}
\usepackage{latexsym}
\usepackage{amssymb}
\usepackage[ansinew]{inputenc}
\usepackage{exscale}
\usepackage{theorem}
%\usepackage[dvips]{graphics}
%\usepackage{pstricks,pst-node,pst-text,pst-3d}
\usepackage{color}
\usepackage{times}                  % Postscript font!!!
\parindent0cm % Absätze nicht einrücken 

%-- Selbstdefinierte Befehle und Parametereinstellungen ------------------------
\setlength{\headheight}{1.1\baselineskip}
\setlength{\textheight}{22cm}       
\setlength{\textwidth}{15cm}
\setlength{\topmargin}{-2cm} %oberer Rand bis Kopfzeile angehoben
\setlength{\oddsidemargin}{0.5cm} %linker Rand f"ur ungerade Seiten


%---- neue Umgebung für Aufgaben
\theoremstyle{break}
\theoremheaderfont{\Large \bf}
\theorembodyfont{\normalfont}


% Definieren einer neuen Farbe
\definecolor{light-gray}{gray}{.9}

\newcounter{zaehler}     % neuen Zähler einführen
\stepcounter{zaehler}    % Zähler einen hochzählen


\newenvironment{aufg}%
%---- Header
{\begin{samepage}
\colorbox{light-gray}{                         % Box in gray
 \makebox[\textwidth]{                           % Box in linewidth
\bf \color{red} Aufgabe \arabic{zaehler} :}}\\[0.2cm]       % Header
\begin{minipage}{0.5cm} \end{minipage} \hfill   % Insert 0.5cm
\begin{minipage}{13.5cm}}                         
%-----  foot
{\end{minipage} \nopagebreak \begin{minipage}{1cm} \end{minipage}
\\[0.2cm] 
\begin{minipage}{0.1cm} \end{minipage} 
\hrulefill \begin{minipage}{1cm} \end{minipage}\\[1cm]  
\stepcounter{zaehler}                           % increase counter
 \end{samepage}%
}


%-------------------------------------------------------------------------------
\begin{document}
%-------------------------------------------------------------------------------

%--------------------------------------------------- Header
\begin{center}
{\LARGE \bf Einf\"uhrung in MATLAB }\\
%{\LARGE \bf (Blockveranstaltung)}\\
%SS 2007\\[0.8cm]
\end{center}
\begin{minipage}{6cm}
Dr. G. Rapin,\\
%Th. Wassong\\
\end{minipage} \hfill 
\begin{minipage}{2cm}
{\bf Einheit 4}\\
%2.08.2007
\end{minipage}\\[1cm]

%-----------------------------------------------------------------------------------
\begin{aufg}
\begin{itemize}
\item Betrachten Sie die Datei \lstinline!daten.dat! mittels des Befehls
  \lstinline!type daten.dat!.
\item Schreiben Sie ein Programm, dass die Daten importiert und die
  Funktion anhand der gegebenen Daten plottet.
\end{itemize}
\end{aufg}
%-----------------------------------------------------------------------------------

\begin{aufg}
\begin{itemize}
\item Erzeugen Sie drei Vektoren durch 
\begin{verbatim}
x = rand(2000,1);
y = rand(2000,1);
z = sin(4*pi*x).*cos(2*pi*y);
\end{verbatim}
\item Plotten Sie zuerst nur die Punkte.
\item Erstellen Sie Grafiken mit \lstinline!surf!, \lstinline!mesh! und \lstinline!contour!.
\item Beschriften Sie die Konturlinien von \lstinline!contour!.
\item Untersuchen Sie den Einflu{\ss} der verschiedenen
  Interpolationsmethoden. 
\end{itemize}
\end{aufg}
%-----------------------------------------------------------------------------------
\newpage
\begin{aufg}
\begin{itemize}
\item [(a)] Laden Sie mittels {\lstinline!load census!} die
  U.S. Population  von
  1790 bis 1990 in ihren Speicher und stellen Sie die Zahlen grafisch
  dar.
\item[(b)] Interpolieren Sie mit Hilfe der Oberfläche die Daten!
  Welche Methode funktioniert am besten!
\item [(c)] Schätzen Sie mit Hilfe des kubischen Splines die
  Bevölkerungszahl 2050.
\end{itemize}
\end{aufg}
%-----------------------------------------------------------------------------------%
\begin{aufg}
Modifizieren Sie das Programm \lstinline!mandel.m! aus der ersten Vorlesung derart,
dass der analysierte Ausschnitt $[x_{min},x_{max}]\times [y_{min},y_{max}]$
interaktiv verändert wird. 
\begin{itemize}
\item Durch Drücken der
linken Maustaste auf einen bestimmten Punkt $(x,y)$ der Grafik soll  die Grafik neu
erstellt werden, wobei $(x,y)$ das Zentrum des neuen Ausschnitts mit
Gr\"o{\ss}e 
\[ \left (  2\frac{x_{max}-x_{min}}{3},  2\frac{y_{max}-y_{min}}{3} \right)\]
ist.
\item Durch Drücken der
rechten Maustaste auf einen bestimmten Punkt $(x,y)$ der Grafik soll  die Grafik neu
erstellt werden, wobei $(x,y)$ das Zentrum des neuen Ausschnitts mit
Gr\"o{\ss}e 
\[ \left (  3\frac{x_{max}-x_{min}}{2},  3\frac{y_{max}-y_{min}}{2} \right)\]
ist.
\item Drücken der mittleren Maustaste beendet das Programm. 
\end{itemize} 
\end{aufg}
%-----------------------------------------------------------------------------------
\begin{aufg}
Schreiben Sie eine Funktion, die eine beliebige Textdatei einliest und
auswertet. Output-Argumente sollen die Anzahl der Zeilen und Zeichen der Datei
und das h\"aufigste vorkommene Zeichen sein. 
\end{aufg}
%-----------------------------------------------------------------------------------
\begin{aufg}
Schreiben Sie eine rekursive Funktion, die aus gegebenen $x$ und $n \in
\mathbb{N}$ die Potenz $x^n$ berechnet. \\
{\it Hinweis:} Benutzen Sie 
\[ x^n = \left \{ \begin{array}{ll} 
x^{n/2} x^{n/2}, & n \mbox{ gerade} \\
x x^{(n-1)/2} x^{(n-1)/2}, &  n \mbox{ ungerade}\\
\end{array} \right.  \]
und die MATLAB-Funktion \lstinline!mod(n,2)!, die $0$ ergibt, falls $n$ gerade ist und
$1$ sonst.
\end{aufg}
%-----------------------------------------------------------------------------------
\begin{aufg}
Wir betrachten die Folge 
\[ x_{n+1} = \frac12 x_n + \frac{1}{x_n}, \qquad n =1,2,\dots \]
Schreiben Sie eine Funktion, die zu einem gegebenen Startwert $x_1$ die kleinste
Zahl $n$ berechnet mit 
\[ |x_{n+1} - x_n| \leq 10^{-4} .\]
\end{aufg}
%-----------------------------------------------------------------------------------
\begin{aufg}
Welche Werte besitzen $x1$, $x2$, $x3$  am Ende der jeweiligen Eingaben?
\begin{itemize}
\item [(a)]
\begin{verbatim}
>> x=[ 1 2 3 4];
>> y=[ 0 3 2 4];
>> x1= (y-x > 0) & (x ~= 1)
\end{verbatim}
\item [(b)]
\begin{verbatim}
>> x=sin(1:100);
>> x2=length(x)
\end{verbatim}
\item [(c)]
\begin{verbatim}
>> x=linspace(0,1,10);
>> y=1:2:6;
>> x3=x(y)
\end{verbatim}
\end{itemize} 
\end{aufg}
%-----------------------------------------------------------------------------------
\newpage 
\begin{aufg}
Die folgenden Befehlszeilen sind fehlerhaft. Erkl\"aren Sie jeweils den Fehler!
\begin{itemize}
\item [(a)]
\begin{verbatim}
>> x=100:200;
>> y=linspace(300,400,100);
>> z=x.*y.*2
\end{verbatim}
\item [(b)]
\begin{verbatim}
>> clear all
>> z(5)=10;
>> a=(6:10).^z;
>> b=a.*ones(5,1)
\end{verbatim}

\end{itemize} 

\end{aufg}

%-------------------------------------------------------------------------------
\end{document}
%-------------------------------------------------------------------------------
