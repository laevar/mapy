\begin{aufg}[0]
Laden sie sich das Programm \mcode{mandel.m}\isage{mandel.py} herunter (im zip der Einheit4) welches das
Mandelbrotfraktal berechnet.
Analysieren sie das Programm und modifizieren sie es dann so, 
dass der sichtbare Ausschnitt $[x_{min},x_{max}]\times [y_{min},y_{max}]$
interaktiv verändert wird. 
\begin{itemize}
\item Durch Drücken der
linken Maustaste auf einen bestimmten Punkt $(x,y)$ der Grafik soll  die Grafik neu
erstellt werden, wobei $(x,y)$ das Zentrum des neuen Ausschnitts mit
Gr\"o{\ss}e 
\[ \left (  2\frac{x_{max}-x_{min}}{3},  2\frac{y_{max}-y_{min}}{3} \right)\]
ist.
\item Durch Drücken der
rechten Maustaste auf einen bestimmten Punkt $(x,y)$ der Grafik soll  die Grafik neu
erstellt werden, wobei $(x,y)$ das Zentrum des neuen Ausschnitts mit
Gr\"o{\ss}e 
\[ \left (  3\frac{x_{max}-x_{min}}{2},  3\frac{y_{max}-y_{min}}{2} \right)\]
ist.
\item Drücken der mittleren Maustaste beendet das Programm. 
\end{itemize} 
\textsl{Anmerkung für Python}: die Rechte-Maustaste ist mit ginput nicht genauso zu verarbeiten. Hier reicht es das hereinzoomen zu implementieren. 
\end{aufg}
