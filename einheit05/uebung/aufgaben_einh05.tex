\documentclass[a4paper,10pt,DIV15]{scrartcl}
\usepackage[psamsfonts]{amssymb}
\usepackage{amsmath}
%\usepackage{latexsym}
\usepackage{theorem}


\usepackage{fontspec,xunicode,xltxtra}
\usepackage[ngerman]{babel}
\selectlanguage{ngerman}


\usepackage[svgnames,hyperref]{xcolor} %color definitions
%\usepackage{tikz}
%\usetikzlibrary{shadows}
%\usetikzlibrary{fit}
%\usetikzlibrary{shapes}
%\usetikzlibrary{backgrounds}


\usepackage{mcode}
%\usepackage{pstricks,pst-node,pst-text,pst-3d}
\parindent0cm % Abs�tze nicht einr�cken 

%---- neue Umgebung f�r Aufgaben
\theoremstyle{break}
\theoremheaderfont{\Large \bf}
\theorembodyfont{\normalfont}


% Definieren einer neuen Farbe
\definecolor{light-gray}{gray}{.9}

\newcounter{zaehler}     % neuen Z�hler einf�hren
\stepcounter{zaehler}    % Z�hler einen hochz�hlen

\newenvironment{aufg}%
%---- Header
{\begin{samepage}
\colorbox{light-gray}{                         % Box in gray
 \makebox[\textwidth]{                           % Box in linewidth
\textbf{Aufgabe} \arabic{zaehler} :}}\\[0.1cm]       % Header
%\begin{minipage}{0.5cm} \end{minipage}    % Insert 0.5cm
\begin{minipage}{\textwidth}}
%-----  foot
{\end{minipage} \nopagebreak %\begin{minipage}{1cm} \end{minipage}
\\[0.1cm] 
%\begin{minipage}{0.1cm} \end{minipage} 
%\hrulefill \begin{minipage}{1cm} \end{minipage}\\[1cm]  
\stepcounter{zaehler}                           % increase counter
 \end{samepage}%
}

%-------------------------------------------------------------------------------
\begin{document}
%-------------------------------------------------------------------------------

%--------------------------------------------------- Header
\begin{center}
\textbf{\LARGE Einf\"uhrung in MATLAB }\\
\end{center}
\begin{minipage}{6cm}
Dr. J. Schulz\\
\end{minipage}\hfill
\begin{minipage}{2cm}
\textbf{Einheit 5}\\
%30.07.2007
\end{minipage}\\[0.3cm]


%-----------------------------------------------------------------------------------
\begin{aufg}[0]
Schreiben Sie eine Funktion, die Funktionen $f:[-1,1] \times [-1,1] \
\rightarrow \mathbb{R}$ plottet. Das Intervall $[-1,1]$ soll
dabei jeweils in $N$ Punkte zerlegt werden. Die Funktion $f$ soll als
function-handle \"ubergeben werden. Schreiben Sie die Funktion so, dass wahlweise
nur die Funktion $f$ \"ubergeben werden kann (dann $N=50$) oder aber $f$ und $N$.
\end{aufg}


%------------------------------------------------------------------------------

\begin{aufg}[0]
Berechnen Sie mit Hilfe von \mcode{integral2.m} approximativ
\[ \int_{-3}^3 x^3 dx, \quad \text{und} \quad \int_1^{4} x^2 \sin ( \pi x) dx\]
für $N=50$. 
\end{aufg}
%-----------------------------------------------------------------------------------


% %-----------------------------------------------------------------------------------
% \begin{aufg}[0]
% Plotten Sie   auf dem
%   Intervall $[-\pi, 2 \pi]$ die Funktion $h(x)=\sin (x) {\cos
%   (2x)}$. Unterteilen Sie  die $x$-Achse in
%   $\pi/2$-Schritte, und beschriften Sie diese. Fügen Sie eine Legende
%   in Schriftgröße $21$ hinzu. Ändern Sie auch die Schriftgröße der
%   Achsenbeschriftung.
% \end{aufg}


\begin{aufg}[0]
Sind die folgenden Vektoren linear unabhängig?
{ \[
v_1=
\left( \begin{array}{c} 0 \\ 1 \\ 0 \\ 1 \end{array} \right), 
\ v_2=
\left( \begin{array}{c} 1 \\ 2 \\ 3 \\ 4 \end{array} \right),
\ v_3=
 \left( \begin{array}{c} 1 \\ 0 \\ 1 \\  0\end{array} \right),
\ v_4=
 \left( \begin{array}{c} 0 \\ 0 \\ 1 \\ 1 \end{array} \right)
. \] }
\end{aufg}



%--------------------------------------------------
\begin{aufg}[0]
Schreiben sie das Programm \mcode{randwertaufgabe} um in eine Funktion welche als Inputparameter 
den Parameter $n$ erhält. Die Funktion soll prüfen ob der Parameter n in dem Bereich 20-200 liegt und
falls nicht die Funktion abbrechen. Das Resultat der Berechnung soll als Vektor zurückgeben werden.

\emph{Hinweis:} das Abbrechen der Funktion kann mit \mcode{return} erreicht werden.
\end{aufg}



%--------------------------------------------------
\begin{aufg}[0]
Berechnen Sie die Frobenius-Norm (Schauen Sie in der Hilfe nach wie das geht)
\[ \|A \|_F := \sqrt{ \sum_{i,j=1}^n a_{ij}^2 }, \quad A=(a_{ij}) \in
\mathbb{R}^{n \times n}  \]
der Vandermode Matrix 
\lstinline!vander(0:0.02:1)! 
\end{aufg}

\newpage

%--------------------------------------------------
\begin{aufg}[0]
\label{randwert}
Zerlegen Sie das Intervall $[0,1]$ durch 
  \lstinline!0:(1/101):1!. Berechnen Sie mit Hilfe von 
Finiten Differenzen eine approx. Lösung von
{
\begin{eqnarray*}
-u''(x) & = & 1, \quad x \in (0,1)\\
u(0) & = & u(1) =0
\end{eqnarray*}}
%\vspace*{-0.5cm}
\end{aufg}


%--------------------------------------------------
\begin{aufg}[0]
Ändern Sie in Aufgabe \ref{randwert} die rechte Seite $1$ in
  $\sin(4 \pi x)$ und berechnen Sie eine Näherungslösung. 
\end{aufg}


%\newpage
\begin{aufg}[0]
Berechnen Sie die Eigenwerte und Eigenvektoren der Matrix
\[ A = \left( \begin{array}{cccc}
30 & 1 & 2 & 3\\
4 & 15 & -4 & -2\\
-1 & 0 & 3 & 5\\
-3 & 5 & 0 & -1 \end{array}
\right) \]
Bestimmen Sie auch die $QR$-Zerlegung.
\end{aufg}


\begin{aufg}[0]
\begin{itemize}
\item Interpolieren Sie an den durch \lstinline!x=linspace(-5,5,13)! gegebenen Stellen  die Funktion $f(x):=x^2\exp(-|x|)$.
\item Berechnen Sie approximativ den maximalen Fehler zwischen $f$ und
  ihrer Interpolierenden auf $[-5,5]$. 
(Hinweis: Befehl \lstinline!max!)
\item Ändern Sie den Vektor der Stützstellen
  \lstinline!x=linspace(-5,5,13)!, so dass 
\[ x_i = - 5 \cos(\pi (i-1)/12), \quad i=1, \dots , 13. \]
Berechnen Sie erneut den maximalen Fehler.
\item Betrachten Sie auch die Stützstellen
\[ x_i = - 5 \cos(\pi (i-1)/49), \quad i=1, \dots , 50. \] 
\end{itemize}
\end{aufg}


%Aufgabe fuer dunnbesetzte Matrizen
% Aufgabe(n) fuer reshape und find

\end{document}
