\documentclass[a4paper,10pt,DIV15]{scrartcl}
\usepackage[psamsfonts]{amssymb}
\usepackage{amsmath}
\usepackage[svgnames]{xcolor} %color definitions

\usepackage{fontspec,xunicode,xltxtra}
%\usepackage{fontspec,xunicode}
%\usepackage{polyglossia}
%\setdefaultlanguage[spelling=new, latesthyphen=true]{german}
%\setsansfont{DejaVu Sans}
%\setsansfont{Verdana}
%\setsansfont{Arial}
%\setromanfont[Mapping=tex-text]{Linux Libertine}
%\setsansfont[Mapping=tex-text]{Myriad Pro}
%\setmonofont[Mapping=tex-text]{Courier New}

%\setsansfont{Linux Biolinum}

\usepackage[ngerman]{babel}
\selectlanguage{ngerman}

%
% math/symbols
%
\usepackage{amssymb}
\usepackage{amsthm}
% \usepackage{latexsym}
\usepackage{amsmath}
%\usepackage{amsxtra} %Weitere Extrasymbole
%\usepackage{empheq} %Gleichungen hervorheben
%\usepackage{bm}
 %\bm{A} Boldface im Mathemodus

\usepackage{multimedia}
%\usepackage{tikz}

\usepackage{cellspace}
\setlength{\cellspacetoplimit}{2pt}
\setlength{\cellspacebottomlimit}{2pt}

%%%%%%%%%%%%%%%%%% Fuer Frames [fragile]-Option verwenden!
%Programm-Listing
%%%%%%%%%%%%%%%%%%
%Listingsumgebung fuer verbatim
%Grauhinterlegeter Text
%Automatischer Zeilenumbruch ist aktiviert
%\usepackage{listings}
\usepackage[framed]{mcode}
%\usepackage{mcode}
% This command allows you to typeset syntax highlighted Matlab
% code ``inline''.
% mcode fuer matlab

\definecolor{lgray}{gray}{0.80}
\definecolor{gray}{gray}{0.3}
\definecolor{darkgreen}{rgb}{0,0.4,0}
\definecolor{darkblue}{rgb}{0,0,0.8}
\definecolor{key}{rgb}{0,0.5,0} 
\newcommand{\imatlab}[1]{\lstset{basicstyle=\color[gray]{0.6}}\lstinline|#1|}
\newcommand{\isage}[1]{\lstset{basicstyle=\color[gray]{0.3}}\lstinline|#1|}
%\lstset{backgroundcolor=\color{lgray}, frame=single, basicstyle=\ttfamily, breaklines=true}
%\lstnewenvironment{sage}{\lstset{,language=python, keywordstyle=color{blue},    commentstyle=color{green}, emphstyle=\color{red}, %frame=single, stringstyle=\color{red}, basicstyle=\ttfamily, ,mathescape =true,escapechar=§}}{}

\lstnewenvironment{matlab}[1][]{\lstset{xleftmargin=0.2cm,frame=none,backgroundcolor=\color{white},basicstyle=\color{darkblue}\ttfamily\small,#1}}{} 
\lstnewenvironment{matlabin}[1][]{\lstset{
%language=python,
backgroundcolor=\color[gray]{0.9},
breaklines=true,
basicstyle=\ttfamily\small,
%otherkeywords={ =},
%keywordstyle=\color{blue},
%stringstyle=\color{darkgreen},
showstringspaces=false,
%emph={for, while, if, elif, else, not, and, or, printf, break, continue, return, end, function},
%emphstyle=\color{blue},
%emph={[2]True, False, None, self, NaN, NULL},
%emphstyle=[2]\color{key},
%emph={[3]from, import, as},
%emphstyle=[3]\color{blue},
%upquote=true,
%morecomment=[s]{"""}{"""},
%commentstyle=\color{gray}\slshape,
%framexleftmargin=1mm, framextopmargin=1mm, 
%title=\tiny matlab,
frame=single,
%mathescape =true,
%escapechar=§
,#1}
}{} 
\newcommand{\matinput}[1]{\lstset{
%language=python,
backgroundcolor=\color[gray]{0.9},
breaklines=true,
basicstyle=\ttfamily\small,
%otherkeywords={ =},
%keywordstyle=\color{blue},
%stringstyle=\color{darkgreen},
showstringspaces=false,
%emph={for, while, if, elif, else, not, and, or, printf, break, continue, return, end, function},
%emphstyle=\color{blue},
%emph={[2]True, False, None, self, NaN, NULL},
%emphstyle=[2]\color{key},
%emph={[3]from, import, as},
%emphstyle=[3]\color{blue},
%upquote=true,
%morecomment=[s]{"""}{"""},
%commentstyle=\color{gray}\slshape,
%framexleftmargin=1mm, framextopmargin=1mm, 
%title=\tiny matlab,
frame=single}\lstinputlisting{#1}}

\lstnewenvironment{pyout}[1][]{\lstset{language=python,xleftmargin=0.2cm,frame=none,backgroundcolor=\color{white},basicstyle=\color{darkblue}\ttfamily\small,#1}}{}
\lstnewenvironment{pyin}[1][]{\lstset{
language=python,
backgroundcolor=\color[gray]{0.7},
breaklines=true,
basicstyle=\ttfamily\small,
%otherkeywords={ =},
keywordstyle=\color{blue},
stringstyle=\color{darkgreen},
showstringspaces=false,
emph={class, pass, in, for, while, if, is, elif, else, not, and, or,
def, print, exec, break, continue, return},
emphstyle=\color{blue},
emph={[2]True, False, None, self},
emphstyle=[2]\color{key},
emph={[3]from, import, as},
emphstyle=[3]\color{blue},
upquote=true,
morecomment=[s]{"""}{"""},
%commentstyle=\color{gray}\slshape,
%framexleftmargin=1mm, framextopmargin=1mm, 
%title=\tiny python,
%caption=python,
frame=single,
%frameround=tttt,
mathescape =true,
escapechar=§,
#1
}
}{}

%\usepackage{caption}
%\DeclareCaptionFont{white}{ \color{white} }
%\DeclareCaptionFormat{listing}{
%  \colorbox[cmyk]{0.43, 0.35, 0.35,0.01 }{
%      \parbox{\textwidth}{\hspace{15pt}#1#2#3}
%        }
%        }
%        \captionsetup[lstlisting]{ format=listing, labelfont=white, textfont=white, singlelinecheck=false, margin=0pt, font={bf,footnotesize} }


\usepackage{mydef}
%\usepackage{cmap} % you can search in the pdf for umlauts and ligatures
\usepackage{colonequals} %corrects the definition-symbols \colonequals (besides others)

\usepackage{ifthen}

%%%%%%%%%%%%%%%%%%%
%Neue Definitionen
%%%%%%%%%%%%%%%%%%%

%Newcommands
\newcommand{\Fun}[1]{\mathcal{#1}}      %Mathcal fuer Funktoren
\newcommand{\field}[1]{\mathbb{#1}}     %Grundkoerper ?? in mathds

\newcommand{\A}{\field{A}}              %Affines A
\newcommand{\Fp}{\field{F}_{\!p}}       %Endlicher Koerper mit p Elementen
\newcommand{\Fq}{\field{F}_{\!q}}       %Endlicher Koerper mit q Elementen
\newcommand{\Ga}{\field{G}_{a}}         %Add Gruppenschema
\newcommand{\K}{\field{K}}              %Generischer Koerper 
\newcommand{\N}{\field{N}}              %Nat Zahlen
\newcommand{\Pj}{\field{P}}             %Projektives P
\newcommand{\R}{\field{R}} 		%Reelle Zahlen
\newcommand{\Q}{\field{Q}}              %Rationale Zahlen  
\newcommand{\Qt}{\field{H}}             %Quaternionen 
\newcommand{\V}{\field{V}}              %Vektorbuendel V
\newcommand{\Z}{\field{Z}}              %Ganze Zahlen
\DeclareMathOperator{\Real}{Re}

\newcommand{\fdg}{\;|\;}                 %fuer die gilt

%Operatoren
\DeclareMathOperator{\Abb}{Abb}
%\usepackage{sagetex}


%
% Aufgaben
%
\parindent0cm % Abs�tze nicht einr�cken 
% Definieren einer neuen Farbe
\definecolor{light-gray}{gray}{.9}

\newcounter{zaehler}     % neuen Z�hler einf�hren
\newenvironment{aufgn}[2][0]
%---- Header
{\begin{samepage}%
%\colorbox{light-gray}{%                         % Box in gray
% \makebox[\textwidth]{%                           % Box in linewidth
%\textbf{Aufgabe \arabic{zaehler} } }\hspace{-\textwidth}\makebox[\textwidth]{\hfill #1 Punkte} }\\[0.05cm]       % Header
\dotfill\\
{\large\textbf{Aufgabe \arabic{zaehler} \ifthenelse{ \equal{#2}{} }{}{: \emph{ #2 } }}\ifthenelse{-1=#1}{(testierbar)}{}\ifthenelse{0=#1 \or -1=#1}{}{\hfill #1 Punkte} }\\[0.4cm]
%{\large\textbf{Aufgabe \arabic{zaehler}  #2 }\ifthenelse{-1=#1}{(testierbar)}{}\ifthenelse{0=#1 \or -1=#1}{}{\hfill #1 Punkte} }\\[0.4cm]
\begin{minipage}{\textwidth}%
}%
%-----  foot
{\end{minipage}\nopagebreak%\begin{minipage}{1cm} \end{minipage}
%\\ 
%\begin{minipage}{0.1cm} \end{minipage} 
%\hrulefill \begin{minipage}{1cm} \end{minipage}\\[1cm]  
\stepcounter{zaehler}                           % increase counter
\end{samepage}%
\\%
\bigskip%
}


\newenvironment{aufg}[1][0]
%---- Header
{\begin{samepage}%
\refstepcounter{zaehler}% increase counter
%\colorbox{light-gray}{%                         % Box in gray
% \makebox[\textwidth]{%                           % Box in linewidth
%\textbf{Aufgabe \arabic{zaehler} } }\hspace{-\textwidth}\makebox[\textwidth]{\hfill #1 Punkte} }\\[0.05cm]       % Header
\dotfill\\
{\large\textbf{Aufgabe \arabic{zaehler} }\ifthenelse{-1=#1}{(testierbar)}{}\ifthenelse{0=#1 \or -1=#1}{}{\hfill #1 Punkte} }\\[0.4cm]
\begin{minipage}{\textwidth}%
}%
%-----  foot
{\end{minipage}\nopagebreak%\begin{minipage}{1cm} \end{minipage}
%\\ 
%\begin{minipage}{0.1cm} \end{minipage} 
%\hrulefill \begin{minipage}{1cm} \end{minipage}\\[1cm]  
\end{samepage}%
\\%
\bigskip%
}


%\usepackage{tikz}
%\usetikzlibrary{shadows}
%\usetikzlibrary{fit}
%\usetikzlibrary{shapes}
%\usetikzlibrary{backgrounds}


%-------------------------------------------------------------------------------
\begin{document}
%-------------------------------------------------------------------------------

%--------------------------------------------------- Header
\begin{center}
\textbf{\LARGE Wissenschaftliches Rechnen mit Matlab/Python}\\
\end{center}
\begin{minipage}{6cm}
Jochen Schulz
\end{minipage}\hfill

\begin{minipage}{2cm}
\textbf{Einheit 5}\\
%30.07.2007
\end{minipage}\\[0.3cm]

%-----------------------------------------------------------------------------------
\begin{aufg}
Lösen Sie  die Dgl. $\frac{d}{dt} y(t) = f(y)$ mit
  \[ f(y_1,y_2)=(y_1 -y_1 y_2, -y_2+y_2 y_1)\]
 und Anfangswert
  $y(0)=(0.5,0.5)$. Plotten Sie die Lösung im $y_1y_2$-Diagramm.
\end{aufg}
%------------------------------------------------------------------------------
\begin{aufg}
Betrachten Sie das folgende System gew\"ohnlicher Differentialgleichungen
\begin{eqnarray*}
y_1'(t) & = & - 0.04 y_1(t) + 10^4 y_2(t)y_3(t) \\
y_2'(t) & = & 0.04 y_1(t) - 10^4 y_2(t) y_3(t) - 3 \cdot 10^7 y_2(t)^2\\
y_3'(t) & = & 3 \cdot 10^7 y_2(t)^2.
\end{eqnarray*}
L\"osen Sie  das System zum Anfangswert $(1,0,0)$ an $0$ auf dem Zeitintervall $0 \leq
t \leq 3$. Nutzen Sie die Solver \mcode{ode45} und \mcode{ode15s} und vergleichen Sie die Ergebnisse.
\end{aufg}
%-----------------------------------------------------------------------------------
\begin{aufg}
Berechnen Sie mit Hilfe von \mcode{integral2.m} approximativ
\[ \int_{-3}^3 x^3 dx, \quad \text{und} \quad \int_1^{4} x^2 \sin ( \pi x) dx\]
für $N=50$. 
\end{aufg}
%-----------------------------------------------------------------------------------
\begin{aufg}
Erstellen Sie die Funktion 
\[ g(x) := \left \{ \begin{array}{ll} 1-|x|^2, & |x|<1 \\
0, & |x| \geq  1 \\
\end{array} \right. \quad x \in \mathbb{R}.  \vspace*{-0.3cm}
\]
Plotten Sie $g$ mit dem Befehl \mcode{ezplot}.  
\end{aufg}
%-----------------------------------------------------------------------------------
\begin{aufg}
Plotten Sie   auf dem
  Intervall $[-\pi, 2 \pi]$ die Funktion $h(x)=\sin (x) {\cos
  (2x)}$. Unterteilen Sie  die $x$-Achse in
  $\pi/2$-Schritte, und beschriften Sie diese. Fügen Sie eine Legende
  in Schriftgröße $21$ hinzu. Ändern Sie auch die Schriftgröße der
  Achsenbeschriftung.
\end{aufg}
%-----------------------------------------------------------------------------------
% \begin{aufg}
% Modifizieren Sie das Programm \mcode{interpolation.m}. Schreiben Sie
% eine Funktion, die  eine als String gegebene Funktion $f$ an $N$
% äquidistanten Punkten in $[-1,1]$ interpoliert. \\
% {\it Hinweis:} Verwenden Sie den Befehl \mcode{f=fcnchk(f)} für den
% String $f$.
% \end{aufg}
%-----------------------------------------------------------------------------------
\begin{aufg}
Schreiben Sie eine Funktion, die Funktionen $f:[-1,1] \times [-1,1] \
\rightarrow \mathbb{R}$ plottet. Das Intervall $[-1,1]$ soll
dabei jeweils in $N$ Punkte zerlegt werden. Die Funktion $f$ soll dabei als
function-handle \"ubergeben werden. Schreiben Sie die Funktion so, dass wahlweise
nur die Funktion $f$ \"ubergeben werden kann (dann $N=50$) oder aber $f$ und $N$.
\end{aufg}

\begin{aufg}[0]
woanders hin?
Sind die folgenden Vektoren linear unabhängig?
{ \[
v_1=
\left( \begin{array}{c} 0 \\ 1 \\ 0 \\ 1 \end{array} \right), 
\ v_2=
\left( \begin{array}{c} 1 \\ 2 \\ 3 \\ 4 \end{array} \right),
\ v_3=
 \left( \begin{array}{c} 1 \\ 0 \\ 1 \\  0\end{array} \right),
\ v_4=
 \left( \begin{array}{c} 0 \\ 0 \\ 1 \\ 1 \end{array} \right)
. \] }
\end{aufg}
% \begin{aufg}
% Finde die Lösung $x$ von $Ax=b$ mit 
% \[ A:=\left( \begin{array}{cccc} 
% 2 & 3 & 4 & 5 \\ 
% 1 &    1 &    1 &    1\\
% 1 &    0 &    1 &    0 \\
% 9 &     3 &    2 &    1
% \end{array} \right) , \qquad b:=\left( \begin{array}{c} 
% 14\\
%      4\\
%      2\\
%     15
% \end{array} \right) . \] 
% \end{aufg}
% %--------------------------------------------------
% \begin{aufg}
% Finde die Lösung $x$ von $Ax=b$ mit 
% { \[ A:=\left( \begin{array}{ccc} 
% 1 & 2 & 3 \\ 
% 4 &    5 &    6\\
% 7 &    8 &    9  
% \end{array} \right) , \qquad b:=\left( \begin{array}{c} 
% 6\\
%      15\\
%      24
% \end{array} \right) . \] }
% \end{aufg}
% %--------------------------------------------------

% \begin{aufg}
% Schreiben sie das Programm \mcode{randwertaufgabe} um in eine Funktion welche als Inputparameter 
% den Parameter $n$ erhält. Die Funktion soll prüfen ob der Parameter n in dem Bereich 20-200 liegt und
% falls nicht das programm abbrechen. Das Resultat der Berechnung soll als Vektor zurückgeben werden.
% 
% \emph{Hinweis:} das Abbrechen des Programms kann mit \mcode{return} erreicht werden.
% \end{aufg}
%--------------------------------------------------
% \begin{aufg}
% Zerlegen Sie das Intervall $[0,1]$ durch 
%   \lstinline!0:(1/101):1!. Berechnen Sie mit Hilfe von 
% Finiten Differenzen eine approx. Lösung von
% {
% \begin{eqnarray*}
% -u''(x) & = & 1, \quad x \in (0,1)\\
% u(0) & = & u(1) =0
% \end{eqnarray*}}
% %\vspace*{-0.5cm}
% \end{aufg}

% %--------------------------------------------------
% \begin{aufg}
% Berechnen Sie die Frobenius-Norm 
% \[ \|A \|_F := \sqrt{ \sum_{i,j=1}^n a_{ij}^2 }, \quad A=(a_{ij}) \in
% \mathbb{R}^{n \times n}  \]
% der Vandermode Matrix 
% \lstinline!vander(0:0.02:1)! 
% \end{aufg}
% %--------------------------------------------------
% \begin{aufg}
% Ändern Sie in Aufgabe 11 die rechte Seite $1$ in
%   $\sin(4 \pi x)$ und berechnen Sie eine Näherungslösung. 
% \end{aufg}
%--------------------------------------------------
%\newpage
% \begin{aufg}
% Berechnen Sie die Eigenwerte und Eigenvektoren der Matrix
% \[ A = \left( \begin{array}{cccc}
% 30 & 1 & 2 & 3\\
% 4 & 15 & -4 & -2\\
% -1 & 0 & 3 & 5\\
% -3 & 5 & 0 & -1 \end{array}
% \right) \]
% Bestimmen Sie auch die $QR$-Zerlegung.
% \end{aufg}
% 
% \begin{aufg}
% Sei \lstinline!A=hilb(n)! und \lstinline!x=ones(n,1)!. Berechnen Sie
%   für $n=5$ und $n=15$ den Vektor $b=A*x$, \lstinline!norm(x-A\b)! und die Kondition
%   von $A$. Was stellen Sie fest? Erklären Sie das Ergebnis!
% \end{aufg}

\begin{aufg}[0]
\begin{itemize}
\item Interpolieren Sie an den durch \lstinline!x=linspace(-5,5,13)! gegebenen Stellen  die Funktion $f(x):=x^2\exp(-|x|)$.
\item Berechnen Sie approximativ den maximalen Fehler zwischen $f$ und
  ihrer Interpolierenden auf $[-5,5]$. 
(Hinweis: Befehl \lstinline!max!)
\item Ändern Sie den Vektor der Stützstellen
  \lstinline!x=linspace(-5,5,13)!, so dass 
\[ x_i = - 5 \cos(\pi (i-1)/12), \quad i=1, \dots , 13. \]
Berechnen Sie erneut den maximalen Fehler.
\item Betrachten Sie auch die Stützstellen
\[ x_i = - 5 \cos(\pi (i-1)/49), \quad i=1, \dots , 50. \] 
\end{itemize}
\end{aufg}

\end{document}
