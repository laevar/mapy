\documentclass[a4paper,10pt,DIV15]{scrartcl}
\usepackage[psamsfonts]{amssymb}
\usepackage{amsmath}
%\usepackage{latexsym}
\usepackage{theorem}


\usepackage{fontspec,xunicode,xltxtra}
\usepackage[ngerman]{babel}
\selectlanguage{ngerman}


\usepackage[svgnames,hyperref]{xcolor} %color definitions
%\usepackage{tikz}
%\usetikzlibrary{shadows}
%\usetikzlibrary{fit}
%\usetikzlibrary{shapes}
%\usetikzlibrary{backgrounds}


\usepackage{mcode}
%\usepackage{pstricks,pst-node,pst-text,pst-3d}
\parindent0cm % Abs�tze nicht einr�cken 

%---- neue Umgebung f�r Aufgaben
\theoremstyle{break}
\theoremheaderfont{\Large \bf}
\theorembodyfont{\normalfont}


% Definieren einer neuen Farbe
\definecolor{light-gray}{gray}{.9}

\newcounter{zaehler}     % neuen Z�hler einf�hren
\stepcounter{zaehler}    % Z�hler einen hochz�hlen

\newenvironment{aufg}%
%---- Header
{\begin{samepage}
\colorbox{light-gray}{                         % Box in gray
 \makebox[\textwidth]{                           % Box in linewidth
\textbf{Aufgabe} \arabic{zaehler} :}}\\[0.1cm]       % Header
%\begin{minipage}{0.5cm} \end{minipage}    % Insert 0.5cm
\begin{minipage}{\textwidth}}
%-----  foot
{\end{minipage} \nopagebreak %\begin{minipage}{1cm} \end{minipage}
\\[0.1cm] 
%\begin{minipage}{0.1cm} \end{minipage} 
%\hrulefill \begin{minipage}{1cm} \end{minipage}\\[1cm]  
\stepcounter{zaehler}                           % increase counter
 \end{samepage}%
}

%-------------------------------------------------------------------------------
\begin{document}
%-------------------------------------------------------------------------------

%--------------------------------------------------- Header
\begin{center}
\textbf{\LARGE Einf\"uhrung in MATLAB }\\
\end{center}
\begin{minipage}{6cm}
Dr. J. Schulz\\
\end{minipage}\hfill
\begin{minipage}{2cm}
\textbf{Einheit 5}\\
%30.07.2007
\end{minipage}\\[0.3cm]

%-----------------------------------------------------------------------------------
\begin{aufg}
Lösen Sie  die Dgl. $\frac{d}{dt} y(t) = f(y)$ mit
  \[ f(y_1,y_2)=(y_1 -y_1 y_2, -y_2+y_2 y_1)\]
 und Anfangswert
  $y(0)=(0.5,0.5)$. Plotten Sie die Lösung im $y_1y_2$-Diagramm.
\end{aufg}
%------------------------------------------------------------------------------
\begin{aufg}
Betrachten Sie das folgende System gew\"ohnlicher Differentialgleichungen
\begin{eqnarray*}
y_1'(t) & = & - 0.04 y_1(t) + 10^4 y_2(t)y_3(t) \\
y_2'(t) & = & 0.04 y_1(t) - 10^4 y_2(t) y_3(t) - 3 \cdot 10^7 y_2(t)^2\\
y_3'(t) & = & 3 \cdot 10^7 y_2(t)^2.
\end{eqnarray*}
L\"osen Sie  das System zum Anfangswert $(1,0,0)$ an $0$ auf dem Zeitintervall $0 \leq
t \leq 3$. Nutzen Sie die Solver \mcode{ode45} und \mcode{ode15s} und vergleichen Sie die Ergebnisse.
\end{aufg}
%-----------------------------------------------------------------------------------
\begin{aufg}
Berechnen Sie mit Hilfe von \mcode{integral2.m} approximativ
\[ \int_{-3}^3 x^3 dx, \quad \text{und} \quad \int_1^{4} x^2 \sin ( \pi x) dx\]
für $N=50$. 
\end{aufg}
%-----------------------------------------------------------------------------------
\begin{aufg}
Erstellen Sie die Funktion 
\[ g(x) := \left \{ \begin{array}{ll} 1-|x|^2, & |x|<1 \\
0, & |x| \geq  1 \\
\end{array} \right. \quad x \in \mathbb{R}.  \vspace*{-0.3cm}
\]
Plotten Sie $g$ mit dem Befehl \mcode{ezplot}.  
\end{aufg}
%-----------------------------------------------------------------------------------
\begin{aufg}
Plotten Sie   auf dem
  Intervall $[-\pi, 2 \pi]$ die Funktion $h(x)=\sin (x) {\cos
  (2x)}$. Unterteilen Sie  die $x$-Achse in
  $\pi/2$-Schritte, und beschriften Sie diese. Fügen Sie eine Legende
  in Schriftgröße $21$ hinzu. Ändern Sie auch die Schriftgröße der
  Achsenbeschriftung.
\end{aufg}
%-----------------------------------------------------------------------------------
\begin{aufg}
Modifizieren Sie das Programm \mcode{interpolation.m}. Schreiben Sie
eine Funktion, die  eine als String gegebene Funktion $f$ an $N$
äquidistanten Punkten in $[-1,1]$ interpoliert. \\
{\it Hinweis:} Verwenden Sie den Befehl \mcode{f=fcnchk(f)} für den
String $f$.
\end{aufg}
%-----------------------------------------------------------------------------------
\begin{aufg}
Schreiben Sie eine Funktion, die Funktionen $f:[-1,1] \times [-1,1] \
\rightarrow \mathbb{R}$ plottet. Das Intervall $[-1,1]$ soll
dabei jeweils in $N$ Punkte zerlegt werden. Die Funktion $f$ soll dabei als
function-handle \"ubergeben werden. Schreiben Sie die Funktion so, dass wahlweise
nur die Funktion $f$ \"ubergeben werden kann (dann $N=50$) oder aber $f$ und $N$.
\end{aufg}
\end{document}
