 %----------------------------------------------------------------
%  Einf�hrung in MuPAD
% 
% Blockpraktikum Februar 2005
% (2 Wochen)
%----------------------------------------------------------------

\documentclass[12pt]{article}
\usepackage{german, a4}
\usepackage{amsmath}
\usepackage{latexsym}
\usepackage{amssymb}
\usepackage[ansinew]{inputenc}
\usepackage{exscale}
\usepackage{theorem}
%\usepackage[dvips]{graphics}
%\usepackage{pstricks,pst-node,pst-text,pst-3d}
\usepackage{color}
\usepackage{times}                  % Postscript font!!!
\parindent0cm % Abs�tze nicht einr�cken 

%-- Selbstdefinierte Befehle und Parametereinstellungen ------------------------
\setlength{\headheight}{1.1\baselineskip}
\setlength{\textheight}{22cm}       
\setlength{\textwidth}{15cm}
\setlength{\topmargin}{-2cm} %oberer Rand bis Kopfzeile angehoben
\setlength{\oddsidemargin}{0.5cm} %linker Rand f"ur ungerade Seiten


%---- neue Umgebung f�r Aufgaben
\theoremstyle{break}
\theoremheaderfont{\Large \bf}
\theorembodyfont{\normalfont}


% Definieren einer neuen Farbe
\definecolor{light-gray}{gray}{.9}

\newcounter{zaehler}     % neuen Z�hler einf�hren
\stepcounter{zaehler}    % Z�hler einen hochz�hlen


\newenvironment{aufg}%
%---- Header
{\begin{samepage}
\colorbox{light-gray}{                         % Box in gray
 \makebox[\textwidth]{                           % Box in linewidth
\bf \color{red} Aufgabe \arabic{zaehler} :}}\\[0.2cm]       % Header
\begin{minipage}{0.5cm} \end{minipage} \hfill   % Insert 0.5cm
\begin{minipage}{13.5cm}}                         
%-----  foot
{\end{minipage} \nopagebreak \begin{minipage}{1cm} \end{minipage}
\\[0.2cm] 
\begin{minipage}{0.1cm} \end{minipage} 
\hrulefill \begin{minipage}{1cm} \end{minipage}\\[1cm]  
\stepcounter{zaehler}                           % increase counter
 \end{samepage}%
}


%-------------------------------------------------------------------------------
\begin{document}
%-------------------------------------------------------------------------------

%--------------------------------------------------- Header
\begin{center}
{\LARGE \bf Einf\"uhrung in MATLAB }\\
%{\LARGE \bf (Blockveranstaltung)}\\
%SS 2005\\[0.8cm]
\end{center}
\begin{minipage}{6cm}
Dr. G. Rapin,\\
%Th. Wassong\\
\end{minipage} \hfill 
\begin{minipage}{2cm}
{\bf Einheit 5}\\
%28.07.2005
\end{minipage}\\[1cm]

%-----------------------------------------------------------------------------------
\begin{aufg}
L�sen Sie  die Dgl $\frac{d}{dt} y(t) = f(y)$ mit
  \[ f(y_1,y_2)=(y_1 -y_1 y_2, -y_2+y_2 y_1)^t\]
 und Anfangswert
  $y(0)=(0.5,0.5)^t$. Plotten Sie die L�sung im $y_1y_2$-Diagramm.
\end{aufg}
%------------------------------------------------------------------------------
\begin{aufg}
Betrachten Sie das folgende System gew\"ohnlicher Differentialgleichungen
\begin{eqnarray*}
y_1'(t) & = & - 0.04 y_1(t) + 10^4 y_2(t)y_3(t) \\
y_2'(t) & = & 0.04 y_1(t) - 10^4 y_2(t) y_3(t) - 3 \cdot 10^7 y_2(t)^2\\
y_3'(t) & = & 3 \cdot 10^7 y_2(t)^2.
\end{eqnarray*}
L\"osen Sie  das System zum Anfangswert $(1,0,0)$ an $0$ auf dem Zeitintervall $0 \leq
t \leq 3$. Nutzen Sie die Solver \verb+ode45+ und \verb+ode15s+ und vergleichen Sie die Ergebnisse.
\end{aufg}
%-----------------------------------------------------------------------------------
\begin{aufg}
Berechnen Sie mit Hilfe von \verb+integral2.m+ approximativ
\[ \int_{-3}^3 x^3 dx, \int_1^{4} x^2 \sin ( \pi x) dx\]
f�r $N=50$. 
\end{aufg}
\newpage
%-----------------------------------------------------------------------------------
\begin{aufg}
Erstellen Sie die Funktion 
\[ g(x) := \left \{ \begin{array}{ll} 1-|x|^2, & |x|<1 \\
0, & |x| \geq  1 \\
\end{array} \right. \quad x \in \mathbb{R}.  \vspace*{-0.3cm}
\]
Plotten Sie $g$ mit dem Befehl \verb+ezplot+.  
\end{aufg}
%-----------------------------------------------------------------------------------
\begin{aufg}
Plotten Sie   auf dem
  Intervall $[-\pi, 2 \pi]$ die Funktion $h(x)=\sin (x) {\cos
  (2x)}$. Unterteilen Sie  die $x$-Achse in
  $\pi/2$-Schritte, und beschriften Sie diese. F�gen Sie eine Legende
  in Schriftgr��e $21$ hinzu. �ndern Sie auch die Schriftgr��e der
  Achsenbeschriftung.
\end{aufg}
%-----------------------------------------------------------------------------------
\begin{aufg}
Modifizieren Sie das Programm \verb+interpolation.m+. Schreiben Sie
eine Funktion, die  eine als String gegebene Funktion $f$ an $N$
�quidistanten Punkten in $[-1,1]$ interpoliert. \\
{\it Hinweis:} Verwenden Sie den Befehl \verb+f=fcnchk(f)+ f�r den
String $f$.
\end{aufg}
%-----------------------------------------------------------------------------------
\begin{aufg}
Schreiben Sie eine Funktion, die Funktionen $f:[-1,1] \times [-1,1] \
\rightarrow \mathbb{R}$ plottet. Das Intervall $[-1,1]$ soll
dabei jeweils in $N$ Punkte zerlegt werden. Die Funktion $f$ soll dabei als
function-handle \"ubergeben werden. Schreiben Sie die Funktion so, dass wahlweise
nur die Funktion $f$ \"ubergeben werden kann (dann $N=50$) oder aber $f$ und $N$.
\end{aufg}
%-----------------------------------------------------------------------------------
\begin{aufg}
Plotten Sie die Sierpinski-Dreiecke zum Level $5$. Aus wievielen grafischen
Objekten besteht die Grafik? Entfernen Sie aus der Grafik alle Dreiecke, die einen Eckpunkt $(x,y)$
besitzen f\"ur den $x+y \geq 1/2$ gilt.
\end{aufg}
%-----------------------------------------------------------------------------------
\newline
\begin{aufg}
Erzeugen Sie durch Kopieren grafischer Objekte $5$ Grafiken mit
Sierpinski-Dreiecken zum Level $5$, wobei Sie nur einmal das Skript
\verb+sierpinski_plot+ ausf\"uhren d\"urfen.
\end{aufg}
%-------------------------------------------------------------------------------
\end{document}
%-------------------------------------------------------------------------------
