%
\subtitle{Einheit 6 \\ Grafik Handle \\ GUI -- Programmierung}
\maketitle
\part{Einheit 6, Teil 1 \\ Grafik-Handle}
\begin{frame}[fragile]\frametitle{Grafik-Objekte}
\begin{itemize}
\item Jedes Grafik-Objekt ist eindeutig bestimmt durch seine Eigenschaften
(engl. {\it properties}). 
\item Die Eigenschaften  der Objekte sind  in  sogenannten
{\it handles} gespeichert. Sie liegen dort als \mcode{double}
(Gleitkommazahl) vor.  
\item Mit Hilfe dieser Handle k\"onnen die Eigenschaften 
existierender Grafik-Objekte ge\"andert werden.    
\end{itemize}
\end{frame}
%
% Slide
%
\begin{frame}[fragile]\frametitle{Hierachische Struktur\\ von Grafik-Objekten}
\vspace*{0.5cm}
{\tiny
\begin{tabular}{lp{3cm}p{6cm}}

Level 1 & Root & Das Wurzel-Objekt  bezieht sich auf den
Computerschirm. Es wird automatisch erzeugt und es gibt nur eins. Alle
anderen Objekte stammen von diesem ab.\\

Level 2 & Figure & Diese Objekte sind durch die einzelnen
Grafik-Fenster gegeben. Alle Figuren sind Kinder von Root. \\

Level 3 & Axes, Uicontrol, Uimenu, Uicontextmenu & Die Objekte
Ui... sind benutzerdefinierte Grafik-Interfaces (werden hier nicht
betrachtet).\linebreak
 Die Axes Objekte
definieren eine Region im Grafikfenster und ordnen ihre Kinder in
dieser Region an.\\

Level 4 & Image, Line, Text, Surface,... & Die Objekte bestimmen das
Aussehen der Grafik-Fenster. Sie sind Kinder der Axes Objekte.\\ 
\end{tabular}
}
\end{frame}
% 
% Slide
%
\begin{frame}[fragile]\frametitle{Umgang mit dem Grafik-Handle}
\begin{itemize}
\item Konstruktion einer Grafik
\begin{lstlisting}
>> x=0:0.2:2*pi; plot(x,sin(x))
\end{lstlisting}
\item Abfragen und Bedeutung der Handles aller Objekte\\
\begin{minipage}{4cm}
\begin{lstlisting}
>> h=findobj
h =
         0
    1.0000
  100.0015
    3.0016
\end{lstlisting}
\end{minipage} \hfill
\begin{minipage}{4cm}
\begin{lstlisting}
>> get(h,'type')
ans = 
    'root'
    'figure'
    'axes'
    'line'
\end{lstlisting} 
\end{minipage}
\end{itemize}
\end{frame}
% 
% Slide
%
\begin{frame}[fragile]\frametitle{Umgang mit dem Grafik-Handle }
\begin{itemize}
\item Momentane Einstellung des 'Axes'-Objekts
\begin{lstlisting}
>>  set(h(3))
        ALim
        ALimMode: [ {auto} | manual ]
        AmbientLightColor
        Box: [ on | {off} ]
              \( {\bf \cdots } \)
\end{lstlisting}
\item Eigenschaften des 'Line'-Objekts
\begin{lstlisting}
>> set(h(4))
        Color
        EraseMode: [ {normal} | background | xor | none ]
        LineStyle: [ {-} | -- | : | -. | none ]
        LineWidth
              \({\bf \cdots } \)
\end{lstlisting}
\end{itemize}
 \end{frame}
% 
% Slide
%
\begin{frame}[fragile]\frametitle{Umgang mit dem Grafik-Handle }
\begin{itemize}
\item \"Andern des Markers:
\begin{lstlisting}
>> set(h(4),'Marker','s','MarkerSize',4)
\end{lstlisting}
\item \"Andern der Einheiten auf der x-Achse
\begin{lstlisting}
set(h(3),'XTick',[0 pi/2 pi 2*pi])
set(h(3),'XtickLabel','0|pi/2|pi|2pi')
\end{lstlisting}
\item \mcode{gca}, \mcode{gcf} und \mcode{gco} sind die Handle f\"ur die
  aktuelle {\it Axes}, die aktuelle {\it Figure} und das aktuelle {\it
  Objekt} des 4. Levels.\\
\begin{minipage}{5cm}

\begin{lstlisting}
>> set(gcf,'Name','Tolle Abb.')
>> set(gca,'Fontsize',15) 
\end{lstlisting}
\end{minipage} \hfill
\begin{minipage}{4cm}
\begin{lstlisting}
>> l=legend('sin(x)');
>>  set(l,'FontSize',20); 
\end{lstlisting}
\end{minipage}\\
\end{itemize}
\end{frame}
%
% Slide
%
\begin{frame}[fragile]\frametitle{Hierachie}
\begin{itemize}
\item Die Grafikobjekte sind hierachisch angeordnet. Sie haben also Kinder
und V\"ater. 
\item Informationen zu den zugeordneten Kindern
\begin{lstlisting}
a=get(l,'Children'), get(a,'type')
\end{lstlisting}
\item Information zum Vater
\begin{lstlisting}
d=get(l,'Parent'), get(d,'type')
\end{lstlisting}
\item \"Andern der Kindeigenschaften
\begin{lstlisting}
set(a(3),'Color',[1 0 0])
\end{lstlisting}
\end{itemize}
\end{frame}
%
% Slide
%
\begin{frame}[fragile]\frametitle{Beispiel}
\begin{lstlisting}
%---------------------------------
% current_figure.m
%
% Aendert die Ueberschriften der Figures so ab,
% das jeweis das aktuelle Fenster die Ueberschrift 
% 'aktuell' hat und die anderen 'nicht aktuell' 
%----------------------------------

% Handle aller Figures
a=get(0,'children')

% Beschrifte alle Figures als 'nicht aktuell'
for i=1:length(a)
    set(a(i),'name','nicht aktuell')
end

% Ueberschreibe den Namen des aktuellen Fensters
set(gcf,'name','aktuell')
\end{lstlisting}
\end{frame}
%
% Slide
%
\begin{frame}[fragile]\frametitle{Umgang mit Objekten}
\begin{itemize}
\item L\"oschen von Objekten: \alert{ \mcode{delete(handle)}}

\item Kopieren von Objekten: \alert{ \mcode{copyobj(handle,new_parent)}}.
H\"angt das Objekt mit Handle \mcode{handle} an einen anderen Vater
\mcode{new_parent} an.

\item Finden von Objekten mit bestimmten Eigenschaften: \alert{
\mcode{findobj(Eigenschaft, Spezifikation)}}.
\end{itemize}
\end{frame}
%
% Slide
%
\begin{frame}[fragile]\frametitle{Defaulteinstellungen}
\begin{itemize}
\item Ansehen der Defaultwerte
\begin{lstlisting}
a=get(0,'Factory')
\end{lstlisting}
\item \"Andern der Defaultwerte
\begin{lstlisting}
>> set(0,'DefaultLineLineWidth',3)
>> set(0,'DefaultFigureColor','g')
>> set(0,'DefaultAxesFontSize',20)
\end{lstlisting}
\item Einstellungen gelten immer auch f\"ur alle Kinder und Kindes-Kinder.
\end{itemize}
\end{frame}
%
% Slide
%
\begin{frame}[fragile]\frametitle{Defaulteinstellungen II}
\begin{itemize}
\item L\"oschen der Defaulteinstellung
\begin{lstlisting}
set(0,'DefaultAxesFontSize','remove')
\end{lstlisting}
\item Die Defaulteinstellungen k\"onnen in der Datei 'startup.m'  (Linux/Unix)
  bzw. \mcode{matlabroot\toolbox\local} (Windows) abgelegt werden. Sie werden so
  beim Start von MATLAB automatisch eingeladen. 
\item Unter Linux muss die Datei durch den Suchpfad \mcode{path} erreichbar
  sein.  
\end{itemize} 
\end{frame}
\begin{frame}[fragile]\frametitle{Ein Beispiel}
\begin{lstlisting}
close all;

% Generate Grid
x=linspace(-2,2,30);
y=linspace(-2,2,30);
[X,Y]=meshgrid(x,y);
% Function Values
Z=exp(-X.^2-Y.^2).*sin(pi*X.*Y);

% Plot graphic
h = surf(X,Y,Z);
k = get(h,'Parent'); 
Angles = get (k,'View');

for l = 1:360
    set(k,'View',Angles+l);
    pause(1/60)
end
\end{lstlisting}
\end{frame}
%
%=============================================================================
%
%
% Slide
%
\part{Einheit  6, 2. Teil\\Grapical User Interface (GUI)}
%
% Slide
%
\begin{frame}[fragile]\frametitle{GUIs in MATLAB}
\begin{itemize}
\item Das Graphical User Interface(GUI) erm\"oglicht das Steuern von Programmen
  mit Hilfe der grafischen Oberfl\"ache. 
\item Hierdurch k\"onnen Programme von Usern ohne MATLAB-Kenntnisse genutzt
  werden. 
\item Im Wesentlichen werden GUIs durch Grafik-Objekte der Typen
  \mcode{uimenus}, \mcode{uicontextmenu} und \mcode{uicontrols} gesteuert. Sie
  sind auf der gleichen Hierachie-Ebene wie \mcode{axes}-Objekte. 
\item Es existiert eine grafische Oberfl\"ache zur Erstellung von GUIs. Aufruf
  \alert{ \mcode{guide}}
\end{itemize}
\end{frame}
%
% Slide
%
\begin{frame}[fragile]\frametitle{Vordefinierte GUIs f\"ur \newline Dialogboxen}
\begin{itemize}
\item { \blue \mcode{helpdlg}}: Hilfebox
\item { \blue \mcode{msgbox}}: Eine beliebige Nachricht
\item { \blue \mcode{warndlg}}: Anzeige einer Warnung
\item { \blue \mcode{inputdlg}}: Abfragen einer Gr\"o{\ss}e
\item { \blue \mcode{questdlg}}: Frage
\end{itemize}
\alert{Beispiel:}
\begin{lstlisting}
>> h1 = warndlg('NameFenster','Nachricht')
>> h2 = errordlg('NameFenster','Nachricht')
>> ans = questdlg('NameFenster','Nachricht')
>> ant = inputdlg({'Frage 1','Frage 2','Frage3'},...
  'NameFenster',[1 2 3], {'defAnt1','defAnt2','defAnt3'})
\end{lstlisting}
\end{frame}
%
% Slide
%
\begin{frame}[fragile]\frametitle{Grafik-Objekte und GUIs}
\begin{itemize}
\item \alert\alert{ uicontrols}\newline
Interaktive grafische Objekte, die Aktionen steuern oder bestimmte Optionen setzen.
\item \alert\alert{ uimenus}\newline
Bentzerdefinierte Men\"uf\"uhrung in einer \mcode{figure} (wird nicht behandelt)
\item \alert\alert{ uicontextmenu}\newline
Ein Pop-up Men\"u, das erscheint, wenn der Benutzer mit der rechten Maustaste
ein grafisches Objekt anklickt (wird nicht behandelt)
\end{itemize}
\end{frame}
%
% Slide
%
\begin{frame}[fragile]\frametitle{Arten von UiControls}
\vspace*{-0.3cm}
\begin{tabular}{|ll|}
\hline
'checkbox' & Wahl von Zust\"anden an/aus \\
'edit'     & Editierbare Texteingabe\\
'frame'    & graf. Gruppierung von Kontrollen\\
'popup'    & Auswahl aus Liste bei Anklicken\\ 
'listbox'  & Auswahl aus einer skrollbaren Liste \\
'pushbutton' & Starten eines Events \\
'radio'    & Auswahl einer Option \\
'toggle'   & Wahl des Zustandes: an/aus  \\
'slider'   & Auswahl aus Wertebereich\\
'text'     & Anzeige von Text in einer Box\\
\hline
\end{tabular}

\end{frame}
%
% Slide
%
\begin{frame}[fragile]\frametitle{Eigenschaften von UiControls}
\vspace*{-0.2cm}
{\small
\begin{tabular}{lp{0.85\linewidth}}
Style & Art des UiControls \\
CallBack & Durch Anklicken des Users auszuf\"uhrende Aktion\\
Position & Lage des Uicontrol-Objekts in der \mcode{figure}\newline
          {\tiny Eingabe: [links unten Breite H\"ohe]} \newline
          {\tiny Einheiten werden durch die \mcode{Units}Eigenschaft gesteuert.}\\
String   & Textdarstellung. Bei mehreren Optionen durch Eingabe \newline
          {\tiny \mcode{string={'opt1';'opt2'}} oder
            \mcode{string='opt1|opt2'}}\\
Units    & Einheit zur Bestimmung der Lages des UiControls,\newline
          {\tiny Werte: \mcode{pixels} (Default), \mcode{centimeters},
          \mcode{normalized} (Werte in $[0,1]$)}\\
Tag      & String zum Auffinden des Objekts durch \mcode{findobj}
\end{tabular}
}
\end{frame}
%
% Slide
%
\begin{frame}[fragile]\frametitle{Erzeugen von UiControls}
UiControl-Objekte k\"onnen durch \alert{ \mcode{uicontrol}} erzeugt
werden. Aufruf: 
\begin{lstlisting}
handle = uicontrol('eig1','wert1',...
                 'eign','wertn')
\end{lstlisting}
\alert{Beispiel:}
\begin{lstlisting}
>>u=uicontrol('style','listbox',...
'string','Option1|Option2|Option3',...
'units','centimeters',...
'position',[0 0 3 3])
\end{lstlisting} 
\end{frame}

%
% Slide
%
\begin{frame}[fragile]\frametitle{CallBack}
\begin{itemize}
\item Sobald ein GUI-Objekt aktiviert wird, f\"uhrt es MATLAB Code aus. Dies
  wird {\it CallBack} genannt.
\item {\it CallBack} ist eine Eigenschaft von UiControl.
\item Ein GUI-Objekt wird in der Regel durch einen Mausklick oder \"ahnliches
  aktiviert. 
\item CallBack kann eine Funktion oder ein String, der durch \mcode{eval}
  ausf\"uhrbar ist, sein. 
\item Syntax der Callback-Funktion
{\tiny \begin{verbatim}
function varargout = objectTag_Callback(h,ev_data,...
                       handles, varargin)
\end{verbatim}
%\begin{itemize}
%\item \mcode{objectTag} ist der Name, der im Tag des Objects gespeichert wird.
%\item \mcode{h} ist der Handle des Objekts, dass die CallBack-Funktion
%  aufruft. 
%\item \mcode{ev_data} wird nicht ben\"otigt.
%\item \mcode{handles} Struktur aller Objekte in der GUI
%\item \mcode{varargin} ist eine  Liste von Argumenten, die an die Funktion
%  \"ubergeben wird.  
%\end{itemize} 
} 
\end{itemize}
\end{frame}
% 
% Slide
% 
\begin{frame}[fragile]\frametitle{Hinweise}
\begin{itemize}
\item Callback-Funktionen werden immer im Haupt-Workspace
  gestartet. \"Ubergabe von Variablen schwierig.
\item Mit Hilfe der \mcode{figure}-Eigenschaft \mcode{UserData} k\"onnen Daten an
  das \mcode{figure}-Objekt geh\"angt werden.
\item Der User kann grafische Objekte per Hand schlie{\ss}en. Dies kann zu
  falschen Aufrufen f\"uhren.
\item Man achte darauf, dass man die richtigen Objekte anspricht. Tipp:
  Eigenschaft \mcode{Tag}. 
\end{itemize}
\end{frame}
% 
% Slide
%
\begin{frame}[fragile]\frametitle{Beispiel einer GUI}
\includegraphics[width=0.8\linewidth]{./figures/guiNeu.ps}
\end{frame}
% 
% Slide
%
\begin{frame}[fragile]\frametitle{bild\_funktion.m}
\begin{lstlisting}
function han = bild_funktion()
%------------------- Hauptprogram
x = linspace(-1,1,50);
y = linspace(-1,1,50);
t = 0:1:30;
A = erzDaten(x,y,t);
han = erzGUI(A);
end
%------------------- Grafische Oberflaeche erzeugen
function han = erzGUI(A);
delete(findobj('tag','figGUI'));
fig = figure('name','Beispiel GUI','UserData',A,'tag','figGUI');
han.pushbutton = uicontrol(fig,'Parent',fig,'Style','pushbutton',...
    'units','normalized','position',[0.8 0.2 0.15 0.15],...
    'String','Aktualisieren', 'Callback','darstGrafik');

han.grafikachse = axes('Position',[0.1 0.5 0.6 0.3],'tag','axesGUI');
han.grafik = surf(A(:,:,1));
\end{lstlisting}
\end{frame}
% 
% Slide
%
\begin{frame}[fragile]\frametitle{bild\_funktion.m}
\begin{lstlisting}han.frame1 = uicontrol(fig,'style','frame', 'units','normalized',...
    'position',[0.1 0.2 0.6 0.1]);
han.slider = uicontrol(fig,'style','slider','sliderstep',[0.2 0.2],...
    'min',0,'max',30, 'units','normalized','position', ...
     [0.1 0.2 0.6 0.05],'tag','slider','Callback','darstGrafik');
han.text1 = uicontrol(fig,'style','text', 'tag','text1',...
    'units','normalized','position', [0.3 0.25 0.1 0.03],...
     'String','Zeit t = 0');
\end{lstlisting}
\end{frame}
% 
% Slide
%
\begin{frame}[fragile]\frametitle{bild\_funktion.m}
\begin{lstlisting}
han.frame2 = uibuttongroup('units','normalized','tag','radio',...
    'position',[0.8 0.5 0.15 0.3]);
han.text2 = uicontrol(fig,'style','text','parent',han.frame2,...
    'units','normalized','position', [0.1 0.6 0.8 0.3],...
     'String','Darstellung');
han.radio1 = uicontrol(fig,'style','radio','parent',han.frame2,...
    'tag','r1', 'units','normalized',...
     'position', [0.1 0.45 0.8 0.15],'String','Surf');
han.radio2 = uicontrol(fig,'style','radio','parent',han.frame2,...
     'tag','r2','units','normalized',...
     'position', [0.1 0.25 0.8 0.15],'String','Mesh');
han.radio3 = uicontrol(fig,'style','radio','parent',han.frame2,...
      'tag','r3','units','normalized',...
      'position', [0.1 0.05 0.8 0.15],'String','Contour');
end
function A = erzDaten(x,y,t)
[X,Y,T] = meshgrid(x,y,t);
A = cos(pi*T.^0.5.*exp(-X.^2-Y.^2));
end
\end{lstlisting}
\end{frame}
% 
% Slide
%
\begin{frame}[fragile]\frametitle{darstGrafik.m}
\begin{lstlisting}
function darstGrafik()

A = get(findobj('tag','figGUI'),'UserData');

t = round(1+get(findobj('tag','slider'),'Value'));
set(findobj('tag','text1'), 'string', ...
    strcat('Zeit: t = ',num2str(t-1)) );
selection = findobj('tag','radio');
switch get(get(selection,'SelectedObject'),'tag')
    case 'r1'
        surf(A(:,:,t));
    case 'r2'
         mesh(A(:,:,t));
    case 'r3'
         contour(A(:,:,t));
    otherwise
        error('Keines oder zuviele entsprech. GUIs geoeffnet');
end;
\end{lstlisting}
\end{frame}
