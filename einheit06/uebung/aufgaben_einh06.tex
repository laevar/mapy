\documentclass[a4paper,10pt,DIV15]{scrartcl}
\usepackage[psamsfonts]{amssymb}
\usepackage{amsmath}
%\usepackage{latexsym}
\usepackage{theorem}


\usepackage{fontspec,xunicode,xltxtra}
\usepackage[ngerman]{babel}
\selectlanguage{ngerman}


\usepackage[svgnames,hyperref]{xcolor} %color definitions
%\usepackage{tikz}
%\usetikzlibrary{shadows}
%\usetikzlibrary{fit}
%\usetikzlibrary{shapes}
%\usetikzlibrary{backgrounds}


\usepackage{mcode}
%\usepackage{pstricks,pst-node,pst-text,pst-3d}
\parindent0cm % Abs�tze nicht einr�cken 

%---- neue Umgebung f�r Aufgaben
\theoremstyle{break}
\theoremheaderfont{\Large \bf}
\theorembodyfont{\normalfont}


% Definieren einer neuen Farbe
\definecolor{light-gray}{gray}{.9}

\newcounter{zaehler}     % neuen Z�hler einf�hren
\stepcounter{zaehler}    % Z�hler einen hochz�hlen

\newenvironment{aufg}%
%---- Header
{\begin{samepage}
\colorbox{light-gray}{                         % Box in gray
 \makebox[\textwidth]{                           % Box in linewidth
\textbf{Aufgabe} \arabic{zaehler} :}}\\[0.1cm]       % Header
%\begin{minipage}{0.5cm} \end{minipage}    % Insert 0.5cm
\begin{minipage}{\textwidth}}
%-----  foot
{\end{minipage} \nopagebreak %\begin{minipage}{1cm} \end{minipage}
\\[0.1cm] 
%\begin{minipage}{0.1cm} \end{minipage} 
%\hrulefill \begin{minipage}{1cm} \end{minipage}\\[1cm]  
\stepcounter{zaehler}                           % increase counter
 \end{samepage}%
}

%-------------------------------------------------------------------------------
\begin{document}
%-------------------------------------------------------------------------------

%--------------------------------------------------- Header
\begin{center}
\textbf{\LARGE Einf\"uhrung in MATLAB }\\
\end{center}
\begin{minipage}{6cm}
Dr. J. Schulz\\
\end{minipage}\hfill
\begin{minipage}{2cm}
\textbf{Einheit 6}\\
%30.07.2007
\end{minipage}\\[0.3cm]

%aufgaben per include einfuegen und ihnen namen geben?

%-----------------------------------------------------------------------------------
\begin{aufg}[0]
Lösen Sie  die Dgl. $\frac{d}{dt} y(t) = f(y)$ mit
  \[ f(y_1,y_2)=(y_1 -y_1 y_2, -y_2+y_2 y_1)\]
 und Anfangswert
  $y(0)=(0.5,0.5)$. Plotten Sie die Lösung im $y_1y_2$-Diagramm.
\end{aufg}

%-----------------------------------------------------------------------------------
\begin{aufg}[0]
Betrachten Sie das folgende System gew\"ohnlicher Differentialgleichungen
\begin{eqnarray*}
y_1'(t) & = & - 0.04 y_1(t) + 10^4 y_2(t)y_3(t) \\
y_2'(t) & = & 0.04 y_1(t) - 10^4 y_2(t) y_3(t) - 3 \cdot 10^7 y_2(t)^2\\
y_3'(t) & = & 3 \cdot 10^7 y_2(t)^2.
\end{eqnarray*}
L\"osen Sie  das System zum Anfangswert $(1,0,0)$ an $0$ auf dem Zeitintervall $0 \leq
t \leq 3$. Nutzen Sie die Solver \mcode{ode45} und \mcode{ode15s} und vergleichen Sie die Ergebnisse.
\end{aufg}

\begin{aufg}[0]
Führen Sie einen Vergleich zwischen einer numerischen und analytischen Differentation anhand der Funktion  $f(x)=\sin(\beta x),\, \beta,x \in \mathbb{R}$ durch.
Fügen Sie  den Werten der Funktion einen normalverteilten Fehler hinzu (Hinweis: \mcode{randn}) und wählen sie verschiedene Schrittweiten $h$.
Stellen Sie beide Ableitungen grafisch dar. Denken sie über die Gründe der Unterschiede nach.
\end{aufg}


\begin{aufg}[0]
Betrachten sie das eindimensionale Integral
\[
\int_0^1 \phi(x)e^{-\abs{x-y}}\sin(\abs{x-y})^2 dx\,,\, \in \mathbb{R}
\]
für $x,y \in [0,1]$ mit der Dichte $\phi(x) \in \mathbb{R}$. Schreiben Sie das Integral für diskrete Werte um in die Gleichung 
\[
 A \phi = f
\]
mit $f(x)=\sin(x),  x \in [0,1]$ und entsprechend konstruierter Matrix $A$.
%Weiterhin kann der Einfachheit halber angenommen werden, dass $x_i = y_i$ für alle Komponenten $i$ der Koordinatenvektoren.
\begin{itemize}
\item Lösen Sie das Gleichungssystem nach $\phi$.
\item Machen Sie einen Konsistenz-Check (ist die rechte Seite rekonstruierbar?)
\item Werten Sie das Integral für $y \in [2,4]$ aus.
\item Plotten Sie die Werte.
\end{itemize}
%\textit{Hinweis:} für $A$ steigt $x$ in den Spalten, und $y$ in den Zeilen.
\end{aufg}
\newpage
\begin{aufg}[0]
Schauen Sie sich die Funktion \mcode{interpolation.m} aus den M-Files von der Vorlesung Einheit 04 an und nutzen sie den Profiler mit der Funktion $f(x) = x^2$.

\begin{itemize}
 \item Finden Sie heraus  wo die Geschwindigkeit der Funktion (und der dort aufgerufenen, selbstgeschriebenen Funktionen) verbesserbar wäre. 
\item Vergleichen Sie die dort genutzte Funktion \mcode{vandermonde} mit \mcode{vandermonde2} 
(welche for-Schleifen nutzt). 
\item Vergleichen Sie ebenso unsere selbstgebaute Polynom-Interpolation mit der von Matlab.
\item Versuchen Sie die Gesamtlaufzeit der Funktion bei gleicher Funktionalität möglichst stark zu reduzieren.
\end{itemize}

\end{aufg}


\begin{aufg}[0]
Vergleichen Sie die Fixpunktiteration mit dem Newton-Verfahren f\"ur das Beispiel des Newton-Verfahrens aus der Vorlesung. Plotten sie beide Fehler in einer Grafik. Schauen 
Sie sich dann verschiedene Startwerte und die Geschwindigkeit beider Algorithmen an.
\end{aufg}

\end{document}