\begin{aufg}[0]
Schauen Sie sich die Funktion \mcode{interpolation.m} aus den M-Files von der Vorlesung Einheit 04 an und nutzen sie den Profiler mit der Funktion $f(x) = x^2$.

\begin{itemize}
 \item Finden Sie heraus  wo die Geschwindigkeit der Funktion (und der dort aufgerufenen, selbstgeschriebenen Funktionen) verbesserbar wäre. 
\item Vergleichen Sie die dort genutzte Funktion \mcode{vandermonde} mit \mcode{vandermonde2} 
(welche for-Schleifen nutzt). 
\item Vergleichen Sie ebenso unsere selbstgebaute Polynom-Interpolation mit der von Matlab.
\item Versuchen Sie die Gesamtlaufzeit der Funktion bei gleicher Funktionalität möglichst stark zu reduzieren.
\end{itemize}

\end{aufg}