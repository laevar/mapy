\documentclass[a4paper,10pt,DIV15]{scrartcl}
\usepackage[psamsfonts]{amssymb}
\usepackage{amsmath}
%\usepackage{latexsym}
\usepackage{theorem}


\usepackage{fontspec,xunicode,xltxtra}
\usepackage[ngerman]{babel}
\selectlanguage{ngerman}


\usepackage[svgnames,hyperref]{xcolor} %color definitions
%\usepackage{tikz}
%\usetikzlibrary{shadows}
%\usetikzlibrary{fit}
%\usetikzlibrary{shapes}
%\usetikzlibrary{backgrounds}


\usepackage{mcode}
%\usepackage{pstricks,pst-node,pst-text,pst-3d}
\parindent0cm % Abs�tze nicht einr�cken 

%---- neue Umgebung f�r Aufgaben
\theoremstyle{break}
\theoremheaderfont{\Large \bf}
\theorembodyfont{\normalfont}


% Definieren einer neuen Farbe
\definecolor{light-gray}{gray}{.9}

\newcounter{zaehler}     % neuen Z�hler einf�hren
\stepcounter{zaehler}    % Z�hler einen hochz�hlen

\newenvironment{aufg}%
%---- Header
{\begin{samepage}
\colorbox{light-gray}{                         % Box in gray
 \makebox[\textwidth]{                           % Box in linewidth
\textbf{Aufgabe} \arabic{zaehler} :}}\\[0.1cm]       % Header
%\begin{minipage}{0.5cm} \end{minipage}    % Insert 0.5cm
\begin{minipage}{\textwidth}}
%-----  foot
{\end{minipage} \nopagebreak %\begin{minipage}{1cm} \end{minipage}
\\[0.1cm] 
%\begin{minipage}{0.1cm} \end{minipage} 
%\hrulefill \begin{minipage}{1cm} \end{minipage}\\[1cm]  
\stepcounter{zaehler}                           % increase counter
 \end{samepage}%
}

%-------------------------------------------------------------------------------
\begin{document}
%-------------------------------------------------------------------------------

%--------------------------------------------------- Header
\begin{center}
\textbf{\LARGE Einf\"uhrung in MATLAB }\\
\end{center}
\begin{minipage}{6cm}
Dr. J. Schulz\\
\end{minipage}\hfill
\begin{minipage}{2cm}
\textbf{Einheit 6}\\
%30.07.2007
\end{minipage}\\[0.3cm]


%-----------------------------------------------------------------------------------
\begin{aufg}[0]
Plotten Sie die Sierpinski-Dreiecke zum Level $5$. Aus wievielen grafischen
Objekten besteht die Grafik? Entfernen Sie aus der Grafik alle Dreiecke, die einen Eckpunkt $(x,y)$
besitzen f\"ur den $x+y \geq 1/2$ gilt.
\end{aufg}

%-----------------------------------------------------------------------------------
\begin{aufg}[0]
Erzeugen Sie durch Kopieren grafischer Objekte $5$ Grafiken mit
Sierpinski-Dreiecken zum Level $5$, wobei Sie nur einmal das Skript
\mcode{sierpinski_plot} ausf\"uhren d\"urfen.
\end{aufg}

%-----------------------------------------------------------------------------------
\begin{aufg}[0]
Erstellen Sie aus der Funktion \mcode{integral.m} eine GUI. Die GUI sollte
Frames enthalten f\"ur die Funktion, die Darstellung der Funktion (mit
Balken), Intervallenden, Anzahl der Balken und das n\"aherungsweise
Ergebnis. 
\end{aufg}
%-----------------------------------------------------------------------------------
% \begin{aufg}
% Ein Graph besteht aus einer Eckenmenge und einer Menge von Kanten. In {MATLAB}
% soll nun die Menge der Eckpunkte  durch einen Cell-Array von Strings realisiert
% werden. Die Kanten sollen durch eine Struktur mit den Feldern \mcode{InEcke}
% und \mcode{OutEcke} realisiert sein. Die Kanten und Ecken sollen wiederum eine
% Struktur bilden.
% \begin{itemize}
% \item Schreiben Sie eine Funktion, die einen Graphen aus einer Datei
%   ausliest. Überlegen Sie sich eine sinnvolle Struktur der Datei. Legen Sie
%   eine Beispieldatei an.
% \item Schreiben Sie eine Funktion, die den Graphen wiederum in eine Datei
%   schreibt. 
% \item Schreiben Sie eine Funktion, die den Graphen grafisch darstellt. Ordnen
%   Sie die Ecken dazu regelm\"a{\ss}ig in einem Kreis an. Beschriften Sie die Ecken!
% \end{itemize} 
% \end{aufg}

%-----------------------------------------------------------------------------------
\begin{aufg}[0]
Entfernen Sie aus der Funktion \mcode{bild_funktion} den
  Pushbutton. Ersetzen Sie ihn durch eine Checkbox mit der man eine Legende
  f\"ur die Grafik ein- und ausschalten kann. 
\end{aufg}

%-----------------------------------------------------------------------------------
\begin{aufg}[0]
Erstellen Sie eine GUI, die die Funktion $f(x) = x^a \sin(1/x)$ auf
  $[0.01, 1]$ plottet. Der Benutzer soll $a$ modifizieren k\"onnen. Benutzen sie, falls möglich, 
das tool \mcode{guide}.
\end{aufg}
%-----------------------------------------------------------------------------------
\begin{aufg}[0]
Schreiben Sie eine Funktion, die zu einem gegebenen $N$ die Matrix 
\begin{eqnarray*} 
A & := & \frac{1}{h^2} tridiag(-I_{N-1}, T, -I_{N-1}) \in \mathbb{R}^{(N-1)^2
 \times (N-1)^2},\\
 T & := & tridiag(-1,4,-1) \in \mathbb{R}^{(N-1)\times (N-1)} 
\end{eqnarray*}
erzeugt. Hierbei gilt $h=1/N$.\\

{\it Bemerkung: } Sie d\"urfen nicht die Matrix aus der Gallery verwenden.
\end{aufg}
%-----------------------------------------------------------------------------------



\end{document}
