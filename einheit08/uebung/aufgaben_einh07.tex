\documentclass[a4paper,10pt,DIV15]{scrartcl}
\usepackage[psamsfonts]{amssymb}
\usepackage{amsmath}
%\usepackage{latexsym}
\usepackage{theorem}


\usepackage{fontspec,xunicode,xltxtra}
\usepackage[ngerman]{babel}
\selectlanguage{ngerman}


\usepackage[svgnames,hyperref]{xcolor} %color definitions
%\usepackage{tikz}
%\usetikzlibrary{shadows}
%\usetikzlibrary{fit}
%\usetikzlibrary{shapes}
%\usetikzlibrary{backgrounds}


\usepackage{mcode}
%\usepackage{pstricks,pst-node,pst-text,pst-3d}
\parindent0cm % Abs�tze nicht einr�cken 

%---- neue Umgebung f�r Aufgaben
\theoremstyle{break}
\theoremheaderfont{\Large \bf}
\theorembodyfont{\normalfont}


% Definieren einer neuen Farbe
\definecolor{light-gray}{gray}{.9}

\newcounter{zaehler}     % neuen Z�hler einf�hren
\stepcounter{zaehler}    % Z�hler einen hochz�hlen

\newenvironment{aufg}%
%---- Header
{\begin{samepage}
\colorbox{light-gray}{                         % Box in gray
 \makebox[\textwidth]{                           % Box in linewidth
\textbf{Aufgabe} \arabic{zaehler} :}}\\[0.1cm]       % Header
%\begin{minipage}{0.5cm} \end{minipage}    % Insert 0.5cm
\begin{minipage}{\textwidth}}
%-----  foot
{\end{minipage} \nopagebreak %\begin{minipage}{1cm} \end{minipage}
\\[0.1cm] 
%\begin{minipage}{0.1cm} \end{minipage} 
%\hrulefill \begin{minipage}{1cm} \end{minipage}\\[1cm]  
\stepcounter{zaehler}                           % increase counter
 \end{samepage}%
}

%-------------------------------------------------------------------------------
\begin{document}
%-------------------------------------------------------------------------------

%--------------------------------------------------- Header
\begin{center}
\textbf{\LARGE Einf\"uhrung in MATLAB }\\
\end{center}
\begin{minipage}{6cm}
Dr. J. Schulz\\
\end{minipage}\hfill
\begin{minipage}{2cm}
\textbf{Einheit 7}\\
%30.07.2007
\end{minipage}\\[0.3cm]



%-----------------------------------------------------------------------------------
\begin{aufg}
 Kompilieren Sie \mcode{ggt_2.c} und testen Sie die Funktion.
\end{aufg}


%-----------------------------------------------------------------------------------
\begin{aufg}
Schreiben Sie eine mex-Funktion, die zu einem gegebenen Vektor den
  Durchschnittswert und das Produkt aller Eintr\"age zur\"uckgibt.  
\end{aufg}


%-----------------------------------------------------------------------------------
\begin{aufg}
 Schreiben Sie eine Funktion in MATLAB, die die 
  mex-Routine \mcode{ggt} gegen falsche Aufrufe sch\"utzt. 
\end{aufg}


%-----------------------------------------------------------------------------------
\begin{aufg}
\begin{itemize}
\item Kompilieren Sie die Datei \mcode{mandel_c.c}. 
\item Vergleichen Sie die Laufzeiten der Algorithmen \mcode{mandelbrot.m} und
  \mcode{mandelbrot2.m} f\"ur verschiedene Anzahlen von Punkten in der
  komplexen Ebene. Was stellen Sie fest?
\end{itemize}
\end{aufg}

%-----------------------------------------------------------------------------------
\begin{aufg}
Modifizieren Sie das Programm \mcode{plot_poisson.c} derart, dass  die
Funktion 
\[ f(x,y) = (x-1)(x+1)(y-1)(y+1) \]
geplottet wird. Benutzen Sie dabei $60$ Punkte in $x$-Richtung und $30$ Punkte
in $y$-Richtung. 
\end{aufg}

%-----------------------------------------------------------------------------------
\begin{aufg}
Erstellen Sie ein C-Programm, das zu einem gegebenen $n \in \mathbb{N}$ in
MATLAB die Hilbertmatrix $H=(h_{ij}) \in \mathbb{R}^{n \times n}$, $h_{ij}=\frac{1}{i+j-1}$
berechnet. Danach soll in MATLAB das Gleichungssystem $H x = b$ mit $b=H (1, \dots,
1)^t$ gel\"ost werden. Die Lösung soll dann an das C-Programm zur\"uckgegeben
werden und dort ausgegeben werden. Testen Sie $n=4,5,8,10,20$!
\end{aufg}


\end{document}
