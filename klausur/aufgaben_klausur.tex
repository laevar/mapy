\documentclass[a4paper,12pt,DIV15]{scrartcl}
\usepackage{amssymb}
\usepackage{amsmath}
\usepackage[svgnames]{xcolor} %color definitions

\usepackage{fontspec,xunicode,xltxtra}
%\usepackage{fontspec,xunicode}
%\usepackage{polyglossia}
%\setdefaultlanguage[spelling=new, latesthyphen=true]{german}
%\setsansfont{DejaVu Sans}
%\setsansfont{Verdana}
%\setsansfont{Arial}
%\setromanfont[Mapping=tex-text]{Linux Libertine}
%\setsansfont[Mapping=tex-text]{Myriad Pro}
%\setmonofont[Mapping=tex-text]{Courier New}

%\setsansfont{Linux Biolinum}

\usepackage[ngerman]{babel}
\selectlanguage{ngerman}

%
% math/symbols
%
\usepackage{amssymb}
\usepackage{amsthm}
% \usepackage{latexsym}
\usepackage{amsmath}
%\usepackage{amsxtra} %Weitere Extrasymbole
%\usepackage{empheq} %Gleichungen hervorheben
%\usepackage{bm}
 %\bm{A} Boldface im Mathemodus

\usepackage{multimedia}
%\usepackage{tikz}

\usepackage{cellspace}
\setlength{\cellspacetoplimit}{2pt}
\setlength{\cellspacebottomlimit}{2pt}

%%%%%%%%%%%%%%%%%% Fuer Frames [fragile]-Option verwenden!
%Programm-Listing
%%%%%%%%%%%%%%%%%%
%Listingsumgebung fuer verbatim
%Grauhinterlegeter Text
%Automatischer Zeilenumbruch ist aktiviert
%\usepackage{listings}
\usepackage[framed]{mcode}
%\usepackage{mcode}
% This command allows you to typeset syntax highlighted Matlab
% code ``inline''.
\newcommand{\isage}[1]{\lstinline|#1|}

\definecolor{lgray}{gray}{0.80}
\definecolor{gray}{gray}{0.3}
\definecolor{darkgreen}{rgb}{0,0.4,0}
\definecolor{darkblue}{rgb}{0,0,0.8}
\definecolor{key}{rgb}{0,0.5,0} 
%\lstset{backgroundcolor=\color{lgray}, frame=single, basicstyle=\ttfamily, breaklines=true}
\lstnewenvironment{matlab}[1][]{\lstset{xleftmargin=0.2cm,frame=none,backgroundcolor=\color{white},basicstyle=\color{darkblue}\ttfamily\small,#1}}{} 
\lstnewenvironment{matlabin}[1][]{\lstset{#1}}{} 
%\lstnewenvironment{sage}{\lstset{,language=python, keywordstyle=color{blue},    commentstyle=color{green}, emphstyle=\color{red}, %frame=single, stringstyle=\color{red}, basicstyle=\ttfamily, ,mathescape =true,escapechar=§}}{}

\lstset{
%language=python,
backgroundcolor=\color[gray]{0.9},
breaklines=true,
basicstyle=\ttfamily\small,
%otherkeywords={ =},
%keywordstyle=\color{blue},
%stringstyle=\color{darkgreen},
showstringspaces=false,
%emph={for, while, if, elif, else, not, and, or, printf, break, continue, return, end, function},
%emphstyle=\color{blue},
%emph={[2]True, False, None, self, NaN, NULL},
%emphstyle=[2]\color{key},
%emph={[3]from, import, as},
%emphstyle=[3]\color{blue},
%upquote=true,
%morecomment=[s]{"""}{"""},
%commentstyle=\color{gray}\slshape,
%framexleftmargin=1mm, framextopmargin=1mm, 
frame=single,
%mathescape =true,
%escapechar=§
}


\usepackage{mydef}
%\usepackage{cmap} % you can search in the pdf for umlauts and ligatures
\usepackage{colonequals} %corrects the definition-symbols \colonequals (besides others)

\usepackage{ifthen}

%%%%%%%%%%%%%%%%%%%
%Neue Definitionen
%%%%%%%%%%%%%%%%%%%

%Newcommands
\newcommand{\Fun}[1]{\mathcal{#1}}      %Mathcal fuer Funktoren
\newcommand{\field}[1]{\mathbb{#1}}     %Grundkoerper ?? in mathds

\newcommand{\A}{\field{A}}              %Affines A
\newcommand{\Fp}{\field{F}_{\!p}}       %Endlicher Koerper mit p Elementen
\newcommand{\Fq}{\field{F}_{\!q}}       %Endlicher Koerper mit q Elementen
\newcommand{\Ga}{\field{G}_{a}}         %Add Gruppenschema
\newcommand{\K}{\field{K}}              %Generischer Koerper 
\newcommand{\N}{\field{N}}              %Nat Zahlen
\newcommand{\Pj}{\field{P}}             %Projektives P
\newcommand{\R}{\field{R}} 		%Reelle Zahlen
\newcommand{\Q}{\field{Q}}              %Rationale Zahlen  
\newcommand{\Qt}{\field{H}}             %Quaternionen 
\newcommand{\V}{\field{V}}              %Vektorbuendel V
\newcommand{\Z}{\field{Z}}              %Ganze Zahlen
\DeclareMathOperator{\Real}{Re}

\newcommand{\fdg}{\;|\;}                 %fuer die gilt

%Operatoren
\DeclareMathOperator{\Abb}{Abb}
%\usepackage{sagetex}


%
% Aufgaben
%
\parindent0cm % Abs�tze nicht einr�cken 
% Definieren einer neuen Farbe
\definecolor{light-gray}{gray}{.9}

\newcounter{zaehler}     % neuen Z�hler einf�hren

\newenvironment{aufg}[1][0]
%---- Header
{\begin{samepage}%
\refstepcounter{zaehler}% increase counter
%\colorbox{light-gray}{%                         % Box in gray
% \makebox[\textwidth]{%                           % Box in linewidth
%\textbf{Aufgabe \arabic{zaehler} } }\hspace{-\textwidth}\makebox[\textwidth]{\hfill #1 Punkte} }\\[0.05cm]       % Header
\dotfill\\
{\large\textbf{Aufgabe \arabic{zaehler} }\ifthenelse{-1=#1}{(testierbar)}{}\ifthenelse{0=#1 \or -1=#1}{}{\hfill #1 Punkte} }\\[0.4cm]
\begin{minipage}{\textwidth}%
}%
%-----  foot
{\end{minipage}\nopagebreak%\begin{minipage}{1cm} \end{minipage}
%\\ 
%\begin{minipage}{0.1cm} \end{minipage} 
%\hrulefill \begin{minipage}{1cm} \end{minipage}\\[1cm]  
\end{samepage}%
\\%
\bigskip%
}


%\usepackage{tikz}
%\usetikzlibrary{shadows}
%\usetikzlibrary{fit}
%\usetikzlibrary{shapes}
%\usetikzlibrary{backgrounds}

\parindent0cm % Abs�tze nicht einr�cken 

% Definieren einer neuen Farbe
\definecolor{light-gray}{gray}{.9}

\newcounter{zaehler}     % neuen Z�hler einf�hren
\stepcounter{zaehler}    % Z�hler einen hochz�hlen

\newenvironment{aufg}[1]
%---- Header
{\begin{samepage}%
%\colorbox{light-gray}{%                         % Box in gray
% \makebox[\textwidth]{%                           % Box in linewidth
%\textbf{Aufgabe \arabic{zaehler} } }\hspace{-\textwidth}\makebox[\textwidth]{\hfill #1 Punkte} }\\[0.05cm]       % Header
{\large\textbf{Aufgabe \arabic{zaehler} }\hfill #1 Punkte}\\[0.4cm]
\begin{minipage}{\textwidth}
}
%-----  foot
{\end{minipage} \nopagebreak %\begin{minipage}{1cm} \end{minipage}
%\\ 
%\begin{minipage}{0.1cm} \end{minipage} 
%\hrulefill \begin{minipage}{1cm} \end{minipage}\\[1cm]  
\stepcounter{zaehler}                           % increase counter
\end{samepage}%
\bigskip
}

%-------------------------------------------------------------------------------
\begin{document}
%-------------------------------------------------------------------------------

%--------------------------------------------------- Header
\begin{center}
\textbf{\LARGE Mathematische Anwendersysteme }\\
\textbf{\LARGE Einführung in MATLAB}\\\medskip
\end{center}
\begin{minipage}{6cm}
Jochen Schulz
\end{minipage}\hfill
\begin{minipage}{4cm}
%\textbf{Klausur}\\
24.09.2010
\end{minipage}\\[1cm]
%------------------------------

\begin{center}
\Huge \textbf{Klausur}
\end{center}
\bigskip\bigskip\bigskip
\Large
\begin{center}
\begin{tabular}{|Sl||p{0.5cm}|p{0.5cm}|p{0.5cm}|p{0.5cm}|p{0.5cm}|p{0.5cm}|p{0.5cm}||Sc|}
\hline
Aufgabe & \textbf{1} & \textbf{2} & \textbf{3} & \textbf{4} & \textbf{5} & \textbf{6} & \textbf{7} & \textbf{Summe}\\
\hline
Mögl. Pkt. &  4  & 2  & 2  & 3  & 5  & 8  & 5  &  29  \\
\hline
Erreichte Pkt. &    &   &   &   &   &   &     &    \\
\hline
\end{tabular}
\end{center}

\bigskip\bigskip\bigskip
Bitte eintragen:\\
\begin{center}
\begin{tabular}{|Sl|p{8cm}|}
\hline
Nachname: & \\
\hline
Vorname: & \\
\hline
Studiengang: & \\
\hline 
Semester: & \\
\hline 
Immatrikulationsnummer: & \\
\hline
\end{tabular}\\[1cm]
\textbf{Hinweise:}
\begin{itemize}
\item Die Klausur beginnt um 10.00 Uhr und endet um 11.30 Uhr.
\item Ben\"otigte Hilfsmittel sind Stift und Papier.
\item Erlaubte Hilfsmittel sind gedruckte sowie handgeschriebene Notizen oder Skripte. 
%\item Benutzen Sie zum Aufschreiben der Aufgaben Sage-Syntax.
\end{itemize}
\end{center}

\newpage
\normalsize

\lstset{basicstyle={\lstbasicfont}}
%-----------------------------------------------------------------------------------
%-----------------------------------------------------------------------------------
\begin{aufg}{3}
Was ist ein Cell-Array? Worin unterscheidet sich ein Cell-Array von einem
normalen Array? Geben sie einen typischen Fall an, in dem mit Cell-Arrays
gearbeitet wird!
\end{aufg}

%-----------------------------------------------------------------------------------
%-----------------------------------------------------------------------------------
\begin{aufg}{3}
Schreiben sie eine anonyme Funktion, die 
\[ f(s) = \int_0^1 e^{-s t} \sin( t^2) dt \]
approximiert. 

{\it Hinweis: } Numerische Integration von $\int_a^b f(x)dx$ erfolgt durch \mcode{quad(f,a,b)}. 
\end{aufg}


%-----------------------------------------------------------------------------------
% das muss komplizierter werden!
%-----------------------------------------------------------------------------------
% \begin{aufg}{3}
% Wie erzeugen sie ohne Schleifen in \textsc{Matlab} aus einem Vektor
%  $x=(x_1, \dots ,x_n)$ die Vandermonde-Matrix
% { \[ V:= \left(\begin{array}{ccccc} 
% 1 & x_1 & x_1^2 & \hdots & x_1^{n-1}\\
% 1 & x_2 & x_2^2 & \hdots & x_2^{n-1}\\
% \vdots & \vdots & \vdots & \vdots & \vdots\\
% 1 & x_n & x_n^2 & \hdots & x_n^{n-1}\\
% \end{array} \right)?  \]}
% \end{aufg}

%-----------------------------------------------------------------------------------
%-----------------------------------------------------------------------------------
% \begin{aufg}{4}
% Betrachten sie die folgenden Eingaben:
% \begin{lstlisting}
% A = reshape(1:4,2,2);
% B = [ 8 9; 10 11];
% C = cat(3,A,B);
% D = C([1 3 5]);  
% \end{lstlisting}
% Welche Werte haben \mcode{A}, \mcode{B}, \mcode{C} und \mcode{D}?
% \end{aufg}

%-----------------------------------------------------------------------------------
%-----------------------------------------------------------------------------------
%-----------------------------------------------------------------------------------
\begin{aufg}{5}
Schreiben sie eine Funktion mit Input-Variablen $x_0$ und $TOL$, die die Folge
\[ x_{n+1} = x_n - \frac{x_n^2-5}{2 x_n}, \quad n \in \mathbb{N} \]
berechnet und abbricht, wenn $|x_{n}-x_{n-1}| \leq TOL$ ist. Die Funktion soll
$x_{n}$ und das zugeh\"orige $n$ zur\"uckgeben. 
\end{aufg}
%\newpage
%-----------------------------------------------------------------------------------
%-----------------------------------------------------------------------------------
% \begin{aufg}{6}
% Welche Werte besitzen $x1$, $x2$, $x3$  am Ende der jeweiligen Eingaben?
% \begin{itemize}
% \item [(a)]
% \begin{lstlisting}
% a = [ 1 2 3 4];
% b = [ 0 3 0 4];
% x1 = (a-b > 1) | (b ~= 0)
% \end{lstlisting}
% 
% \item [(b)] 
% \begin{lstlisting}
% A = diag (1:10);
% A(1:8,:) = [];
% x2 = size(A(:,2:4));
% \end{lstlisting}
% 
% \item [(c)]
% \begin{lstlisting}
% >> x = [Inf NaN];
% >> y = [Inf NaN];
% >> x3 = ( x == y );
% \end{lstlisting}
% \end{itemize} 
% \end{aufg}
%-----------------------------------------------------------------------------------
%-----------------------------------------------------------------------------------
%-----------------------------------------------------------------------------------

% \begin{aufg}{4}
% Die folgenden Befehlszeilen sind fehlerhaft. Erkl\"aren sie jeweils den Fehler!
% \begin{itemize}
% \item [(a)]
% \begin{lstlisting}
% >> f = 'x.^2+1';
% >> g = [1 1 1];
% >> h = g(f(1));
% \end{lstlisting}
% 
% \item [(b)]
% \begin{lstlisting}
% >> a = 1 + 0:4;
% >> b(6) = 5;
% >> plot(a,b);
% \end{lstlisting}
% \end{itemize} 
% 
% \end{aufg}

%-----------------------------------------------------------------------------------
%-----------------------------------------------------------------------------------
%-----------------------------------------------------------------------------------
\begin{aufg}{6}
Erkl\"aren sie die Funktionsweise von \mcode{meshgrid} am Beispiel eines Plots
der Funktion 
\[ f(x,y,z) = \sin(2x) \cos(3y) \sin(z), \quad (x,y,z) \in [0,1]^3 \]
mit Hilfe des Kommandos \mcode{slice}.
\end{aufg}


%\newpage
\begin{aufg}{10 (rausnehmen?)}
%GUI
Schreiben sie ein Programm, das eine Matlab-GUI erzeugt, welches folgende Elemente 
enthalten soll:
\begin{itemize}
 \item 2 Axen-Elemente und
 \item ein editierbares Text-Feld. 
\end{itemize}
Die GUI soll $f(x,y) = sin(x)cos(y)$ verarbeiten. Die Datenpunkte für $x$ und $y$ sollen 
durch zufällig verteilte Punkte erzeugt werden (Hinweis: \mcode{rand}) und die Funktionswerte sollen in der einen
\mcode{axes} mit \mcode{plot3} ausgegeben werden. In der anderen \mcode{axes} soll die Funktion per \mcode{surf} dargestellt 
werden. Dafür müssen die Datenpunkte auf einem regelmässigen Gitter interpoliert werden. In dem Textfeld soll die
Interpolationsmethode gewählt werden können (\mcode{'cubic'} oder \mcode{'linear'})
(Dies ist auch das einzige Element mit einem Callback).
%Welche Hierarchie besitzen die grafischen Elemente von Matlab beginnend vom root-Element ?
%Wie werden Grafik-Handles abgespeichert und wie bekommt man Informationen über die handles ?
%\item [$\square$] Die Callback-Funktion einer GUI benutzt den globalen Workspace. 

\emph{Hinweis:} Die \mcode{positions} können aus Gründen der Vereinfachung weggelassen werden.

\emph{Tip:} für \mcode{axes} muss ein Vater angegeben werden.
\end{aufg}

%-----------------------------------------------------------------------------------
\begin{aufg}{10}
Schreiben sie eine Funktion, die einen bis mehrere zu übergebene Strings in inline-functions 
konvertiert und vektorisiert. Die Funktion soll ansonsten noch die Anzahl der Evaluationspunkte, die untere Schranke und die obere Schranke übergeben bekommen können (in dieser Reihenfolge). Setzen sie für alle Variablen 
Default-Werte, falls diese nicht übergeben werden. Danach sollen alle Funktionen in einem 2D-Plot grafisch dargestellt werden.
\end{aufg}

%-----------------------------------------------------------------------------------
%\begin{aufg}{6}
%Woraus bestehen die beiden Möglichkeiten C-Programme mit Matlab zu kombinieren ? 
%Erklären sie möglichst genau die Grundstruktur beider Varianten in dem sie jeweils ein Grundgerüst
%aufschreiben und Kommentieren: Welche Funktionen werden gebraucht, was muss wie übersetzt werden etc.
%\end{aufg}

%-----------------------------------------------------------------------------------
%\begin{aufg}{4}
%Schreiben sie ein Skript-file welches eine Grafik erstellt, die mehrere unterschiedlich 
%gezeichnete Kurven und eine Legende besitzt. Ändern sie nachträglich mittels des handles
%den Titel und die Achsen-Beschiftung. 
%\end{aufg}

%\begin{aufg}{4}
%LGS loesen ? cond ? norm ? (gleich sparse mit einbauen ?)
%LGS von einer 3dim-funktion loesen. (entsprechendes umsortieren, repmat, plot)
%\end{aufg}


\begin{aufg}{6}
Gegeben sei eine Datei \mcode{'dummy.csv'} mit folgendem Inhalt:

\begin{lstlisting}
0,5,3,7
76,23,1,8
34,176,84,0
\end{lstlisting}
Schreiben sie ein Programm, welches diese Datei einliest und die jeweiligen 
Werte in einer Matrix abspeichert. Das Programm soll so flexibel sein, dass es auch
Dateien einlesen kann, welche eine unterschiedliche Anzahl von kommaseparierten 
Zeilen und Spalten besitzen (vorausgesetzt in jeder Zeile sind stets genauso viele Einträge).

\emph{Hinweis:} \mcode{csvread} darf nicht benutzt werden und es brauchen keine Überprüfungen gemacht werden.
\end{aufg}


%-----------------------------------------------------------------------------------
%\item [$\square$] \mcode{fopen} erlaubt nicht das parallele Bearbeiten mehrerer Dateien.
%\item [$\square$] Skripte operieren auf den Daten des globalen Workspace. 
%\item [$\square$] Die Eintr\"age von \mcode{Logical Arrays} bestehen nur aus $0$ und $1$.  
%\item [$\square$] Werden durch \mcode{load} Variablen in den Workspace geladen, so werden
%  bereits existierende Variablen mit dem gleichen Namen \"uberschrieben.

% \item [$\square$] Die Multiplikation eines \mcode{single}-Datentyps mit einem
%   \mcode{double}-Datentyp ergibt einen \mcode{double}-Datentyp.
% \item [$\square$] Ein String ist ein Vektor von Eintr\"agen des Datentyps \mcode{char}.
% \item [$\square$] \mcode{sparse}-Matrizen sind besonders gut geeignet zur Speicherung von
%   Matrizen bei denen fast alle Eintr\"age $0$ sind.  
% \item [$\square$] Im Datentyp \mcode{int8} k\"onnen Zahlen zwischen $0$ und $2^8-1$
%   dargestellt werden.

\end{document}
