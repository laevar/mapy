
\subsection{Erste Schritte}
%-------------------------------------------------
%  Folie:
%-------------------------------------------------
\begin{frame}[fragile]\frametitle{Erste Schritte}
\begin{itemize}
\item MATLAB als Taschenrechner \newline (Ergebnis wird in \mcode{ans} gespeichert.) 
\begin{lstlisting}
>> 1+(sin(pi/2)+ exp(2))*0.5
ans = 5.1945
\end{lstlisting}
\item Eingabe von (Zeilen-)Vektoren
\begin{lstlisting}
>> x = [1 2 3]  
\end{lstlisting}
\item Transponieren und speichern in  Variable \mcode{b}
\begin{lstlisting}
>> b = transpose(x)
\end{lstlisting}
\end{itemize}
\end{frame}

%-------------------------------------------------
%  Folie:
%-------------------------------------------------
\begin{frame}[fragile]\frametitle{Erste Schritte II}
\begin{itemize}
\item Erzeugen einer Matrix
\begin{lstlisting}
A = [0 2 3 ; 4 5 6; 7 8 9];
\end{lstlisting}
\item Lösen des Gleichungssystems $A \cdot z=b$
\begin{lstlisting}
z = A \ b
\end{lstlisting}
\item Probe 
\begin{lstlisting}
A*z
\end{lstlisting}
\end{itemize}
\end{frame}
%-------------------------------------------------
%  Folie:
%-------------------------------------------------
\begin{frame}[fragile]\frametitle{Erste Schritte III}
\begin{itemize}
\item Berechnen der Determinante von $A$
\begin{lstlisting}
det(A)
\end{lstlisting}
\item Hilfe zu \mcode{det} 
\begin{lstlisting}
help det 
\end{lstlisting}
\begin{matlab}
DET    Determinant.
  DET(X) is the determinant of the square matrix X.
  Use COND instead of DET to test for matrix singularity.
\end{matlab}
\item Erzeugen einer Einheitsmatrix
\begin{lstlisting}
B = eye(3,3)
\end{lstlisting}
\end{itemize}
\end{frame}
%-------------------------------------------------
%  Folie:
%-------------------------------------------------
\begin{frame}[fragile]\frametitle{Erste Schritte IV}
\begin{itemize}
\item Matrizenoperationen
\begin{lstlisting}
A+B, A-B, A*B, inv(A)
\end{lstlisting}
\item Anwendung von Vektoren
\begin{lstlisting}
y = sqrt(x)
\end{lstlisting}
\begin{matlab}
y =
    1.0000    1.4142    1.7321 
\end{matlab}

\end{itemize}
\end{frame}
%-------------------------------------------------
%  Folie:
%-------------------------------------------------
\begin{frame}[fragile]\frametitle{Erste Schritte V}
\begin{itemize}
\item Komponentenweise Multiplikation
\begin{lstlisting}
y = x.*x
\end{lstlisting}
\begin{matlab}
 y =
     1     4     9 
\end{matlab}

\item Zeilenvektor mit Werten von $1$ bis $100$
\begin{lstlisting}
a = [1:100];
\end{lstlisting}
\item Berechne $\sum_{j=1}^{100} \frac{1}{j^2}$
\begin{lstlisting}
(1./a)*transpose(1./a)
\end{lstlisting}
\begin{matlab}
ans = 1.6350 
\end{matlab}
\end{itemize}
\end{frame}
\subsection{Etwas komplexeres Beispiel}
%-------------------------------------------------
%  Folie:
%-------------------------------------------------
\begin{frame}[fragile]\frametitle{Die Mandelbrot-Menge}
\begin{columns}[c,onlytextwidth]
\column{0.6\textwidth}
\pgfimage[width=\textwidth]{../figures/mandel}
\column{0.4\textwidth}
Die Mandelbrot-Menge ist die Menge von Punkten $c \in \mathbb{C}$
bei denen die Folge $(z_n)_n$, die durch
\begin{align*} 
z_0&:=c\\
z_{n+1} &= z_n^2 +c, \quad n \in \mathbb{N} 
\end{align*}
definiert ist, beschränkt ist.
\end{columns}
\end{frame}
%-------------------------------------------------
%  Folie:
%-------------------------------------------------
\begin{frame}[fragile]\frametitle{Die Mandelbrot-Menge}
\lstinputlisting{mandel.m}
\end{frame}
%-------------------------------------------------
%  Folie:
%-------------------------------------------------
\begin{frame}[fragile]\frametitle{Verwendete Befehle}
\begin{itemize}
\item \mcode{linspace(a,b,n)}\\ ist ein Vektor mit n Einträgen der Form
$a, a+(b-a)/(n-1), \dots ,b$
\item \mcode{[X,Y] = meshgrid(x,y)}\\ erzeugt Matrizen 
\begin{equation*}
X = \left( \begin{array}{ccc} x_1 & \ldots & x_n\\  & \vdots & \\x_1 & \ldots & x_n\end{array}
\right), \quad  Y = \left( \begin{array}{ccc} y_1 & \ldots & y_1\\  & \vdots & \\y_n & \ldots & y_n\end{array}  \right)
\end{equation*}
\item \mcode{C = complex(X,Y)}\\ erzeugt $C=(C(j,k))_{jk}$ mit
$C(j,k)=X(j,k)+i \ Y(j,k)$ 
\end{itemize}
\end{frame}
%-------------------------------------------------
%  Folie:
%-------------------------------------------------
\begin{frame}[fragile]\frametitle{Verwendete Befehle}
\begin{itemize}
\item \mcode{B = isfinite(A)}\\ Matrix $B$ hat gleiche Gr\"o{\ss}e wie $A$. Die Einträge sind $1$, 
wenn der entsprechende Eintrag von $B$ finit ist und $0$ sonst.  
\item \mcode{image(x,y,A)}\\ erzeugt eine Grafik auf der Basis des Gitters
  $(x,y)$ mit Werten $A$. Durch den entsprechenden Eintrag von $A$ wird die
  Farbe bestimmt.  
\item \mcode{title}\\ Überschrift der Grafik.
\item \mcode{for}, \mcode{end}\\ 
Schleife (Details später).
\end{itemize}
\end{frame}
