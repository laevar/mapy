\documentclass[utf8x,notes=hide]{beamer}

\mode<article>
{
  \usepackage{fullpage}
  \usepackage{pgf}
  \usepackage{hyperref}
  \setjobnamebeamerversion{beamer}
}

\mode<presentation>
{
  %\usetheme{Frankfurt}
 %\usetheme{My}
  \usetheme{Madrid}
  % or ...
%\usecolortheme{seagull}
  %\setbeamercovered{transparent}
  %\setbeamercovered{dynamic}
  % or whatever (possibly just delete it)
}
\usenavigationsymbolstemplate{}
\usefonttheme{structurebold}
\usepackage{multimedia}
\usepackage{tikz}
\usepackage{fontspec,xunicode,xltxtra}
%\usepackage[scaled=.90]{helvet}
% Or whatever. Note that the encoding and the font should match. If T1
% does not look nice, try deleting the line with the fontenc.

\setbeamertemplate{footline}
{
\leavevmode
%\hbox{\begin{beamercolorbox}[wd=.5\paperwidth,ht=2.5ex,dp=1.125ex,
%leftskip=.3cm plus1fill,rightskip=.3cm]{author in head/foot}%
%    \usebeamerfont{author in head/foot}\insertshortauthor
%  \end{beamercolorbox}%
%  \begin{beamercolorbox}[wd=.5\paperwidth,ht=2.5ex,dp=1.125ex,leftskip=.3cm,
%rightskip=.3cm plus1fil]{title in head/foot}%
%    \usebeamerfont{title in head/foot}\insertshorttitle\hfill

\hfill\insertframenumber  \hspace{3pt}

%\inserttotalframenumber
%\hspace*{2ex}
%  \end{beamercolorbox}}%
  \vskip3pt%
}

%\usepackage[english]{babel}
\usepackage[ngerman]{babel}
\selectlanguage{ngerman}

%
% math/symbols
%
\usepackage{amssymb}
\usepackage{amsthm}
% \usepackage{latexsym}
\usepackage{amsmath}
%\usepackage{listings}
\usepackage[framed]{mcode}
%\usepackage{mcode}

\usepackage{mydef}
\usepackage{cmap} % you can search in the pdf for umlauts and ligatures
%\usepackage{colonequals} %corrects the definition-symbols \colonequals (besides others)
\title{Einführung in Matlab}
%
%\subtitle{Disputation} % (optional)

\author{Jochen Schulz}
% - Use the \inst{?} command only if the authors have different
%   affiliation.

\institute{Georg-August Universit\"at G\"ottingen \pgfimage[height=0.5cm]{figures/unilogo3}}
% - Use the \inst command only if there are several affiliations.
% - Keep it simple, no one is interested in your street address.

\date{\today}

\subject{Einführung in Matlab}
% This is only inserted into the PDF information catalog. Can be left
% out. 



% If you have a file called "university-logo-filename.xxx", where xxx
% is a graphic format that can be processed by latex or pdflatex,
% resp., then you can add a logo as follows:

%\logo{\pgfimage[height=0.5cm]{figures/unilogo3}}


% Delete this, if you do not want the table of contents to pop up at
% the beginning of each subsection:
% \AtBeginSubsection[]
% {
%   \begin{frame}<beamer>
%     \frametitle{Aufbau}
%     \tableofcontents[currentsection,currentsubsection]
%   \end{frame}
% }

\AtBeginSection[]
{
  \begin{frame}<beamer>
    \frametitle{Aufbau}
    \tableofcontents[currentsection,currentsubsection]
  \end{frame}
}


\begin{document}

%\begin{frame}
%  \titlepage
%\note{}
%\end{frame}

%\begin{frame}
%  \frametitle{Aufbau}
%  \tableofcontents
%\note{}
%  % You might wish to add the option [pausesections]
%\end{frame}

%\documentclass[hyperref={xetex}]{beamer}
\title{Wissenschaftliches Rechnen mit Matlab/Python}
\subtitle{Einheit1 - Einleitung, Vektoren und Matrizen}
\mode<article>
{
  \usepackage{fullpage}
  \usepackage{pgf}
  \usepackage[xetex]{hyperref}
  \setjobnamebeamerversion{beamer}
}

\mode<presentation>
{
  %\usetheme{Frankfurt}
 %\usetheme{My}
  \usetheme{Madrid}
  % or ...
%\usecolortheme{seagull}
  %\setbeamercovered{transparent}
  %\setbeamercovered{dynamic}
  % or whatever (possibly just delete it)
}
\usenavigationsymbolstemplate{}
\usefonttheme{structurebold}
\usepackage{multimedia}
%\usepackage{tikz}
\usepackage{fontspec,xunicode,xltxtra}

%\usepackage{polyglossia}
%\setdefaultlanguage[spelling=new, latesthyphen=true]{german}
%\setsansfont{DejaVu Sans}
%\setsansfont{Verdana}
%\setsansfont{Arial}
%\setromanfont{Linux Libertine O}
%\setsansfont{Linux Biolinum O}

\setbeamertemplate{footline}
{
\leavevmode
%\hbox{\begin{beamercolorbox}[wd=.5\paperwidth,ht=2.5ex,dp=1.125ex,
%leftskip=.3cm plus1fill,rightskip=.3cm]{author in head/foot}%
%    \usebeamerfont{author in head/foot}\insertshortauthor
%  \end{beamercolorbox}%
%  \begin{beamercolorbox}[wd=.5\paperwidth,ht=2.5ex,dp=1.125ex,leftskip=.3cm,
%rightskip=.3cm plus1fil]{title in head/foot}%
%    \usebeamerfont{title in head/foot}\insertshorttitle\hfill

\hfill\insertframenumber  \hspace{3pt}

%\inserttotalframenumber
%\hspace*{2ex}
%  \end{beamercolorbox}}%
  \vskip3pt%
}

%\usepackage[english]{babel}
\usepackage[ngerman]{babel}
\selectlanguage{ngerman}

%
% math/symbols
%
\usepackage{amssymb}
\usepackage{amsthm}
% \usepackage{latexsym}
\usepackage{amsmath}
%\usepackage{listings}
\usepackage[framed]{mcode}
%\usepackage{mcode}

\usepackage{mydef}
%\usepackage{cmap} % you can search in the pdf for umlauts and ligatures
%\usepackage{colonequals} %corrects the definition-symbols \colonequals (besides others)
\title{Einführung in Matlab}
%
%\subtitle{Disputation} % (optional)

\author{Jochen Schulz}
% - Use the \inst{?} command only if the authors have different
%   affiliation.

\institute{Georg-August Universit\"at G\"ottingen \pgfimage[height=0.5cm]{../figures/unilogo3}}
% - Use the \inst command only if there are several affiliations.
% - Keep it simple, no one is interested in your street address.

\date{\today}

\subject{Einführung in Matlab}
% This is only inserted into the PDF information catalog. Can be left
% out. 



% If you have a file called "university-logo-filename.xxx", where xxx
% is a graphic format that can be processed by latex or pdflatex,
% resp., then you can add a logo as follows:

%\logo{\pgfimage[height=0.5cm]{figures/unilogo3}}


% Delete this, if you do not want the table of contents to pop up at
% the beginning of each subsection:
% \AtBeginSubsection[]
% {
%   \begin{frame}<beamer>
%     \frametitle{Aufbau}
%     \tableofcontents[currentsection,currentsubsection]
%   \end{frame}
% }

\AtBeginSection[]
{
  \begin{frame}<beamer>
    \frametitle{Aufbau}
    \tableofcontents[currentsection,currentsubsection]
  \end{frame}
}

\AtBeginSubsection[]
{
  \begin{frame}<beamer>
    \frametitle{Aufbau}
    \tableofcontents[currentsection,currentsubsection]
  \end{frame}
}

\begin{document}
\lstset{}


\begin{document}
\titlepage


\begin{frame}{Organisatorisches}
\begin{itemize}
\item Anmeldung $\rightarrow$ StudIP \\
      \url{https://studip.uni-goettingen.de/}

{\color{blue}{Wissenschaftliches Rechnen mit Matlab/Python (Mathematische Anwendersysteme)}}
\item Alle Unterlagen (Aufgabenblätter, Vorlesungsfolien, Beispiele, Musterlösungen) $\rightarrow$ StudIP
\pause
\begin{block}{Dozent}
Jochen Schulz\\
NAM, Zimmer 04 (Erdgescho{\ss})\\
\textbf{Telefon}: 39-4525\\
\textbf{Email}: \href{mailto:schulz@math.uni-goettingen.de}{\texttt{schulz@math.uni-goettingen.de}}\\
\textbf{XMPP}: \url{jschulz1@jabber.gwdg.de}\\

\end{block}
\end{itemize}
\end{frame}


\begin{frame}{Ablauf der Veranstaltung}
\begin{itemize}
\item Blockveranstaltung vom  22.9 - 10.10.2014
\item \alert{Vorlesung:} 9.15 - ca. 11.30  (MN55)
\item \alert{Übungsbetrieb/Praktikum}: ca. 11:30 - ca. 17:00  (MultimediaRaum MI)
\begin{itemize}
\item 1 Übungszettel/Tag.
\item Besprechung Aufgaben vom Vortag (individuell)
\item Betreuung: Hilfestellung beim Bearbeiten der Aufgaben
%\item Klausurzulassung: 3 beliebige markierte Aufgaben/Woche testieren lassen. 
\end{itemize}
\item \alert{Klausur:} 10.10.2014; 13:00 - 14:30
\end{itemize}

\end{frame}

\begin{frame}{Inhalt der Vorlesung}
\begin{description}
\item[1. Tag] Organisatorisches, Basis von Matlab und Python
\item [2. Tag] Programmieren, Datenstrukturen
\item [3. Tag] Rekursionen, Grafik
\item [4. Tag] ? Polynome, Interpolation, Debugging
\item [5. Tag] ? Mehrdimensionale Arrays, Funktionen, Numerische Lineare Algebra, Dünnbesetzte Matrizen
\item [6. Tag] ? Numerische Mathematik, Profiler
\item [7. Tag] ? Visualisierung und Validierung
\item [8. Tag] ? Schnittstelle zu C (Optional)
\item [9. Tag] ? Wunschvorlesung
\item [10. Tag] Fragestunde
\end{description}
\end{frame}

\begin{frame}{Aufbau}
\tableofcontents
\end{frame}

\section{MATLAB und Python Basis}

\subsection{Einleitung}

\begin{frame}{Programmieren für den Wissenschaftler}
  \begin{itemize}
    \item Daten erzeugen oder erheben (Simulation, Experiment)
    \item Weiterverarbeitung von Daten
    \item Visualisierung und Validierung
    \item Ergebnisse veröffentlichen bzw. kommunizieren
  \end{itemize}

  \begin{block}{}
Wir wollen: eine \emph{High-Level} Sprache:
\begin{itemize}
\item Programmieren ist leicht
\item Vorhandene Elemente nutzen
\item geeignet für Prototyping und Debugging (Interaktion)
\item Möglichst nur ein Werkzeug für alle Probleme
\end{itemize}
\end{block}

\end{frame}

%-------------------------------------------------
%  Folie:
%-------------------------------------------------
\begin{frame}[fragile]{MATLAB}

\begin{itemize}
\item MATLAB steht für \alert{Mat}rix \alert{lab}oratory; ursprünglich speziell Matrizenrechnung.
\item Interaktives System für numerische Berechnungen und Visualisierungen (Skriptsprache).
\end{itemize}
\begin{block}{Vorteile}
\begin{itemize}
\item Vielfältige Visualisierungsmöglichkeiten.
\item Viele zusätzliche Toolboxes (Symb. Math T., PDE T., Wavelet T.)  
\item Ausgereifte und integrierte Oberfläche.
\end{itemize}
\end{block}
\begin{block}{Nachteile}
    \begin{itemize}
        \item Kostenintensiv.
        \item Ein/Ausgabe von Dateien kann umständlich sein.
        \item Spezialisierter Funktionsumfang macht manche Programmierung schwer.
    \end{itemize}
  
\end{block}
\end{frame}

\begin{frame}[fragile]{Python: NumPy, SciPy, SymPy}

\begin{itemize}
  \item Modulare Skriptsprache.
\end{itemize}

\begin{block}{Vorteile}
\begin{itemize}
  \item Viele Module mit wissenschaftlichen Fokus.
  \item Klare Code-Struktur.
  \item Ebenso viele Module für den nicht-wissenschaftlichen Gebrauch (nützlich z.B. für Ein-/Ausgabe).
  \item Frei und open-source.
\end{itemize}
  
\end{block}
\begin{block}{Nachteile}
\begin{itemize}
  \item Entwicklungsumgebung etwas komplizierter (Spyder,ipython).
  \item Nicht alle spezialisierten Möglichkeiten anderer Software.
\end{itemize}
  
\end{block}
\end{frame}


%-------------------------------------------------
%  Folie:
%-------------------------------------------------
\begin{frame}[fragile]{Literatur}
  \begin{block}{MATLAB}
  \begin{thebibliography}{10}
\small
\bibitem{1} \alert{Matlab online-help :-)}.
\bibitem{3} \alert{Introduction to Scientific Computing}, C.F. van Loan, Prentice Hall,
New Jersey, 1997,
\bibitem{4} \alert{Scientific Computing with MATLAB}, A. Quarteroni, F. Saleri, Springer, 2003,
\end{thebibliography}
\end{block}
\begin{block}{Python}
  \begin{thebibliography}{10}
      \small
    \bibitem{1} \alert{NumPy, SciPy} SciPy developers (\url{http://scipy.org/}),
    \bibitem{2} \alert{SciPy-lectures}, F. Perez, E. Gouillart, G. Varoquaux, V. Haenel (\url{http://scipy-lectures.github.io}),
    \bibitem{3} \alert{Matplotlib} (\url{http://matplotlib.org})
    \bibitem{4} \alert{scitools} (\url{https://code.google.com/p/scitools/}) 
    \bibitem{5} \alert{mayavi} (\url{http://docs.enthought.com/mayavi/mayavi/mlab.html})
  \end{thebibliography}
\end{block}
\end{frame}


\subsection{Grundlegende Bedienung MATLAB}
%-------------------------------------------------
%  Folie:
%-------------------------------------------------
\begin{frame}[fragile]{MATLAB Fenster-Aufbau}
Starten von MATLAB: Eingabe von \imatlab{matlab &} (in einem Terminal).
\centering\includegraphics[width=1\textwidth]{figures/Screenshot-MATLAB}

\end{frame}

%basis-Befehlsreferenz? clear? pfeiltasten etc. Kommentare?
% vor ort zeigen

%-------------------------------------------------
%  Folie:
%-------------------------------------------------
%\begin{frame}[fragile]{Workspace - globale Variablen}
%\begin{itemize}
%\item Alle definierten (globalen) Variablen werden im Workspace gespeichert.
%\item Zugriff während einer MATLAB-Sitzung.
%\item Inhalt des Arbeitsspeichers: \imatlab{whos} oder \imatlab{who}
%\begin{matlabin}
%>> whos
%  Name      Size            Bytes  Class   
%
%  ans       1x1                 8  double    
%\end{matlabin}
%\item Löschen von Variablen : \imatlab{clear <var>};\\
%\imatlab{clear} löscht den gesamten Arbeitsspeicher (Workspace).
%\end{itemize}
%\end{frame}



%\subsection{Programm-Dateien und der Editor}


%----------------------------
% Folie 
%----------------------------
\begin{frame}[fragile]{Struktur von Skript-Files}
\begin{itemize}
\item Skript-Files bestehen aus einer Sequenz von Befehlen, die
  nacheinander abgearbeitet werden (imperatives Programmierparadgima)

\item operiert auf Variablen im \textit{Workspace}.

\item Gestartet wird das Programm \imatlab{name.m} durch Eingabe von
  \imatlab{name}.

\item Beschreibung des Skript-Files (oder der Funktion):
\begin{matlabin}
help sin
\end{matlabin}
\begin{matlab}
   sin    Sine of argument in radians.
      sin(X) is the sine of the elements of X.
\end{matlab}


\end{itemize}
\end{frame}

%----------------------------
% Folie 
%----------------------------
\begin{frame}[fragile]{Struktur von Function-Files}
\begin{matlabin}
function [Out_1,..,Out_k] = myfunction(In_1,..,In_l)
% Beschreibung der Funktion
 ..
Out_1=..
 ..
Out_k=..
\end{matlabin}
Soll keine Variable zurückgegeben werden, so besteht die erste Zeile aus
\begin{matlabin}
function myfunction(In_1,..,In_k)
\end{matlabin}
\begin{itemize}
\item Call-by-value (Argumente werden im Speicher kopiert)

\item Variablen lokal, d.h.
\begin{itemize}
 \item Variablen des Workspace sind  nicht verfügbar.
\item definierte Variablen werden nicht im  Workspace gespeichert.
\end{itemize}
\end{itemize}
\alert{Wichtig:} Funktionsname = Dateiname.
\end{frame}

%----------------------------
% Folie 
%----------------------------
\begin{frame}[fragile]{Priorität beim Programmaufruf}
Beispiel-Programmaufruf
\begin{matlabin}
name
\end{matlabin}

Testet ob,..
\begin{enumerate}
\item  \alert{Variable}
\item  \alert{Unterfunktion}. Eine
  Unterfunktion ist ein Programm/Funktion, die in derselben Datei wie der
  Aufruf steht.
\item  Programm im \alert{aktuellen Verzeichnis}.
\item  \textit{private function}.
\item  Programm im \alert{Suchpfad}. 
\end{enumerate}
Bei gefundenem Namen wird die Suche beendet.
\end{frame}

%----------------------------
% Folie 
%----------------------------
\begin{frame}[fragile]{Suchpfad}

Der Suchpfad (Variable \imatlab{path}) enthält Verzeichnisse in einer geordneten Liste.

\begin{itemize}
\item Abarbeitung erfolgt der Ordnung gemäss.
\item  Suchpfade hinzufügen:
\begin{matlabin}
addpath <pfadname>
\end{matlabin}
\item Suchpfade entfernen:
\begin{matlabin}
rmpath <pfadname>
\end{matlabin}
\end{itemize}
\end{frame}


%----------------------------
% Folie 
%----------------------------
\begin{frame}[fragile]{Bedienung: Kurzreferenz}
\begin{itemize}
\item \alert{ \imatlab{doc <name>}}\\ öffnet grafisches Hilfefenster zum jeweiligen Programm.
  \item \alert{F5}\\ führt offene Datei im command window aus.
  \item \alert{F9}\\ führt markierte Code-Zeilen aus.
  \item \alert{clear}\\ löscht Variablen im Workspace
\item \alert{ \imatlab{lookfor <name>}} \\ Suche nach \imatlab{name} in den
  Kommentaren zu den Funktionen (auch: grafisches Hilfefenster).
\item  \alert{ \imatlab{what}}\\ m-Files im aktuellen Verzeichnis.
\item  \alert{ \imatlab{type <name>}}\\ Inhalt von \imatlab{name.m} (Command Window).
\item  \alert{ \imatlab{which <name>}}\\ absoluter Pfad der Datei, in dem die  Funktion
  \imatlab{name} gespeichert ist. 
\end{itemize}
\end{frame}


\subsection{Grundlegende Bedienung Python (Spyder)}


\begin{frame}[fragile]{Spyder Fenster-Aufbau}
Starten von Spyder: Eingabe von \imatlab{spyder &} (in einem Terminal).
\centering\includegraphics[width=1\textwidth]{figures/Screenshot_spyder}

\end{frame}

\begin{frame}[fragile]{Struktur von Skript-Files}
\begin{itemize}
\item Skript-Files bestehen aus einer Sequenz von Befehlen, die
  nacheinander abgearbeitet werden, sowie von definierten Funktionen
\item enthält geladene Module.
\item operiert auf globalen Variablen.

\item Gestartet wird das Programm \isage{name.py} durch F5 im Editor oder in einem terminal.

\end{itemize}
\end{frame}

\begin{frame}[fragile]{Funktionen}
Eine Funktion kann man wie folgt definieren:
\begin{pyin}
def <name>(<Argumente>) : 
    """ Beschreibung """
    <Code Block>
    return <Rueckgabe>
\end{pyin}
\begin{itemize}
\item Variablen lokal.
\item \textsl{Call-by-reference} (Referenz zum Objekt wird übergeben)
\item \alert{Objekt-Methoden}:\\
  \begin{itemize}
    \item Objekte besitzen Funktionen, sogenannte Objekt-Methoden.
\item Ein Objekt kennt alle auf sich selbst anwendbaren Funktionen.
  \end{itemize}

\begin{pyin}
f = 3.14
f.is_integer()
\end{pyin}
\end{itemize}

\end{frame}


\begin{frame}[fragile]{Priorität beim Programmaufruf}

Beispiel-Programmaufruf
\begin{pyin}
name
\end{pyin}

Testet ob,..
\begin{enumerate}
\item  \alert{Variable}
\item  \alert{Funktion/Objekt} 
\end{enumerate}
Bei gefundenem Namen wird die Suche beendet.
\end{frame}


\begin{frame}{Bedienung : Kurzreferenz}
  Programm:  spyder 
\begin{itemize}
\item \alert{ \isage{<name>?}}\\ öffnet Hilfe zur jeweiligen Funktion.
\item \alert{F5}\\ führt offene Datei in dedizierter oder vorhandener Shell aus.
\item \alert{F9}\\ führt markierte Code-Zeilen aus.
\item \alert{\isage{<name>.TAB}} \\ zeigt alle Objektmethoden.
\end{itemize}
\end{frame}

\subsection{Erstes Beispiel}
%----------------------------
% Folie 
%----------------------------
\begin{frame}[fragile]{Graph eines Polynoms}

\alert{Aufgabe:}\\
Zeichnen Sie  den Graphen eines Polynoms
\vspace*{-0.4cm}
\[ p(x)= \sum_{i=0}^N a_i x^i, \quad a_i \in \mathbb{R} 
\vspace*{-0.4cm} \]
Zu Werten $(x_i)_{i=1}^n$ muß man $(p(x_i))_{i=1}^n$ berechnen.

\end{frame}

%----------------------------
% Folie 
%----------------------------
\begin{frame}[fragile]{Skalare Version}
\begin{matlabin}[title=\tiny matlab]
function y=ausw_poly1(a,x)

n = length(a);
aux_vector = x.^(0:n-1);
y = aux_vector*transpose(a);
\end{matlabin}
\begin{pyin}[title=\tiny python]
def ausw_poly1(a,x):
    n = len(a)
    aux_vector = x**np.array(range(0,n))
    return dot(aux_vector,a)
\end{pyin}


\end{frame}
%----------------------------
% Folie 
%----------------------------
\begin{frame}[fragile]{Vektorielle Version}
\begin{matlabin}[title=\tiny matlab]
function y = ausw_poly2(a,x)
n = length(a);
k = length(x);
A = repmat(transpose(x),1,n);
B = repmat(0:(n-1),k,1);
y = (A.^B)*transpose(a);
\end{matlabin}
\begin{pyin}[title=\tiny python]
def ausw_poly2(a,x):
    n = len(a)
    k = len(x)
    xm = np.array([x])
    A = sp.repeat(xm.T, n,1)
    B = sp.repeat(np.array([range(0,n)]), k,0)
    return dot(A**B,a)
\end{pyin}
\end{frame}
%----------------------------
% Folie 
%----------------------------
\begin{frame}[fragile]{Plotten des Polynoms}
\begin{matlabin}[title=\tiny matlab]
a = [9 0 -10 0 1]; % Koeffizienten

x = linspace(0,4,30); % Betrachte [0,4]
y = ausw_poly2(a,x);
% Plotten
plot(x,y,'r*-','LineWidth',3,'MarkerSize',4)
\end{matlabin}
\begin{pyin}[title=\tiny python]
a = np.array([9,0,-10,0,1])

x = np.linspace(0,4,30) # Betrachte [0,4]
y = ausw_poly2(a,x)
#Plotten
plot(x,y,'r*-',linewidth=3,markersize=8)
show()
\end{pyin}
\end{frame}
%----------------------------
% Folie 
%----------------------------
\begin{frame}[fragile]{Plotten des Polynoms}
\begin{center}
\begin{columns}[c]
\column{0.48\textwidth}
\includegraphics[width=0.8\textwidth]{figures/polynom_ma} 

\centerline{Matlab}
\column{0.48\textwidth}
\includegraphics[width=0.8\textwidth]{figures/polynom_py} 

\centerline{Python}
\end{columns}
\end{center}

\end{frame}

\section{Programmieren: Basis}

\begin{frame}[fragile]{Bezeichner}
\begin{itemize}
\item \alert{Bezeichner} sind Namen, wie z.B. $x$ oder $f$. Sie können
im mathematischen Kontext sowohl Variablen als auch Unbestimmte repräsentieren.
\item Bezeichner sind aus Buchstaben, Ziffern und
Unterstrich \_ zusammengesetzt.
\item Man unterscheidet zwischen Groß- und Kleinschreibung.
\item Bezeichner dürfen nicht mit einer Ziffer beginnen.
\end{itemize}
\textbf{Beispiele}
\begin{itemize}
\item zulässige Bezeichner:
\isage{x}, \isage{f}, \isage{x23}, \isage{_x_1}
\item unzulässige Bezeichner:
\isage{12x}, \isage{p~}, \isage{x>y}, \isage{Das System}
\end{itemize}
\end{frame}

\begin{frame}[fragile]{Typen und Werte eines Bezeichners}
\begin{itemize}
\item \alert{Datentyp}: Eigenschaften der Daten. \\
  \textbf{Beispiel}: ganze Zahlen, Zeichenketten, Gleitkommazahlen, Listen  \ldots  
  \begin{itemize}
    \item Matlab: 2D-Arrays vom Typ \imatlab{Double} vorherrschend
    \item Python: \isage{integer}, \isage{float}, \isage{numpy.ndarray}, \isage{list},..
  \end{itemize}
\item \alert{Objekt}: Instanz (Einheit) eines Datentyps.
\item \alert{Wert} eines Bezeichners: ein \alert{Objekt} eines bestimmten
\alert{Datentyps}.
\end{itemize}
\end{frame}

\begin{frame}[fragile]{Zuweisungsoperator $=$}
    \begin{pyin}
<bezeichner> = <wert>
    \end{pyin}   
Zuweisung des Wertes \isage{wert} zu dem Bezeichner \isage{bezeichner}. Dabei wird der Typ automatisch festgelegt.
\begin{itemize}
\item \textbf{Warnung:} Unterscheiden Sie  stets zwischen dem Zuweisungsoperator {\color{blue} $=$} und
dem logischen Operator {\color{blue} $==$}.   
\item Löschen von Zuweisungen/Variablen.
\begin{itemize}
\item Python: {\color{blue} \isage{\%clear <bezeichner>}}
\item Matlab: {\color{blue} \isage{clear <bezeichner>}}
\end{itemize}
\end{itemize}

\textsl{Beispiel}:

\begin{pyin}
  a = 5.5
  type (a)
\end{pyin}
\begin{pyout}
  float
\end{pyout}

%\item {\color{blue} \isage{func(arg)=expr(arg)}}: Definition der Funktion \isage{func} mit dem Argument \isage{arg} und Zuweisung des Ausdrucks \isage{expr} zu (abhängig von \isage{arg})
%%\item Rückgabeparameter ist die rechte Seite (Eine Ausgabe erfolgt jedoch normalerweise nicht)
\end{frame}

\begin{frame}[fragile]{Python: Listen und Tuple}
 \begin{itemize}
\item Eine \alert{Liste} ist in Python mit \isage{[..,..]} gekennzeichnet (hat Ordnung, veränderbar (mutable)) 
\begin{pyin}
liste = [21,22,24,23]
liste.sort(); liste 
\end{pyin}
\begin{pyout}
 [21, 22, 23, 24]
\end{pyout}
\item Ein \alert{Tuple} ist in Python mit \isage{(..,..)} gekennzeichnet (hat Struktur, nicht veränderbar (immutable))
\begin{pyin}
tuple = (liste[0], liste[2])
tuple, tuple[0]
\end{pyin}
\begin{pyout}
(( 21, 24), 21) 
\end{pyout}
 \end{itemize}
\end{frame}


\section{Vektoren und Matrizen}



\subsection{Erzeugen von Vektoren}
%
% Slide: 
%
\begin{frame}[fragile]{}
\begin{itemize}
\item Erzeugen 'per Hand'
\begin{matlabin}
b = [1 2 4]
\end{matlabin}
\begin{pyin}
b = np.array([1,2,4])
\end{pyin}

\item Abfragen der Einträge von $b$
\begin{matlabin}
b(2)
\end{matlabin}
\begin{pyin}
b[1]
\end{pyin}

Index $\equiv$ Position im Vektor\\

\alert{Achtung}: 

\begin{itemize}
  \item \emph{Matlab}: Indizes beginnen immer mit $1$!
\item \emph{Python}: Indizes beginnen immer mit $0$!
\end{itemize}

\end{itemize}
\end{frame}

%
% Slide: 
\begin{frame}[fragile]{Matlab: Doppelpunkt - Notation}
\imatlab{x:s:z} erzeugt einen Vektor der Form 
\[ (x,x+s,x+2s,x+3s, \ldots ,z). \]
$x + ns \leq z$ für alle $n$. 
Schrittweite $s$ ist 1 wenn nur ein Doppelpunkt benutzt wird.
\begin{matlabin}
>> a = 2:11
a =
 2  3  4  5  6  7  8  9  10  11

>> c = -2:0.75:1
c =
 -2.0000 -1.2500 -0.5000 0.2500 1.0000
\end{matlabin}
\end{frame} 

\begin{frame}[fragile]{Python: ogrid-Slicing}
\isage{ogrid[x:s:z]} erzeugt einen Vektor der Form 
\[ (x,x+s,x+2s,x+3s, \ldots , z). \]
$x + ns \le z$ für alle $n$.
Schrittweite $s$ ist 1 wenn nur ein Doppelpunkt benutzt wird.
\begin{pyin}
>>> a = ogrid[2:12]
array([ 2,  3,  4,  5,  6,  7,  8,  9, 10, 11])

>>> c = ogrid[-2:1.1:0.75]
array([-2.  , -1.25, -0.5 ,  0.25,  1.  ])

\end{pyin}
\end{frame} 

%
% Slide: 
%
\begin{frame}[fragile]{}
\begin{itemize}
\item \imatlab{length(a)} |  \isage{len(a)}\\ Länge des Vektors $a$ an.
\item \isage{linspace(x1,x2,N)} oder \isage{ogrid[x1:x2:Nj]}\\ Vektor
\[ x1, x1+\frac{x2-x1}{N-1}, x1+2 \frac{x2-x1}{N-1}, \dots ,x2  \]
der Länge $N$.
\begin{pyin}
linspace(1,2,4)
\end{pyin}
\begin{pyout}
array([ 1.  ,  1.33333333,  1.66666667,  2. ])
\end{pyout}

\item \imatlab{logspace(x1,x2,N)}\\ wie \imatlab{linspace}, nur logarith. Skalierung
\end{itemize}
\end{frame}

\subsection{Erzeugen von Matrizen}
%
% Slide: 
%
\begin{frame}[fragile]{}
\begin{itemize}
\item Erzeugen 'per Hand'
\begin{matlabin}
B = [1 3 4; 5 6 7]
\end{matlabin}
\begin{pyin}
B = np.array([[1,3,4],[5,6,7]])
#alternativ (Unterschiedlicher Datentyp!)
B = matrix([[1,3,4],[5,6,7]]) 
\end{pyin}

\item \imatlab{eye(n,m)}\\ $(n \times m)$-EinheitsMatrix)
\begin{matlabin}
eye(2,3)
\end{matlabin}
\begin{matlab}
ans =
     1     0     0
     0     1     0 
\end{matlab}

($1$ auf der Hauptdiagonalen und 0 sonst).
\end{itemize}
\end{frame} 
%
% Slide: 
%
\begin{frame}[fragile]{}
\begin{itemize}
\item \imatlab{zeros(n,m)} | \isage{zeros((n,m))} \\$(n \times m)$- Matrix mit $0$ als Einträge.
\item \imatlab{ones(n,m)} |  \isage{ones((n,m))}\\$(n \times m)$- Matrix mit $1$ als Einträge.
\item Blockmatrizen
\begin{matlabin}
C = [B zeros(2,2); eye(2,3) eye(2,2)]
\end{matlabin}
\begin{pyin}
C = vstack([hstack([B, zeros((2,2))] ) ,hstack([eye(2,3), eye(2,2)]) ])
\end{pyin}
\begin{matlab}
C =
     1     3     4     0     0
     5     6     7     0     0
     1     0     0     1     0
     0     1     0     0     1 
\end{matlab}

\alert{Achtung:} Matrizen in einer Zeile müssen dieselbe
Zeilenanzahl haben und Matrizen in einer Spalte dieselbe Spaltenanzahl.
\end{itemize}
\end{frame} 
%
% Slide: 
%
\begin{frame}[fragile]{}
\begin{itemize}
\item \imatlab{repmat(A,n,m)}| \isage{tile(A,(n,m)}\\ Blockmatrix mit $(n \times m)$
  aus A bestehenden Blöcken zusammenhängen
\begin{matlabin}
D = repmat(B,1,2) 
\end{matlabin}
\begin{pyin}
D = tile(B,(1,2))
\end{pyin}
\begin{matlab}
D =
     1     3     4     1     3     4
     5     6     7     5     6     7 
\end{matlab}


%\item \imatlab{blkdiag(A,B)}\\ Blockdiagonalmatrix.
\item \imatlab{diag(v,k)} \\Matrix der Größe $(n+|k|) \times
  (n+|k|)$ mit den Einträgen des Vektors $v$ (Länge $n$) auf der $k$-ten Nebendiagonalen. 
\end{itemize}
\end{frame}
%
% Slide: 
%
\begin{frame}[fragile]{Beispiel- und Spezial-Matrizen}
\begin{itemize}
\item Beispiel
\begin{matlabin}
E = vander(linspace(1,3,3))
\end{matlabin}
\begin{matlab}
E =
     1     1     1
     4     2     1
     9     3     1
\end{matlab}

\item Hilbert-Matrix: 
  \begin{itemize}
    \item \imatlab{hilb}(Matlab) 
    \item \isage{scipy.linalg.hilbert}(Python)
  \end{itemize}
\item Hadamard-Matrix:
  \begin{itemize}
    \item \imatlab{hadamard} (Matlab) 
    \item \isage{scipy.linalg.hadamard}(Python)
  \end{itemize}
\end{itemize}
\end{frame}

\subsection{Zugriff und Manipulation von Matrizen}
%
% Slide: 
%
\begin{frame}[fragile]{Matlab: Zugriff auf Matrizen}
\begin{columns}[c]%
\column{0.45\textwidth}%
\begin{matlabin}[basicstyle=\tiny]
A = [1 2 3; 4 5 6; 7 8 9]
\end{matlabin}%
\begin{matlab}
A =
     1     2     3
     4     5     6
     7     8     9 
\end{matlab}%
\end{columns}%
\begin{columns}[t,onlytextwidth]
\column{0.45\textwidth}
Abfragen eines Eintrags
\begin{matlabin}
>> A(2,1)
ans =
     4
\end{matlabin}
\column{0.45\textwidth}
Abfrage von Blöcken
\begin{matlabin}
>> A(2:3,1:2)
ans =
     4     5     
     7     8     
\end{matlabin}
\end{columns}
\begin{columns}[t,onlytextwidth]
\column{0.45\textwidth}
Abfrage einer Zeile
\begin{matlabin}
>> A(2,:)
ans =
     4     5     6
\end{matlabin}
\column{0.45\textwidth}
Abfrage mehrerer Zeilen
\begin{matlabin}
>> A([1 3],:)
ans =
     1     2     3
     7     8     9
\end{matlabin}
\end{columns}
\end{frame}


\begin{frame}[fragile]{Python: Zugriff auf Matrizen}
\begin{columns}[c]%
\column{0.45\textwidth}%
\begin{pyin}[basicstyle=\tiny]
A = np.array([[1,2,3],[4,5,6],[7,8,9]])
\end{pyin}%
\begin{pyout}
array([[1, 2, 3],
       [4, 5, 6],
       [7, 8, 9]])
\end{pyout}%
\end{columns}%
\begin{columns}[t,onlytextwidth]
\column{0.45\textwidth}
Abfragen eines Eintrags
\begin{pyin}
>>> A[1,0]
4
\end{pyin}
\column{0.45\textwidth}
Abfrage von Blöcken
\begin{pyin}
>>> A[1:3,0:2]
array([ 4, 5],
       [7, 8]])]
\end{pyin}
\end{columns}
\begin{columns}[t,onlytextwidth]
\column{0.45\textwidth}
Abfrage einer Zeile
\begin{pyin}
>>> A[1,:]
array([4, 5, 6])
\end{pyin}
\column{0.45\textwidth}
Abfrage mehrerer Zeilen
\begin{pyin}
>>> A[(0,2),:]
array([[1, 2, 3],
       [7, 8, 9]])
\end{pyin}
\end{columns}
\end{frame}

%
% Slide: 
%
\begin{frame}[fragile]{Matlab: Setzen und Löschen}
\begin{columns}[c]%
\column{0.45\textwidth}%
\end{columns}%
\begin{columns}[t]
\column{0.45\textwidth}%
Setzen einer Zeile
\begin{matlabin}
>> A(2,:) = [10,10,10]
A =

     1     2     3
    10    10    10
     7     8     9
\end{matlabin}
\column{0.45\textwidth}%
Setzen von Spalten
\begin{matlabin}
>> A(:,1) = [10,10,10]
A =

    10     2     3
    10     5     6
    10     8     9
\end{matlabin}
\end{columns}
\begin{columns}[c]%
\column{0.45\textwidth}%
\end{columns}%
\begin{columns}[t]
\column{0.45\textwidth}%
Löschen einer Zeile
\begin{matlabin}
>> A(2,:) = []
A =
     1     2     3
     7     8     9
\end{matlabin}
\column{0.45\textwidth}%
Löschen von Spalten
\begin{matlabin}
>> A(:,[1 3]) = []
A =
     2
     5
     8
\end{matlabin}
\end{columns}
\end{frame}

%
% Slide: 
%
\begin{frame}[fragile]{Python: Setzen und Löschen}
\begin{columns}[c]%
\column{0.45\textwidth}%
\end{columns}%
\begin{columns}[t]
\column{0.45\textwidth}%
Setzen einer Zeile
\begin{pyin}
>>> A[1,:] = [10,10,10]
array([[ 1,  2,  3],
       [10, 10, 10],
       [ 7,  8,  9]])
\end{pyin}
\column{0.45\textwidth}%
Setzen von Spalten
\begin{pyin}
>>> A[:,0] = [10,10,10]
array([[10,  2,  3],
       [10,  5,  6],
       [10,  8,  9]])
\end{pyin}
\end{columns}
\begin{columns}[c]%
\column{0.45\textwidth}%
\end{columns}%
\begin{columns}[t]
\column{0.45\textwidth}%
Löschen einer Zeile
\begin{pyin}
>>> np.delete(A,1,0)
array([[1, 2, 3],
       [7, 8, 9]])
\end{pyin}
\column{0.45\textwidth}%
Löschen von Spalten
\begin{pyin}
>>> np.delete(A,[0,2],1)
array([[2],
       [5],
       [8]])
\end{pyin}
\end{columns}
\end{frame}

\subsection{Matrix- und Vektoroperationen}
%
% Slide: 
%
\begin{frame}[fragile]{Skalarprodukt}
Skalarprodukt: $a b^t$\\ mit 
Vektoren $a=(a_1, \dots ,a_n)$, $b=(b_1, \dots b_n)$ \\
\begin{itemize}
\item \imatlab{a*transpose(b)} | \isage{dot(a,b)}\\
  Skalarprodukt im Code
\end{itemize}
Beispiel:
\begin{matlabin}
>> a=1:100; b=linspace(0,1,100);
>> a*transpose(b)
   3.3667e+03
>> ones(1,100)*transpose(a)
        5050
\end{matlabin} 
\begin{pyin}
>>> a=linspace(1,100,100); b=linspace(0,1,100)
>>> dot(a,b)
3366.666666666667
>>> dot(ones((1,100)),a)
array([ 5050.])
\end{pyin}
\end{frame}
%
% Slide: 
%
\begin{frame}[fragile]{Matrix-Matrix und Matrix-Vektor}

  Standard-Matrix Operationen \imatlab{+,-,*} | \isage{+,-,dot()}
\begin{matlabin}
A = [1 2; 3 4]; B = 2*ones(2,2);
\end{matlabin}
\begin{pyin}
A = np.array([[1,2],[3,4]]); B = 2*ones((2,2))
\end{pyin}
\begin{columns}[t]%
\column{0.25\textwidth}%
Multiplikation
\begin{matlabin}
>>  A*B
ans =

     6     6
    14    14
\end{matlabin}
\begin{pyin}
>>> dot(A,B)
[[  6.,   6.],
 [ 14.,  14.]]
\end{pyin}
\column{0.25\textwidth}%
Addition
\begin{matlabin}
>> A+B
ans =

     3     4
     5     6
\end{matlabin}
\begin{pyin}
>>> A + B
([[ 3.,  4.],
  [ 5.,  6.]])
\end{pyin}

\column{0.25\textwidth}%
Subtraktion
\begin{matlabin}
>> A-B
ans =

    -1     0
     1     2
\end{matlabin}
\begin{pyin}
>>> A - B
([[ -1.,  0.],
  [ 1.,  2.]])
\end{pyin}
\end{columns}
\end{frame}
%
% Slide: 
%
\begin{frame}[fragile]{}
\begin{itemize}
\item \imatlab{.*, ./} |  \isage{*,/}\\
\alert{Wichtig:} komponentenweise Multiplikation/Division 
\item \imatlab{A\\\B} |  \isage{solve(A,B)}\\ Lösung $X$ von \imatlab{A*X=B}. \\

\item \imatlab{A/B} | \isage{solve(A.T,B.T)}\\ Lösung $X$ von \imatlab{X*A=B}.\\

\item \imatlab{inv(A)}\\ Inverse von $A$.\\

\item \imatlab{A'} | \isage{A.conj().T}\\ komplex Transponierte von $A$. \\

\item \imatlab{A.'} |  \isage{A.T}\\ Transponierte von $A$. \\

\item \imatlab{A^z} |  \isage{power(A,z)}\\ (quadratische Matrizen) $\underbrace{A*A*\cdots *A}_{z-mal}$ \\

\item \imatlab{size(A)} | \isage{A.shape}\\ Gr\"o{\ss}e einer Matrix $A$ . 
\end{itemize}
\end{frame}


\end{document}

%\subtitle{Einheit 2}

\begin{frame}[fragile]
  \titlepage
\note{}
\end{frame}

\begin{frame}[fragile]
  \frametitle{Aufbau}
  \tableofcontents
\note{}
  % You might wish to add the option [pausesections]
\end{frame}


\section{Programmieren mit MATLAB}
\subsection{Motivation}

%----------------------------
% Folie 
%----------------------------
\begin{frame}[fragile]\frametitle{Zwei-Punkt-Randwert-Aufgabe}

Suche eine Funktion 
\[ u:[0,1] \quad \rightarrow \quad \mathbb{R}, \] 
so dass 
\begin{eqnarray*}
-u''(x) & = & f(x), \quad x \in (0,1)\\
u(0) & = & u(1) =0
\end{eqnarray*}

\alert{Lösen mit Finite-Differenzen-Verfahren}\\
Diskretisierung: $0=x_{0} < \dots < x_{n}=1$ mit $x_i=\frac{i}{n}$\\
\end{frame}
%----------------------------
% Folie 
%----------------------------
\begin{frame}[fragile]\frametitle{Implementierung in MATLAB}
$n=101$, $f \equiv 1$

\begin{lstlisting}
>> x=0:(1/101):1;
>> x_i=x(2:101);
>> A=diag(2*ones(1,100),0)...
   +diag(-1*ones(1,99),-1)...
   +diag(-1*ones(1,99),1);
>> F=(1/101)^2*ones(100,1);
>> z_i=A\F;
>> z=[0; z_i; 0];
>> plot(x,z,'r*-');
\end{lstlisting}
\end{frame}

%----------------------------
% Folie 
%----------------------------
\begin{frame}[fragile]\frametitle{Motivation}
\alert{Probleme:} 
\begin{itemize}
\item
Bei jeder Änderung von $n$ muss alles erneut im
interaktiven Modus eingegeben werden.\\
\item 
Abrufen der Befehle bei späteren Sitzungen ist kaum möglich. 
\item Bei komplexen Algorithmen wird es unübersichtlich.
\end{itemize}
\alert{Ausweg:} Die Befehlsfolge wird in einer Datei
abgelegt. MATLAB arbeitet dann sukzessive die einzelnen Kommandos
ab. \\
\end{frame}

%----------------------------
% Folie 
%----------------------------
\begin{frame}[fragile]\frametitle{randwertaufgabe.m}
\begin{lstlisting}
%------------------------------------
%     randwertaufgabe.m 
%   
%  berechnet mit Finiten Differenzen die Lösung u von
%  -u''=f in (0,1), u(0)=u(1)=0
%
%-------------------------------------------

% Anzahl Stützstellen
n=5;

% Erzeugen des Gitters
x=0:(1/n):1;
x_i=x(2:n);
\end{lstlisting}
\end{frame}

\begin{frame}[fragile]\frametitle{randwertaufgabe.m}
\begin{lstlisting}
% Aufstellen des lin. Gls.
A=diag(2*ones(1,n-1),0)...
   +diag(-1*ones(1,n-2),-1)...
   +diag(-1*ones(1,n-2),1);
F=(1/n)^2*ones(n-1,1); % rechte Seite für f=1 

% Lösen des lin. Gls.
z_i=A \F;

% Darstellen der Lösung
z=[0; z_i; 0];
plot(x,z,'r*-');
\end{lstlisting}
\end{frame}

%----------------------------
% Folie 
%----------------------------
\begin{frame}[fragile]\frametitle{Erzeugen eines Programms}
\begin{itemize}
\item Starten des Editors: \lstinline! >> edit !; \lstinline! >> edit datei_name!
  öffnet die Datei \lstinline!datei_name!.
\item Speichern der Datei mit Hilfe des Menüs: \lstinline!File->Save!
  bzw. \lstinline!File->Save As!.
\end{itemize}
\alert{Achtung:} Alle MATLAB-Dateien haben die Endung '.m'. Man
spricht deswegen auch von $m$-Files.
\end{frame}

%----------------------------
% Folie 
%----------------------------
\begin{frame}[fragile]\frametitle{Starten von Programmen}
\begin{itemize}
\item Befindet man sich im selben Verzeichnis wie das Programm
  \lstinline!name.m!, so kann man das Programm starten durch Eingabe von  
\lstinline!name!. 
\item Danach durchsucht MATLAB die in \alert{ \lstinline!path!} angegebenen
  Verzeichnisse nach dem Programm.
\item Mit dem Befehl \lstinline!addpath pfadname! kann man eigene Suchpfade
  hinzufügen.  
\item Durch \lstinline!rmpath pfadname! kann man Suchpfade entfernen.  
\end{itemize}
\end{frame}

%----------------------------
% Folie 
%----------------------------
\begin{frame}[fragile]\frametitle{Graph eines Polynoms}

\alert{Aufgabe:}\\
Zeichnen Sie  den Graphen eines Polynoms
\vspace*{-0.4cm}
\[ p(x)= \sum_{i=0}^N a_i x^i, \quad a_i \in \mathbb{R} 
\vspace*{-0.4cm} \]
\alert{Problem:}\\
Zu Werten $(x_i)_{i=1}^n$ muß man $(p(x_i))_{i=1}^n$ berechnen,
d.h. Funktionswerte müssen sehr oft berechnet werden.

\alert{Lösung:}\\
Es gibt Funktionen in MATLAB.
\end{frame}

%----------------------------
% Folie 
%----------------------------
\begin{frame}[fragile]\frametitle{Skalare Version}
\begin{lstlisting}
function y=ausw_poly1(a,x)
%----------------------------------------------------
% ausw_poly berechnet den Funktionswert von 
%           p(x)=a_1 +a_2 x + a_3 x^2+ ... +a^n x^(n-1)
%           INPUT:  a Vektor der Koeffizienten 
%                   x  auszuwertender Punkt
%           OUTPUT: y  Funktionswert (y=p(x))
%  Gerd Rapin           1.11.2003
%------------------------------------------------------

n=length(a);
aux_vector=x.^(0:n-1);
y=aux_vector*transpose(a);
\end{lstlisting}
\end{frame}
%----------------------------
% Folie 
%----------------------------
\begin{frame}[fragile]\frametitle{Vektorielle Version}
\begin{lstlisting}
function y=ausw_poly2(a,x)
%----------------------------------------------------
% ausw_poly berechnet den Funktionswert von 
%           p(x)=a_1 +a_2 x + a_3 x^2+ ... +a^n x^(n-1)
%           INPUT:  a Vektor der Koeffizienten 
%                   x Vektor der auszuwertenden Punkte
%           OUTPUT: y Vektor der Funktionswerte
%  Gerd Rapin           1.11.2003
%------------------------------------------------------

n=length(a);
k=length(x);
A=repmat(transpose(x),1,n);
B=repmat(0:(n-1),k,1);

y=(A.^B)*transpose(a);
\end{lstlisting}
\end{frame}
%----------------------------
% Folie 
%----------------------------
\begin{frame}[fragile]\frametitle{Plotten des Polynoms}
\begin{lstlisting}
%------------------------------------
%     plot_poly.m 
%   
%  zeichnet den Graphen eines Polynoms
%  Gerd Rapin           1.11.2003
%-------------------------------------------

% Koeffizienten
a=[-6 11 -6 1]; 

x=linspace(0,4,30); % Betrachte [0,4]
y=ausw_poly2(a,x);

% Plotten
plot(x,y,'r*-','LineWidth',3,'MarkerSize',4)
\end{lstlisting}
\end{frame}
%----------------------------
% Folie 
%----------------------------
\begin{frame}[fragile]\frametitle{Plotten des Polynoms}
\[p(x) = (x-1)(x-2)(x-3) \]
\begin{center}
\includegraphics[width=0.6\textwidth]{./figures/polynom_03_11} 
\end{center}

\end{frame}
\subsection{Skript-Files}
%----------------------------
% Folie 
%----------------------------
\begin{frame}[fragile]\frametitle{Struktur von Skript-Files}
\begin{itemize}
\item Skript-Files bestehen aus einer Sequenz von Befehlen, die
  nacheinander abgearbeitet werden.
\item Files werden mit der Endung '.m' gespeichert. 
\item Gestartet wird das Programm \lstinline!name.m! durch Eingabe von
  \lstinline!name!.
\item Kommentare beginnen mit \lstinline!%!.
\end{itemize}
\end{frame}
%----------------------------
% Folie 
%----------------------------
\begin{frame}[fragile]\frametitle{Struktur von Skript-Files II}
\begin{itemize}
\item Am Anfang des Files soll als Kommentar der Name des Programms,
  eine kurze Beschreibung, Name des Autors und das Erstellungsdatum stehen. 
\item operiert auf Daten im {\it Workspace}.
\item Beschreibung des Skript-Files erhält man mit
\begin{lstlisting}
>> help plot_poly
 ------------------------------------
      plot_poly.m    
   zeichnet den Graphen eines Polynoms 
   Gerd Rapin           1.11.2003
 -------------------------------------------
\end{lstlisting}
\end{itemize}
\end{frame}
\subsection{Function-Files}
%----------------------------
% Folie 
%----------------------------
\begin{frame}[fragile]\frametitle{Struktur von Function-Files}
Beispiel: 'my-file.m'
\begin{lstlisting}
function [Out_1,...,Out_k]=my-file(In_1,...,In_l)
% Beschreibung der Funktion
 ...
Out_1=...
 ...
Out_k=...
\end{lstlisting}
\alert{Wichtig:} Funktionsname muss identisch sein mit dem Dateinamen.
\end{frame}
%----------------------------
% Folie 
%----------------------------
\begin{frame}[fragile]\frametitle{Function-Files}
\begin{itemize}
\item Funktionen sind mit Kommentaren zu versehen, was das Programm
  macht, welche Input- und Output-Argumente es hat, und wann und von
  wem es erstellt wurde.
\item Variablen werden nur lokal gehalten; die Variablen des Workspace
  sind innerhalb des Workspace nicht verfügbar; im Programm definierte Variablen werden nicht im
  Workspace gespeichert.
\item Soll keine Variable zurückgegeben werden, so besteht die erste
  Zeile aus
\begin{lstlisting}
function []=myfile(In_1,...,In_k)
\end{lstlisting}
%oder kurz \alert{ \lstinline!function myfile(In_1,...,In_k)!}.
\end{itemize}
\end{frame}
%----------------------------
% Folie 
%----------------------------
\begin{frame}[fragile]\frametitle{Verwalten von m-Files}
\begin{itemize}
\item \alert{ \lstinline!lookfor name!} sucht nach dem Stichwort \lstinline!name! in den
  Kommentaren zu den Funktionen.
\item  \alert{ \lstinline!what!} zeigt die m-Files im aktuellen Verzeichnis an.
\item  \alert{ \lstinline!type name!} zeigt den Inhalt von \lstinline!name.m! im 'Command
  Window' an.
\item  \alert{ \lstinline!which name!} gibt den genauen Pfad an, in dem die  Funktion
  \lstinline!name.m! gespeichert ist. 
\end{itemize}
\end{frame}
%----------------------------
% Folie 
%----------------------------
\begin{frame}[fragile]\frametitle{Priorität beim Programmaufruf}
\begin{itemize}
\item [1.] Testet, ob der Name eine Variable ist.
\item [2.] Testet, ob der Name eine Unterfunktion ist. Eine
  Unterfunktion ist ein Programm, das in derselben Datei wie der
  Aufruf steht.
\item [3.] Testet, ob das Programm im aktuellen Verzeichnis steht.
\item [4.] Testet, ob der Name eine {\it private function} ist.
\item [5.] Testet, ob das Programm im Suchpfad enthalten ist. 
\end{itemize}
\end{frame}


\section{Programmieren - Teil II}
\subsection{Gültigkeitsbereich von Variablen}
%\item Schleifen
%\item Bedingungen
%\item Aufgaben
 
%
% Slide
%
\begin{frame}[fragile]\frametitle{Gültigkeitsbereich von Variablen}
\begin{itemize}
\item \alert{Variablen in Skript-Files} benutzen den globalen Workspace,
  d.h. bereits vorhandene Variablen können direkt benutzt oder
  überschrieben werden. Sie sind gültig bis sie explizit gelöscht
  werden.
\item \alert{Variablen in Function-Files} sind nur innerhalb der
  Funktion definiert und werden bei Verlassen der Funktion
  gelöscht. Variablen des globalen Workspace können nicht benutzt
  werden. 
\end{itemize}
\end{frame}
%
% - Folie
%
\begin{frame}[fragile]\frametitle{Fixpunkt}
Suche ein $x_f \in \mathbb{R}$ so dass
\[ x_f = \cos (x_f ) \]
\begin{center}
\includegraphics[height=5cm]{./figures/fixpunkt}
\end{center}
\end{frame}
%
% - Folie
%
\begin{frame}[fragile]\frametitle{Fixpunkt-Iteration}
Fixpunkt-Iteration 
\[ x_{k+1}=cos(x_k) \]
bei geeignetem Startwert $x_0$.  \\
\centering{\includegraphics[height=5cm]{./figures/fixpunkt1}}\\
\end{frame}

%
% - Folie
%
\begin{frame}[fragile]\frametitle{Implementierung}
\begin{lstlisting}
% Plot 1
x=linspace(0,1.5,50);
y=cos(x);
plot(x,x,x,y,'LineWidth',3),
axis([-0.1 1.5 -0.1 1.1]);
hold on;
pause; % stoppt bis eine Taste gedrückt wird
z(1)=0.1; % Anfangswert
it_max=10; % Iterationsschritte 
for i=1:it_max
    z(i+1)=cos(z(i));
    plot([z(i) z(i)], [z(i) z(i+1)],'r--','LineWidth',1);
    pause;
    plot([z(i) z(i+1)],[z(i+1) z(i+1)],'r--','LineWidth',1);
    hold on;
    pause; % stoppt bis eine Taste gedrückt wird
end;
\end{lstlisting}
\end{frame}
%
% - Folie
%
\begin{frame}[fragile]\frametitle{Einige Grafikbefehle}
\begin{itemize}
\item Durch \alert{ \lstinline!figure!} wird ein Grafik-Fenster gestartet.
\item Mittels \alert{ \lstinline!hold on!} werden alle Grafiken in einem Fenster
  \"ubereinander gezeichnet. 
\item Im Standardmodus wird bei jedem Grafikbefehl die bestehende Grafik
  gel\"oscht und durch die neue Grafik ersetzt.
\item Mittels \alert{ \lstinline!hold off!} wird zur\"uck in den Standardmodus
  gewechselt.  
\end{itemize}
\end{frame}
%
% - Folie
%
\begin{frame}[fragile]\frametitle{for - Schleife}
\begin{lstlisting}
for variable = Ausdruck
  Befehle
end
\end{lstlisting}
\alert{Bemerkungen:} 
\begin{itemize}
\item Der Ausdruck ist normalerweise von der Form
\lstinline!i:s:j!. 
\item Die {\it Befehle} werden eingerückt. 
\end{itemize}
\end{frame}
%
% - Folie
%
\begin{frame}[fragile]\frametitle{Beispiele}
\begin{itemize}
\item Berechne $\sum_{i=1}^{1000} \frac{1}{i}$
\begin{lstlisting}
>> sum=0; for j=1:1000, sum=sum+1/j; end, sum
sum =  7.4855
\end{lstlisting}
\item Berechnen dreier Werte
\begin{lstlisting}
>> for x=[pi/6 pi/4 pi/3], sin(x), end
ans =    0.5000
ans =    0.7071
ans =    0.8660
\end{lstlisting}
\item Matrix als {\it Ausdruck}
\begin{lstlisting}
>> for x=eye(3),  x' ,end
ans =     1     0     0
ans =     0     1     0
ans =     0     0     1
\end{lstlisting}
\end{itemize}
\end{frame}
\begin{frame}[fragile]\frametitle{Vandermonde-Matrix }
Berechne zu einem gegebenen Vektor
  $x=(x_1, \dots ,x_n)$ die Vandermonde-Matrix
{ \[ V:= \left(\begin{array}{ccccc} 
1 & x_1 & x_1^2 & \hdots & x_1^{n-1}\\
1 & x_2 & x_2^2 & \hdots & x_2^{n-1}\\
\vdots & \vdots & \vdots & \vdots & \vdots\\
1 & x_n & x_n^2 & \hdots & x_n^{n-1}\\
\end{array} \right).  \]}
\end{frame}
%
%
%
\begin{frame}[fragile]\frametitle{Implementierung II}
\begin{lstlisting}
function V=vandermonde2(x)
%----------------------------------------------------
% vandermonde2 berechnet die Vandermonde Matrix zu einem
%              Vektor x
%             INPUT:             x Zeilenvektor 
%             OUTPUT:            V Vandermonde-Matrix
%  Gerd Rapin      8.11.2003
%------------------------------------------------------
n=length(x);
V=zeros(n,n);
for i=1:n
    for j=1:n
       V(i,j)=x(i)^(j-1);
   end
end
\end{lstlisting}
\alert{Bem.:} Die vektorielle Variante (\"UA) ist  4-mal  so schnell.
\end{frame}
%
%
%
\begin{frame}[fragile]\frametitle{Quadratische Gleichung}
\alert{ \[  \left\{ \begin{array}{l} \mbox{Suche }  x \in \mathbb{R},
 \mbox{ so dass } \\
 x^2+px +q =0  \end{array} \right. \]}
Fallunterscheidung für $d:=\frac{p^2}{4} -q$:
\begin{itemize}
\item  Fall a): \alert{ $d>0$} \quad 2 Lösungen: $x=-\frac{p}{2} \pm \sqrt{d}$ \\
\item  Fall b): \alert{ $d=0$} \quad 1 Lösung: $x=-\frac{p}{2}$\\
\item  Fall c): \alert{ $d<0$} \quad keine Lösung
\end{itemize}
\end{frame} 
%
%
%
\begin{frame}[fragile]\frametitle{Implementierung}
\begin{lstlisting}
function [anz_loesungen, loesungen]=quad_gl(p,q)
%----------------------------------------------------
% quad_gl berechnet die Loesungen der quadratischen   
%         Gleichung x^2 + px + q =0
%           INPUT:   Skalare   p
%                              q
%                 
%           OUTPUT: anz_loesungen   Anzahl der Loesungen
%                   loesungen       Vektor der Loesungen
%
%  Gerd Rapin      8.11.2003
%------------------------------------------------------
d=p^2/4-q; % Diskriminante


\end{lstlisting}
\end{frame}
%
%
%
\begin{frame}[fragile]\frametitle{Implementierung II}
\begin{lstlisting}
% 2 Loesungen
if d>0 
    anz_loesungen=2;
    loesungen=[-p/2-sqrt(d) -p/2+sqrt(d)];
end

% 1 Loesung
if d==0 
    anz_loesungen=1;
    loesungen=[-p/2];
end

% 0 Loesungen
if d<0 
    anz_loesungen=0;
    loesungen=[];
end
\end{lstlisting}
\end{frame}
%
%
%
\begin{frame}[fragile]\frametitle{Logische Operationen}
\begin{itemize}
\item Es gibt in MATLAB logische Variablen. Der Datentyp ist {\it
  logical}. 
\item Variablen dieses Typs sind entweder \lstinline!TRUE! (1) oder
  \lstinline!FALSE! (0).
\item Numerische Werte ungleich $0$ werden als \lstinline!TRUE! gewertet.
\end{itemize}
\begin{lstlisting}
>> a= (1<2)
a = 1
>> b= ([ 1 2 3 ] < [ 2 2 2 ])
b =   1     0     0
>> whos
  Name Size Bytes  Class
  a     1x1  1  logical array
  b     1x3  3  logical array
\end{lstlisting}
\end{frame}
%
%
%
\begin{frame}[fragile]\frametitle{Vergleichs-Operatoren}
\begin{lstlisting} 
>> a=[1 1 1], b=[0 1 2] 
\end{lstlisting}
\begin{tabular}{cll}
a == b & gleich &   \lstinline!0     1     0!\\
a \lstinline!~=! b & ungleich & \lstinline!1     0     1!\\
a < b & kleiner & \lstinline!0     0     1!\\
a > b & größer & \lstinline!1     0     0!\\
a <= b & kleiner oder gleich & \lstinline!0     1     1!\\
a >= b & größer oder gleich & \lstinline!1     1     0!\\
\end{tabular}
\\
\alert{Bem:} \alert{ 1 = wahre Aussage, 0 = falsche Aussage}\\
\alert{Bem:} Komponentenweise Vergleiche sind auch für Matrizen
gleicher Größe möglich! 
\end{frame}
%
%
%
\begin{frame}[fragile]\frametitle{Logische Operatoren}
\begin{tabular}{||c|l||c|l||}
\hline
\lstinline!&! & logisches und & \lstinline!~! & logisches nicht \\
\lstinline!|! & logisches oder & \lstinline!xor! & exklusives oder\\
\hline
\end{tabular}
\\
Beispiele:\\
\begin{lstlisting} 
>>  x=[-1 1 1]; y=[1 2 -3];
\end{lstlisting}
\vspace*{0.5cm}
\begin{minipage}{5cm}
\begin{lstlisting}
>> (x>0) & (y>0)
ans =
     0     1     0
\end{lstlisting}
\vspace*{0.5cm}
\begin{lstlisting}
>> (x>0) | (y>0)
ans =
     1     1     1
\end{lstlisting}
\end{minipage} \hfill
\begin{minipage}{5cm}
\begin{lstlisting}
>> ~( (x>0) & (y>0))
ans =
     1     0     1
\end{lstlisting}
\vspace*{0.5cm}
\begin{lstlisting}
>> xor(x>0,y>0)
ans =
     1     0     1
\end{lstlisting}
\end{minipage}
\end{frame}
%
%
%
\begin{frame}[fragile]\frametitle{Bedingung}
\begin{minipage}{5cm}
Einfache Bedingung
\begin{lstlisting}
if  Ausdruck
   Befehle
end
\end{lstlisting}
\end{minipage} \hfill 
\begin{minipage}{5.5cm}
Bed. mit Alternative
\begin{lstlisting}
if  Ausdruck
   Befehle
else
   Befehle
end
\end{lstlisting}
\end{minipage}
Die Befehle zwischen \lstinline!if! und \lstinline!end! werden ausgeführt, wenn
der {\it Ausdruck} wahr (\lstinline!TRUE!) ist. 
Andernfalls werden (soweit vorhanden) die
Befehle zwischen \lstinline!else! und \lstinline!end! ausgeführt.

{\it Ausdruck} ist wahr, wenn   alle Einträge von {\it
  Ausdruck} ungleich $0$ sind.
\end{frame}
%
%
%
\begin{frame}[fragile]\frametitle{While-Schleifen}
\begin{lstlisting}
while Ausdruck
   Befehle
end
\end{lstlisting}
Die Befehle werden wiederholt,  so lange die Bedingung {\it Ausdruck}
wahr ist.  {\it Ausdruck} ist wahr, wenn   alle Einträge von {\it
  Ausdruck} ungleich $0$ sind. \\[1cm] 

Beispiel: Berechne \alert{ $\sum_{i=1}^{1000} \frac{1}{i}$}.
\begin{lstlisting}
>> n=1000; sum=0; i=1; 
>> while (i<=n); sum=sum+(1/i); i=i+1;  end, sum
sum =    7.4855
\end{lstlisting}

\end{frame}
%
%
%
\begin{frame}[fragile]\frametitle{Größter gemeins. Teiler (ggT)}
Berechnung des ggT von natürlichen Zahlen $a$ und $b$ mit Hilfe des
euklidischen Algorithmus\\[1cm]

\alert{Idee:} Es gilt \alert{ $ggT(a,b)=ggT(a,b-a)$} für $a<b$.\\[1cm]

\alert{Algorithmus:} \\
Wiederhole,  bis $a=b$
\begin{itemize}
\item Ist $a>b$, so $a=a-b$.
\item Ist $a<b$, so $b=b-a$ 
\end{itemize}
\end{frame}

\begin{frame}[fragile]\frametitle{Implementierung}
\begin{lstlisting}
function a=ggt(a,b)
%----------------------------------------------------
% ggt berechnet den grten gemeinsamen Teiler (ggT)    
%         zweier natrlichen Zahlen a und b
%           INPUT:   naturliche Zahlen  a
%                                       b
%                 
%           OUTPUT: ggT
%                   
%  Gerd Rapin      8.11.2003
%------------------------------------------------------
while (a~=b)
    if (a>b)
        a=a-b;
    else
        b=b-a;
    end
end
\end{lstlisting}
\end{frame}
%
%
%
\begin{frame}[fragile]\frametitle{{\it break} and {\it continue}}
\begin{itemize}
\item  Der Befehl \lstinline!break! verläßt die \lstinline!while! oder
  \lstinline!for!-Schleife.
\begin{lstlisting}
x=1; while 1, xmin=x; x=x/2;
 if x==0, break, end,
end, xmin

xmin = 4.9407e-324
\end{lstlisting} 
\vspace*{-0.5cm}
\item  Durch \lstinline!continue! springt man sofort in die
  nächste Iteration der Schleife, ohne die restlichen Befehle zu
  durchlaufen.
\vspace*{-0.5cm}\\
\begin{lstlisting}
for i=1:10,
 if i<5, continue, end,
 x(i)=i; end, x

                 x =  0  0  0  0  5  6  7  8  9  10
\end{lstlisting}
\end{itemize}
\end{frame}

%
% Slide
% 
\begin{frame}[fragile]\frametitle{Rekursive Funktionen}
Rekursive Funktionen sind Funktionen, die sich selbst aufrufen.\\
Bei jedem Aufruf wird ein neuer lokaler Workspace erzeugt.\\[1cm]

Beispiel Fakult"at: $n!=fak(n)$\\
\begin{eqnarray*}
 n!& = & n(n-1)!=nfak(n-1)\\
& = & n(n-1)fak(n-2)\\
& = & \cdots= n(n-1)\cdots 1 
\end{eqnarray*}
\end{frame}
%
% Slide
%
\begin{frame}[fragile]\frametitle{Fakult"at-rekursiv}
\begin{lstlisting}
function res=fak(n)

if (n==1)
    res=1;
else 
    res=n*fak(n-1);
end
\end{lstlisting}
\end{frame}
%
% Slide
%
\begin{frame}[fragile]\frametitle{Fakult"at - direkt}
\begin{lstlisting}
function res=fak_it(n)

res=1;
for i=1:n
    res=res*i;
end
\end{lstlisting}
\end{frame}
%
% Slide
%
\begin{frame}[fragile]\frametitle{Fakult"at - Zeitvergleich}
\begin{lstlisting}
\begin{verbatim}
% fak_vergleich.m

% iterativ
tic
for i=1:100
    fak_it(20);
end
time1=toc;
fprintf('\nZeitverbrauch direktes Verfahren: %f',time1);

% rekursiv
tic
for i=1:100
    fak(20);
end
time2=toc;
fprintf('\nZeitverbrauch rekursives  Verfahren: %f',time2);
\end{verbatim}
\end{lstlisting}
\end{frame}
%
% Slide
%
\begin{frame}[fragile]\frametitle{rekursive Implementierung}

\begin{lstlisting}
\begin{verbatim}
function [a,b] = ggt_rekursiv(a,b)
% ggt_rekursiv berechnet den goessten 
% gemeinsamen Teiler (ggT) 
if a~=b
    if a>b
        a = a-b;
    else
        b = b-a;
    end;
        [a,b] = ggt_rekursiv(a,b);
end;
\end{verbatim}
\end{lstlisting}
\end{frame}
%
% Slide
% 
\begin{frame}[fragile]\frametitle{Sierpinski Dreieck}
\begin{itemize}
\item Wir beginnen mit einem Dreieck mit Eckpunkten $P_a$, $P_b$ und $P_c$. 
\item Wir entfernen daraus das Dreieck, das durch die Mittelpunkte der
  Kanten entsteht.
\item Die verbliebenden drei Dreiecke werden der gleichen Prozedur
  unterzogen.
\item Diesen Prozess können wir rekursiv wiederholen.
\item Das Ergebnis ist das Sierpinski Dreieck.
\end{itemize}
\end{frame}
%
% Slide
% 
\begin{frame}[fragile]\frametitle{Sierpinski Dreieck}
\begin{minipage}{5cm}
\includegraphics[width=5cm,
  height=3cm]{./figures/sierpinski_0}
\end{minipage} \hfill
\begin{minipage}{5cm}
\includegraphics[width=5cm,
  height=3cm]{./figures/sierpinski_1}
\end{minipage}\\ 
\begin{minipage}{5cm}
\includegraphics[width=5cm,
  height=3cm]{./figures/sierpinski_2}
\end{minipage} \hfill
\begin{minipage}{5cm}
\includegraphics[width=5cm,
  height=3cm]{./figures/sierpinski_3}
\end{minipage} \\
\end{frame}
%
% Slide
%
\begin{frame}[fragile]\frametitle{Implementierung}
\begin{lstlisting}
%  sierpinski_plot.m
level=3;

ecke1=[0;0];
ecke2=[1;0];
ecke3=[0.5; sqrt(3)/2];

figure; axis equal;
hold on;
sierpinski(ecke1,ecke2,ecke3,level);
hold off;
title(['Sierpinski Dreieck, Level =' ...
        num2str(level)],'FontSize',16);
\end{lstlisting}
\end{frame}
%
% Slide
%
\begin{frame}[fragile]\frametitle{Implementierung}
\begin{lstlisting}
function sierpinski(ecke1,ecke2,ecke3,level)
% Teilt das Dreieck auf in 3 Dreiecke (level>0)
% Plotten des Dreiecks (level=0)

if level ==0 
    fill([ecke1(1),ecke2(1),ecke3(1)],...
         [ecke1(2),ecke2(2),ecke3(2)],'r');
else
    ecke12=(ecke1+ecke2)/2;
    ecke13=(ecke1+ecke3)/2;
    ecke23=(ecke2+ecke3)/2;
    sierpinski(ecke1,ecke12,ecke13,level-1);
    sierpinski(ecke12,ecke2,ecke23,level-1);
    sierpinski(ecke13,ecke23,ecke3,level-1);
    
end;
\end{lstlisting}
\end{frame}
% 
% Slide
% 
\begin{frame}[fragile]\frametitle{Zeichnen von Polygonen}

Ein Polygon sei durch die Eckpunkte $(x_i,y_i)_{i=1}^n$ gegeben. Dann
kann er in MATLAB durch den Befehl
\begin{lstlisting}
fill(x,y,char)
\end{lstlisting}
dargestellt werden. \lstinline!char! gibt die Farbe des Polygons an, z.B. rot
wäre 'r'.
\end{frame}
%
% Slide
%
\begin{frame}[fragile]\frametitle{Warnung}
\begin{center}
\alert{ Wiederholte Anwendung von Script-Files kann zu Fehlern führen}\\[0.7cm]
\end{center}
\begin{minipage}{5cm}
\alert{ Programm}
\begin{lstlisting}
% plotte_sin.m

disp(['Plot der Sinus'...
  'Funktion auf [0,10]']);
n = input(['Plot an '...
   'wievielen Punkten?']);
x = linspace(0,10,n);
for i=1:n
y(i) = sin(x(i));
end; 
plot(x,y);
\end{lstlisting}
\end{minipage} \hfill
\begin{minipage}{5cm}
\alert{ Aufruf}
\begin{lstlisting}
>> plotte_sin
Plot der Sinus Funktion auf [0,10]
Plot an wievielen Punkten?20
>> plotte_sin
Plot der Sinus Funktion auf [0,10]
Plot an wievielen Punkten?10
\alert{ ??? Error using ==> plot
Vectors must be the same lengths.

Error in ==> /slides/m-files/
  stunde_13/plotte_sin.m
On line 9  ==> plot(x,y);}
\end{lstlisting}
\end{minipage} \\
\end{frame}
%
% Slide
%
\begin{frame}[fragile]\frametitle{globale Variablen}
Mittels des Befehls \alert{ \lstinline!global!} können Variablen des
globalen Workspace auch für
Funktionen manipulierbar gemacht werden.\\[0.6cm]
\begin{minipage}{5cm}
\alert{ Funktion}
\begin{lstlisting}
function f=myfun(x)
% myfun.m
% f(x)=x^alpha sin(1/x)

global alpha
f=x.^alpha.*sin(1./x);
\end{lstlisting}
\end{minipage} \hfill
\begin{minipage}{5cm}
\alert{ Plotten}
\begin{lstlisting}
% plot_myfun
global alpha
alpha_w=[0.4 0. 6 1 1.5 2];
for i = 1:length(alpha_w);
    alpha = alpha_w(i);
    fplot(@myfun,[0.1,1]);
    hold on;
end
hold off;
\end{lstlisting}
\end{minipage} \\
\end{frame}
%
% Slide
%
\begin{frame}[fragile]\frametitle{Guter Stil in MATLAB}
\begin{itemize}
\item Alle Programme sollten zu Beginn einen Kommentar enthalten, in
  dem beschrieben wird, was das Programm macht. Insbesondere sollten
  die Eingabe- und Ausgabevariablen  genau beschrieben
  werden. 
\item Vor und nach logischen Operatoren und $=$ sollte ein Lehrzeichen
  gesetzt werden.
\item Man sollte pro Zeile nur einen Befehl verwenden.
\item Befehle in  Strukturen, wie  \lstinline!if!, \lstinline!for!
  oder \lstinline!while!, sollten eingerückt werden. 
%\item Variablen Namen für Matrizen sollten mit einem Großbuchstaben
 % beginnen. 
\end{itemize}
\end{frame}
%
% Slide
%
\begin{frame}[fragile]\frametitle{Guter Stil in MATLAB}
\begin{itemize}
\item Die Namen der Variablen sollten, soweit möglich, selbsterklärend
  sein.
\item Verfasst man umfangreiche Programme, so sollten M-Funktionen, die
  eine logische Einheit bilden in einem separaten Unterverzeichnis
  gespeichert sein. Die Verzeichnisse können durch \lstinline!addpath!
  eingebunden werden. 
\item Potentielle Fehler sollten, soweit möglich, aufgefangen
  werden. Speziell sollten die Ein- \\
 gabeparameter der Funktionen
  geprüft werden. 
\end{itemize}
\end{frame}

%
%
%



%%% Local Variables: 
%%% mode: latex
%%% TeX-master: t
%%% End: 

\documentclass[hyperref={xetex}]{beamer}
\title{Einführung in Matlab - Einheit 3}
\subtitle{Rekursionen, Grafik}
\mode<article>
{
  \usepackage{fullpage}
  \usepackage{pgf}
  \usepackage[xetex]{hyperref}
  \setjobnamebeamerversion{beamer}
}

\mode<presentation>
{
  %\usetheme{Frankfurt}
 %\usetheme{My}
  \usetheme{Madrid}
  % or ...
%\usecolortheme{seagull}
  %\setbeamercovered{transparent}
  %\setbeamercovered{dynamic}
  % or whatever (possibly just delete it)
}
\usenavigationsymbolstemplate{}
\usefonttheme{structurebold}
\usepackage{multimedia}
%\usepackage{tikz}
\usepackage{fontspec,xunicode,xltxtra}

%\usepackage{polyglossia}
%\setdefaultlanguage[spelling=new, latesthyphen=true]{german}
%\setsansfont{DejaVu Sans}
%\setsansfont{Verdana}
%\setsansfont{Arial}
%\setromanfont{Linux Libertine O}
%\setsansfont{Linux Biolinum O}

\setbeamertemplate{footline}
{
\leavevmode
%\hbox{\begin{beamercolorbox}[wd=.5\paperwidth,ht=2.5ex,dp=1.125ex,
%leftskip=.3cm plus1fill,rightskip=.3cm]{author in head/foot}%
%    \usebeamerfont{author in head/foot}\insertshortauthor
%  \end{beamercolorbox}%
%  \begin{beamercolorbox}[wd=.5\paperwidth,ht=2.5ex,dp=1.125ex,leftskip=.3cm,
%rightskip=.3cm plus1fil]{title in head/foot}%
%    \usebeamerfont{title in head/foot}\insertshorttitle\hfill

\hfill\insertframenumber  \hspace{3pt}

%\inserttotalframenumber
%\hspace*{2ex}
%  \end{beamercolorbox}}%
  \vskip3pt%
}

%\usepackage[english]{babel}
\usepackage[ngerman]{babel}
\selectlanguage{ngerman}

%
% math/symbols
%
\usepackage{amssymb}
\usepackage{amsthm}
% \usepackage{latexsym}
\usepackage{amsmath}
%\usepackage{listings}
\usepackage[framed]{mcode}
%\usepackage{mcode}

\usepackage{mydef}
%\usepackage{cmap} % you can search in the pdf for umlauts and ligatures
%\usepackage{colonequals} %corrects the definition-symbols \colonequals (besides others)
\title{Einführung in Matlab}
%
%\subtitle{Disputation} % (optional)

\author{Jochen Schulz}
% - Use the \inst{?} command only if the authors have different
%   affiliation.

\institute{Georg-August Universit\"at G\"ottingen \pgfimage[height=0.5cm]{../figures/unilogo3}}
% - Use the \inst command only if there are several affiliations.
% - Keep it simple, no one is interested in your street address.

\date{\today}

\subject{Einführung in Matlab}
% This is only inserted into the PDF information catalog. Can be left
% out. 



% If you have a file called "university-logo-filename.xxx", where xxx
% is a graphic format that can be processed by latex or pdflatex,
% resp., then you can add a logo as follows:

%\logo{\pgfimage[height=0.5cm]{figures/unilogo3}}


% Delete this, if you do not want the table of contents to pop up at
% the beginning of each subsection:
% \AtBeginSubsection[]
% {
%   \begin{frame}<beamer>
%     \frametitle{Aufbau}
%     \tableofcontents[currentsection,currentsubsection]
%   \end{frame}
% }

\AtBeginSection[]
{
  \begin{frame}<beamer>
    \frametitle{Aufbau}
    \tableofcontents[currentsection,currentsubsection]
  \end{frame}
}

\AtBeginSubsection[]
{
  \begin{frame}<beamer>
    \frametitle{Aufbau}
    \tableofcontents[currentsection,currentsubsection]
  \end{frame}
}

\begin{document}
\lstset{}


\begin{document}
\titlepage


\section{Rekursionen}
%
% Slide
% 
\begin{frame}[fragile]\frametitle{Rekursive Funktionen}
Rekursive Funktionen sind Funktionen, die sich selbst aufrufen.\\
Bei jedem Aufruf wird ein neuer lokaler Workspace erzeugt.\\[1cm]

\textbf{Beispiel:} Fakult"at: $n!=\fak(n)$\\
\begin{eqnarray*}
 n!& = & n(n-1)!=n \fak(n-1)\\
& = & n(n-1)\fak(n-2)\\
& = & \cdots= n(n-1)\cdots 1 
\end{eqnarray*}
\end{frame}
%
% Slide
%
\begin{frame}[fragile]\frametitle{Fakult"at - rekursiv}
\lstinputlisting{fak.m}
\end{frame}
%
% Slide
%
\begin{frame}[fragile]\frametitle{Fakult"at - direkt}
\lstinputlisting{fak_it.m}
\end{frame}
%
% Slide
%
\begin{frame}[fragile]\frametitle{Fakult"at - Zeitvergleich}
\lstinputlisting{fak_vergleich.m}
\end{frame}
%
% Slide
%
\begin{frame}[fragile]\frametitle{rekursive Implementierung GGT}
\lstinputlisting{ggt_rekursiv.m}
\end{frame}
%
% Slide
% 
\begin{frame}[fragile]\frametitle{Sierpinski Dreieck}
\begin{itemize}
\item Wir beginnen mit einem Dreieck mit Eckpunkten $P_a$, $P_b$ und $P_c$. 
\item Wir entfernen daraus das Dreieck, das durch die Mittelpunkte der
  Kanten entsteht.
\item Die verbliebenden drei Dreiecke werden der gleichen Prozedur
  unterzogen.
\item Diesen Prozess können wir rekursiv wiederholen.
\item Das Ergebnis ist das Sierpinski Dreieck.
\end{itemize}
\end{frame}
%
% Slide
% 
\begin{frame}[fragile]\frametitle{Sierpinski Dreieck}
\begin{minipage}{5cm}
\includegraphics[height=4cm]{figures/sierpinski_0}
\end{minipage} \hfill
\begin{minipage}{5cm}
\includegraphics[height=4cm]{figures/sierpinski_1}
\end{minipage}\\ 
\begin{minipage}{5cm}
\includegraphics[height=4cm]{figures/sierpinski_2}
\end{minipage} \hfill
\begin{minipage}{5cm}
\includegraphics[height=4cm]{figures/sierpinski_3}
\end{minipage} \\
\end{frame}
%
% Slide
%
\begin{frame}[fragile]\frametitle{Implementierung}
\lstinputlisting{sierpinski_plot.m}
\end{frame}
%
% Slide
%
\begin{frame}[fragile]\frametitle{Implementierung}
\lstinputlisting{sierpinski.m}
\end{frame}
% 
% Slide
% 
\begin{frame}[fragile]\frametitle{Zeichnen von Polygonen}

Ein Polygon sei durch die Eckpunkte $(x_i,y_i)_{i=1}^n$ gegeben. Dann
kann durch den Befehl
\begin{lstlisting}
fill(x,y,char)
\end{lstlisting}
dargestellt werden. \mcode{char} gibt die Farbe des Polygons an, z.B. rot
wäre \mcode{'r'}.
\end{frame}


\section{Einf\"uhrung Grafik}
\subsection{einfache zweidimensionale Grafiken}
% 
% Slide
% 
\begin{frame}[fragile]\frametitle{Standard-Plot}
\begin{lstlisting}
plot(<x>,<y>)
\end{lstlisting}
zeichnet für Vektoren $x=(x_1, \ \dots \ ,x_N)$ und  $y=(y_1, \dots \ ,y_N)$
eine Grafik, die die Punkte $(x_i,y_i)$ und $(x_{i+1},y_{i+1})$ miteinander
verbindet.

\begin{columns}[c]
 \column{0.45\textwidth}
\textit{Beispiel:}
\begin{lstlisting}
x = linspace(0,2*pi,100);
y1 = sin(3*x);
plot(x,y1)
\end{lstlisting}
 \column{0.5\textwidth}
\pgfimage[width=\textwidth]{figures/grafik_1}
\end{columns}
\end{frame}
% 
% Slide
% 
\begin{frame}[fragile]\frametitle{Erweiterungen}
\begin{lstlisting}
plot(<x>,<y>,<string>)
\end{lstlisting}
\alert{String} besteht aus drei Elementen, die die Farbe, Linienstil
und die Markierung der Punkte kontrollieren. Die Reihenfolge der drei
Elemente ist beliebig.
\begin{columns}[c]
 \column{0.4\textwidth}
\textit{Beispiel:} Durch \\
\begin{lstlisting}
plot(x,y,'r*--') 
\end{lstlisting}
wird die Linie
gestrichelt (- -) in rot (r) gezeichnet und die Punkte durch *
markiert.
\column{0.55\textwidth}
\includegraphics[width=\textwidth]{figures/grafik_2}
\end{columns}
\end{frame}
% 
% Slide
% 
\begin{frame}[fragile]\frametitle{Optionen}
\begin{tabular}{cp{8.5cm}}
\alert{ Farben} & r (rot), g (grün), b (blau), c (hellblau), m (magenta),
  y (gelb), k (schwarz), w (weiß)\\
\alert{ Marker} & o (Kreis), * (Stern), . (Punkt), + (Plus), x (Kreuz), s
  (Quadrat), d (Raute),... \\
\alert{ Linien-Stil} &  - (durchgezogene Linie), \mcode{--} (gestrichelte
  Linie), \mcode{:} (gepunktete Linie), \mcode{-.} (Strich-Punkt Linie)\\
\end{tabular}

Läßt man den Linien-Stil weg, so werden die Punkte nicht verbunden.
\end{frame}

% 
% Slide
% 
\begin{frame}[fragile]\frametitle{Optionen II}
\begin{lstlisting}
plot(<x>,<y>,<string>,<Eigenschaft>, <Spez.>) 
\end{lstlisting}
\alert{ Eigenschaften:}\\
\mcode{'MarkerSize'} (Default 6), \mcode{'LineWidth'} (Default 0.5),
\mcode{'MarkerEdgeColor'}, \mcode{'MarkerFaceColor'}\\

\begin{columns}[c]
 \column{0.6\textwidth}
\textit{Beispiel:} \\
\begin{lstlisting}
plot(x,y1,'b-.d','LineWidth',...
3,'MarkerEdgeColor','g')
\end{lstlisting}
\column{0.4\textwidth}
\includegraphics[width=\textwidth]{figures/grafik_3}
\end{columns}
\end{frame}
% 
% Slide
% 
\begin{frame}[fragile]\frametitle{Alternativen}
\begin{itemize}
\item Mehrere Plots in eine Grafik:
\begin{lstlisting}
plot(x1,y1,string1,x2,y2,string2,...) 
\end{lstlisting}
\item Logaritmische Skalierung in $x$- bzw in $y$-Richtung:
\begin{lstlisting}
semilogx(x1,y1) 
\end{lstlisting}
  bzw. 
\begin{lstlisting}
semilogy(x1,y1) 
\end{lstlisting}
\item Logarithmische Skalierung beider Achsen: 
\begin{lstlisting}
loglog(x1,y1)
\end{lstlisting}
\item Ist $X$ ein Vektor mit komplexen Einträgen, so ergibt \alert{
  \mcode{plot(X)}} 
\begin{lstlisting}
plot(real(X),imag(X))
\end{lstlisting}
\end{itemize}
\end{frame}
% 
% Slide
% 
\begin{frame}[fragile]\frametitle{Beispiel - Legendre Polynome}
%\begin{columns}[c]
% \column{0.52\textwidth}
\begin{lstlisting}[basicstyle=\scriptsize]
x = linspace(-1,1,100);
p1 = x;
p2 = (3/2)*x.^2-1/2;
p3 = (5/2)*x.^3-(3/2)*x;
p4 = (35/8)*x.^4 - (15/4)*x.^2+3/8;
plot(x,p1,'r:',x,p2,'g--',x,p3,'b-.',x,p4,'m-','LineWidth',2)
\end{lstlisting}
%\column{0.48\textwidth}
\hfil\includegraphics[width=0.55\textwidth]{figures/grafik_4}\hfil
%\end{columns}
\end{frame}
% 
% Slide
% 
\begin{frame}[fragile]\frametitle{Achseneinstellungen}
\begin{tabular}{p{4cm}p{7cm}}
\mcode{axis([x1 x2 y1 y2])} & Setzen der $x$- und $y$-Achsen
Grenzen\\
\mcode{axis auto} & Rückkehr zu Default Achsen Grenzen\\
\mcode{axis equal} & Gleiche Dateneinheiten auf allen Achsen\\
\mcode{axis off} & Enfernen der Achsen\\
\mcode{axis square} & quadratische Achsen-Box\\
\mcode{axis tight} & Achsen Grenzen werden passend zu den Daten
gewählt. \\
\mcode{xlim([x1 x2])} & Setzen der $x$-Achse\\
\mcode{ylim([y1 y2])} & Setzen der $y$-Achse\\
\mcode{grid on} & Gitter aktivieren\\
\mcode{box on}, \mcode{box off}  & Box um die Grafik
legen, Box entfernen
\end{tabular}
\end{frame}


\subsection{Beschriftungen}
% 
% Slide
% 
\begin{frame}[fragile]\frametitle{Beschriften der Grafik}
\begin{itemize}
\item Titel:  \alert{ \mcode{title('Titel')}}
\item Achsenbeschriftung:  \alert{ \mcode{xlabel('Text')}},
  \alert{ \mcode{ylabel('Text')}} 
\item Legende: \alert{ \mcode{legend('Text1','Text2',...,nr)}} \\
{\scriptsize \alert{ nr} gibt die Position der Legendenbox in der Grafik an:
  -1 (rechts vom Plot), 0 'bester' Ort, 1 oben rechts (default), 2
  oben links, 3 unten links, 4 unten rechts. }
\item zusätzlicher Text: \alert{ \mcode{text(x,y,'Text')}} Plaziert
  'Text' an die Position $(x,y)$ bzgl. der Werte auf der $x$-
  bzw. $y$-Achse. 
\end{itemize}
\end{frame}
% 
% Slide
% 
\begin{frame}[fragile]\frametitle{Bemerkungen zur Beschriftung}
\begin{itemize}
\item  In den strings kann direkt eine abgespeckte \LaTeX-Notation verwendet werden. (nahezu vollständige 
\LaTeX-Unterstützung: latex-interpreter).
Beispiele: 
\begin{itemize} \item \mcode{\\alpha} $\Rightarrow$ $\alpha$
\item \mcode{sin^\{3/2\}(x)} $\Rightarrow$ $\sin^{3/2}(x)$ .
\item \mcode{title('f(x) = \\frac\{1\}\{x^2+a\}','interpreter','latex')}
$\Rightarrow$ $f(x) = \frac{1}{x^2+a}$
\end{itemize}
\item Ändern der Schriftgröße, z.B. \mcode{title('Titel','FontSize', 20)}.
\item Auflistung aller modifizierbaren Texteigenschaften: \mcode{doc text_props}
\end{itemize}
\end{frame}
% 
% Slide
% 
\begin{frame}[fragile]\frametitle{Beispiel - Legendre Polynome II}
\begin{lstlisting}[basicstyle=\scriptsize]
title('Legendre Polynome','FontSize', 20)
xlabel('x','FontSize', 20)
text(0,0.45,'Maximum')
legend('n=1','n=2','n=3','n=4',4)
grid on, box on;
xlim([-1.1,1.1])
\end{lstlisting}
\hfil\includegraphics[width=0.5\textwidth]{figures/grafik_5}\hfil
%\end{columns}
\end{frame}
% 
% Slide
% 
\begin{frame}[fragile]\frametitle{Umgang mit Grafikfenster}
\begin{itemize}
\item Öffnen eines (weiteren) Grafikfensters: \alert{\mcode{figure}}.  Eine
  Grafik wird immer im aktuellen Fenster erzeugt. Ist noch kein
  Fenster geöffnet, so wird ein Fenster erzeugt.
\item Durch den Befehl \alert{\mcode{hold on}} werden bestehende Grafiken im
  aktuellen Fenster erhalten. Neue Grafiken werden den bestehenden
  hinzugefügt. 
\item \alert{\mcode{hold off}} (default) überschreibt Grafiken im
  aktuellen Fenster
\item Schliessen: \alert{\mcode{close}}, \alert{\mcode{close all}}
\end{itemize}
\end{frame}


\subsection{Weitere zweidimensionale Darstellungsmöglichkeiten}

% 
% Slide
% 
\begin{frame}[fragile]\frametitle{Darstellung von Daten}
\begin{itemize}
\item Balkendiagramm:
\begin{lstlisting}
bar(<Daten>) 
\end{lstlisting}
\item Histogramm: 
\begin{lstlisting}
hist(<Daten>,<Anzahl Bars>)
\end{lstlisting}
\item einfacher ausgefüllter Plot: 
\begin{lstlisting}
area(<x>,[<y1>,<y2>])
\end{lstlisting}
(y1 und y2 werden addiert)
\item Tortengrafik: 
\begin{lstlisting}
pie3([<anteil1> <anteil2> .. <anteilx>])
\end{lstlisting}
\end{itemize}
\end{frame}
% 
% Slide
% 
\begin{frame}[fragile]\frametitle{Darstellung von Daten}
\begin{center}\includegraphics[height=0.8\textheight]{figures/darstellung_daten_2d}\end{center}
\end{frame}
% 
% Slide
% 
\begin{frame}[fragile]\frametitle{Darstellung von Daten}
\begin{lstlisting}
n = linspace(0,10,40);
y = n.^2.*exp(-n);

% Balkendiagramm
subplot(2,2,1),
bar(y); title('Balkendiagramm');

% Histogramm
subplot(2,2,2),
hist(y,5); title('Histogramm');

% Area plot
subplot(2,2,3),
area(n,[y',2*y']); title('Area plot');

% Tortengrafik
subplot(2,2,4),
pie3([ 1 2 3 4]); title('Tortengrafik');
\end{lstlisting}
\end{frame}
% 
% Slide
% 
\begin{frame}[fragile]\frametitle{Approximation von Integralen}
Approximiere $\int_0^1 f(x) dx$ durch (Mittelpunktsregel)
{\scriptsize \[ \int_0^1 f(x) dx \approx  \sum_{i=1}^{N} \frac{1}{N} f \left(
\frac{i-\frac{1}{2}}{N} \right ) \]}
für gegebenes $N \in \mathbb{N}$. \textbf{Beispiel}: $f(x)=x^3$
\begin{columns}[t]
\column{0.5\textwidth}
\includegraphics[width=\textwidth]{figures/integral_N=10} 
\column{0.5\textwidth}
\includegraphics[width=\textwidth]{figures/integral_N=20}
\end{columns}
\end{frame}
% 
% Slide
% 
\begin{frame}[fragile]\frametitle{Integral - Implementation}
\lstinputlisting{integral.m}
\end{frame}

\subsection{Dreidimensionale Grafiken}

% 
% Slide
% 
% 
\begin{frame}[fragile]\frametitle{Dreidimensionale Grafiken}
\begin{itemize}
\item Dreidimensionale Version von \mcode{plot}: \alert{ \mcode{plot3}}
\item Darstellung von Funktionen $f:\mathbb{R}^2 \ \rightarrow \
  \mathbb{R}$:
\begin{itemize}
\item Contourplot (zeichnet die Niveaulinien): \alert{ \mcode{contour}}, \alert{
    \mcode{contourf}}, \alert{ \mcode{contour3}}
\item Darstellung des Graphen mit Gitterlinien: \alert{ \mcode{mesh} ,
  \mcode{meshc}} 
\item Flächige Darstellung des Graphen: \alert{ \mcode{surf}, \mcode{surfc}}
\end{itemize} 
\item Darstellung von Funktionen $f:\mathbb{R}^3 \ \rightarrow \
  \mathbb{R}$:
\begin{itemize}
\item Streifenansichten \alert{ \mcode{slice}}
\end{itemize}
\end{itemize}
\alert{ \mcode{mesh(X,Y,Z)}} z.B. stellt für Matrizen $X,Y,Z \in
\mathbb{R}^{n \times k}$ die Punkte 
\[\alert{  (X(i,j), Y(i,j), Z(i,j))} \quad \mbox{dar.}\]
\end{frame}

% 
% Slide
% 
\begin{frame}[fragile]\frametitle{plot3}
Bei gegebenen Vektoren $x=(x_i)_{i=1}^n$, $y=(y_i)_{i=1}^n$,
$z=(z_i)_{i=1}^n$ erzeugt \alert{ \mcode{plot3(x,y,z)}} einen Plot der die Punkte
$(x_i,y_i,z_i)$ und $(x_{i+1},y_{i+1},z_{i+1})$ miteinander
verbindet. \\
\begin{center}\includegraphics[width=0.6\textwidth]{figures/beispiel_plot3}\end{center}
\end{frame}
% 
% Slide
% 
\begin{frame}[fragile]\frametitle{Beispiel plot3}
\begin{lstlisting}
t = 0:0.1:20*pi;
x = exp(-t/20).*sin(t);
y = exp(-t/20).*cos(t);
z = t;

plot3(x,y,z,'b-o','LineWidth',1);
grid on
xlabel('x(t)'), ylabel('y(t)');
 zlabel('z(t)');
title('Beispiel: plot3','FontSize',15);
\end{lstlisting}
\end{frame}
% 
% Slide
% 
\begin{frame}[fragile]\frametitle{Blickwinkel}
\centering\alert{ \mcode{view(az,el)}}
\begin{itemize}
\item \alert{ \mcode{az}} ist die horiz. Rotation in Grad (Def. \alert{
  $-37.5$}) 
\item \alert{ \mcode{el}} ist die vertikale Rotation in Grad (Def. \alert{
  $30$})
\end{itemize}
\begin{center}\includegraphics[width=0.6\textwidth]{figures/beispiel_plot3_2}\end{center}
\end{frame}
% 
% Slide
% 
\begin{frame}[fragile]\frametitle{3D-Funktionenplots}

Darstellung von Funktionen
\[ f: \mathbb{R}^2 \quad  \rightarrow \quad \mathbb{R} \]
\hspace*{1cm}\\

\textbf{Beispiel:}\\
\alert{ \[ f(x,y):=\exp(-x^2-y^2)\sin(\pi x y) \]}
\end{frame}
% 
% Slide
% 
\begin{frame}[fragile]\frametitle{Beispiel: Funktionenplot}
\hfil\includegraphics[width=0.8\textwidth]{figures/beispiel_function_plot_3d}\hfil
\end{frame}
% 
% Slide
% 
\begin{frame}[fragile]\frametitle{Funktionenplot - Implementation}
\begin{lstlisting}
% Erzeugen des Gitters
x = linspace(-2,2,30);
y = linspace(-2,2,30);
[X,Y] = meshgrid(x,y);
% Funktionswerte
Z = exp(-X.^2-Y.^2).*sin(pi*X.*Y);

% verschiedenen Darstellungen
subplot(2,2,1),
 mesh(X,Y,Z), title('mesh');
subplot(2,2,2),
 surf(X,Y,Z), title('surf');
subplot(2,2,3),
 contour(X,Y,Z,10), title('contour');
subplot(2,2,4),
 surfc(X,Y,Z);
 view(-26,20), title('surfc');
\end{lstlisting}
\end{frame}
% 
% Slide
% 
\begin{frame}[fragile]\frametitle{subplot}
\begin{lstlisting}
subplot(<n>,<m>,<k>)
\end{lstlisting}
zerlegt das Grafikfenster in $n \times m$ Teilfenster. 

Die Zahl $1
\leq k \leq nm$ gibt an, welches Teilfenster gerade aktiv
ist. \\

Durchnumeriert wird zeilenweise, also $(1,1), (1,2), \dots$.

\end{frame}

% 
% Slide
% 
\begin{frame}[fragile]\frametitle{meshgrid}
Zu Vektoren $x=(x_i)_{i=1}^k$, $y=(y_j)_{j=1}^n$ erzeugt 
\begin{lstlisting}
[X,Y]=meshgrid(x,y)
\end{lstlisting}
Matrizen $X,Y \in \mathbb{R}^{n \times k}$, wobei jede Zeile von $X$
eine Kopie des Vektors $x$ ist und $Y$ als Spalten den Vektor $y$
enthält. \\
Dann hat \alert{ \mcode{Z=X.*Y}} die Komponenten 
\[ Z(i,j)=x(j)*y(i). \]
\end{frame}

%\end{frame}
% 
% Slide
% 
\begin{frame}[fragile]\frametitle{Weitere Möglichkeiten}
\begin{itemize}
\item Darstellung versteckter Linien (bei \mcode{mesh}): \alert{ \mcode{hidden off}}, Default:
\alert{ \mcode{hidden on}}
\item Verschmieren des Gitters: \alert{ \mcode{shading('interp')}}
\item Blickwinkel: \alert{ \mcode{view(az,el)}}
\item ähnlich wie \mcode{mesh}; nur mit 'Vorhang': \alert{
  \mcode{meshz(X,Y,Z)}}
\end{itemize}
\end{frame}
% 
% Slide
% 
\begin{frame}[fragile]\frametitle{Beispiel: Funktionenplot}
\hfil\includegraphics[width=0.8\textwidth]{figures/beispiel_function_plot_3d_2}\hfil
\end{frame}
% 
% Slide
% 
\begin{frame}[fragile]\frametitle{Funktionenplot - Listing}
\begin{lstlisting}
x = linspace(-2,2,30);
y = linspace(-2,2,30);
[X,Y] = meshgrid(x,y);
% Funktionswerte
Z = exp(-X.^2-Y.^2).*sin(pi*X.*Y);

% verschiedenen Darstellungen
subplot(2,2,1),
 mesh(X,Y,Z), title('Default');
subplot(2,2,2),
 mesh(X,Y,Z), hidden off,
 title('keine versteckten Linien');
subplot(2,2,3), surf(X,Y,Z);
 shading('interp'), title('Shading');
subplot(2,2,4), Z=X.^2.*Y;
 meshz(X,Y,Z), title('f(x,y)=x^2 y');
\end{lstlisting}
\end{frame}

% 
% Slide
% 
\begin{frame}[fragile]\frametitle{Contour Plots}
\hfil\includegraphics[width=0.8\textwidth]{figures/beispiel_function_plot_contour}\hfil
\end{frame}
% 
% Slide
% 
\begin{frame}[fragile]\frametitle{Contour Plots - Listing}
\lstinputlisting{contour_plot.m}
\end{frame}

% 
% Slide
% 
\begin{frame}[fragile]\frametitle{Erläuterungen zu Contour-Befehlen}
\begin{itemize}
\item \alert{\mcode{contour(X,Y,Z,n)}} zeichnet f\"ur $n\in \mathbb{N}$
  $n$-Konturlinien. Ist $n$ ein Vektor, werden Konturlinien zu den Werten in
  dem Vektor $n$ geplottet.
\item \alert{\mcode{contourf}} funktioniert wie \mcode{contour} nur das die Flächen
  zwischen den Konturlinien ausgefüllt werden.
\item \alert{\mcode{clabel(C,h)}} beschriftet die Konturlinien, deren Werte in $C$
  gespeichert sind und die zum Grafik-Handle $h$ gehören.
\item \alert{\mcode{contour3}} zeichnet jede Konturlinie auf einer anderen H\"ohe.
\end{itemize}
\end{frame}
%
% Slide
% 
\begin{frame}[fragile]\frametitle{Slice}
\begin{lstlisting}
slice(X,Y,Z,V,sx,sy,sz)
\end{lstlisting}
zeichnet  Schnitte zu den Funktionswerten $V(i)$ zu
$(X(i),Y(i),Z(i))$. Schnitte sind durch die Vektoren $sx$, $sy$ und $sz$
gegeben.\\
\textbf{Beispiel}: \alert{ \[ f(x,y,z):=\exp(-x^2-y^2)\sin(\pi x y z) \]}
\lstinputlisting{beispiel_slice.m}
\end{frame}

\begin{frame}[fragile]\frametitle{Beispiel: slice}
\hfil\includegraphics[width=0.8\textwidth]{figures/slice}\hfil
\end{frame}

\subsection{Animation}
% 
% Slide
% 
\begin{frame}[fragile]\frametitle{Animation-Beispiel}
\lstinputlisting{animation.m}
\end{frame}
%
% Slide
% 
\begin{frame}[fragile]\frametitle{Erstellen einer Animation}
\begin{itemize}
\item Mit \alert{\mcode{F(j)=getframe}} wird die aktuelle Grafik in das Array
  $F$ gespeichert.
\item Sequenz der Bilder $F$ darstellen: \alert{\mcode{movie(F,n,fps)}},
  wobei $n$ die Anzahl der Wiederholungen angibt und $fps$ der
  gezeigten Frames pro Sekunde entspricht (Default: $n=1$, $fps=12$). 

\item Speichern des Movies in AVI Format: \alert{\mcode{movie2avi(F,Dateiname)}}
\end{itemize}
\end{frame}

%
% Slide
%


\end{document}






\end{document}

